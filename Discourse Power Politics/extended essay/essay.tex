\documentclass[12pt,a4paper,titlepage]{article}
\usepackage{geometry}
\usepackage{helvet}
\usepackage{harvard}
\usepackage{setspace}
\renewcommand{\familydefault}{\sfdefault}
\geometry{a4paper}
\title{Do you agree with Hardt and Negri's contention that we now live in the age of Empire? Justify your response}
\author{Course Code PO53018A, Student No. 22164733}
\date{\today}

\begin{document}
\harvardparenthesis{none}
\citationstyle{dcu}
\bibliographystyle{dcu}
\maketitle
\doublespacing
\paragraph{}My reading of Hardt and Negri's \textit{Empire}\footnote{\cite{Hardt:2001jl}} has taken me on a personal political journey. The text has provoked a critical examination of my own politics, bringing about a new consciousness of subject and sovereignty, of agency in a hyperglobalised world and of resistance to global capital. To a greater or lesser degree I do support Hardt and Negri's contention that we now live in the age of Empire.

\paragraph{}To justify my position of qualified agreement with these authors I will explore the following four broad themes that emerge from the work: global informational networked capitalism, of subject, sovereignty and supra-nationality, of biopolitical (re)production and of political agency for the authors' subject of liberation, the multitude. This paper continues in three parts. I will first consider the economic order of Empire, examining the antagonism between the biopolitical and capital in global society. Secondly I will examine the post-modern reformulations of the modern political categories of state, subject and sovereignty. The third and final section of this paper is devoted to considering what hope there is for politics in the age of Empire.

\paragraph{}Michael Hardt and Antonio Negri's post-modern narrative erects a theoretical scaffold capable of connecting with contemporary global neo-liberal capitalism. It is an attempt to break from the teleology of historical and materialist Marxisms, yet it is of a piece with Marxist theory in that it is a work of political economy. As every good Marxist will agree, the legal and political order of any given society serves the interests of the economic order in the same. To move beyond Marx, Hardt \& Negri have drawn extensively on the work of Michel Foucault. It is Foucault's reunion of social reproduction, in his terms the biopolitical, with the economic, i.e. material, base that provides the authors of Empire with a critical concept for their analysis. In Empire this concept is developed further, in order to engage with ``the real dynamics of production in biopolitical society.''\footnote{\cite[p. 28]{Hardt:2001jl}} Hardt \& Negri's conception of Empire articulates a global legal and political order serving the interests of global capital. Indeed, as they authors state, ``[t]he first task of Empire, then, is to enlarge the realm of the consensuses that support its own power.''\footnote{\cite[p. 15]{Hardt:2001jl}} The key, global, concensus is that of global capital, for it is in capital that global networked relations are made and sustained.

\paragraph{}With the description of the contemporary character of capitalism as informational, leading to immaterial production\footnote{\cite[p. 289-300]{Hardt:2001jl}} I am in agreement with the authors. `Communicative capitalism'\footnote{After \cite{dean:2005cc}} and its perceived effects; the deterritorialisation, decentralisation and destabilisation of imperial power leading to Empire, is central to Hardt \& Negri's argument. The great role of telecommunications and information technologies in the transition from material to immaterial production, from industrial production to knowledge production, can not be downplayed. Some authors, most noteably Panitch \& Gindin have criticised this informationalist political economy for being based on relatively recent developments in computational power and fear that this approach exhibits a certain techno-determinism.\footnote{\cite{panitch:2003de}} Yet global communication networks need not be understood so uni-dimensionally. For Hardt \& Nergi global communication networks are primarily social and symbolic before they are technical. ``The communications industries integrate the imaginary and the symbolic within the biopolitical fabric, not merely putting them at the service of power but actually integrating them into its very functioning.''\footnote{\cite[p. 33]{Hardt:2001jl}}

\paragraph{}How are we to understand `the imaginary'? The role of images, the imagined, the imaginary in our contemporary global society, as described by Appadurai, is helpful here. For him `the imaginary' is understood as ``a constructed landscape of collective aspirations''\footnote{\cite[p. 31]{Appadurai:1996lp}} and is crucial to understanding the dynamics of networked global flows of people, media, technology, money and ideology. Like the authors of Empire, although anticipating some of their thought by a few years, Appadurai also attempts to move beyond the essentialising epistemology of structuralist approaches to the social world. He argues that ``the imagination has become an organized field of social practices, a form of work (in the sense of both labour and culturally organized practice), and a form of negotiation between sites of agency (individuals) and globally defined fields of possibility.''\footnote{\cite[p. 31]{Appadurai:1996lp}}

\paragraph{}Descriptions of production under capitalism have frequently used biological metaphors. In the imperial age both colonial plantation labourers and urban industrial labourers were counted in units of `hands'. The informational economy continues this theme, now counting `heads' or reckoning up `human capital'. From hands to headcounts the individual is masked by the depersonalising and homogenising units of capitalist calculation. Under contemporary informational capitalism it is not simply the brawn of a human that labours but the brain. The accelerated destruction of both Fordist, i.e. manual, repetitive labouring, and Taylorist, i.e. unthinking labouring, modes of production by communicative capitalism, that is to say the displacing of material labour with immaterial labour, has two consequences. Firstly labour becomes further abstracted, becoming labour power in general. Secondly, the previously hidden dimension of affective labour, the production and manipulation of human emotional states, now comes to the forground. This dimension is crucial as it is within the realm of affective labour that ``social networks, forms of community''\footnote{\cite[p. 293]{Hardt:2001jl}} are produced and reproduced.

\paragraph{}Three types of immaterial labour are distinguished by Hardt \& Negri. The emergence and growth of each are demonstrative of the passage to a post-modern global economy. These are the reshaped instances of industrial production which have embraced communication as their lifeblood; the ‘symbolic analysis and problem solving’ undertaken by knowledge workers, from a Chief Technical Officer to the first-tier of call-centre support; and affective labour, the creation and manipulation of emotional states, found above all within the service sector. \footnote{Adapted from \cite[p. 2]{Wright:2005wb} and from  \cite[p. 30 \& p. 293]{Hardt:2001jl}} Common across each of these labour types are social-interactions conditioned by cooperation rather than by competition, making cooperation ``\textit{completely immanent to the laboring activity itself}.''\footnote{\cite[p. 294]{Hardt:2001jl}, original emphasis}

\paragraph{}The post-workerist approach of Hardt \& Negri forgrounds immaterial labour, yet clearly we are not living in an immaterial world.\footnote{For critical `reality check' on immaterial labour, see \cite{Wright:2005wb}} However it is very important to distinguish between different modes of labour and also to recognise previously `hidden' labour, without which the ecosystem and infrastructure of contemporary global capital would not exist. The gendered and racist historical divisions of labour are two such examples of hidden productive labour. A second important face of the capitalist system exposed by immaterial labour is that of false class-consciousness and class-divisions. The concept of immaterial labour reunites knowledge and affective workers with the material, manual and industrial labourers of the world presenting the chance of an end to inter-class antagonism.

\paragraph{}These changes at the level of the economy demand a new political theory of value formulated to reflect the networked production of wealth and social surpluses. These changes demand a new subjectivity, an ontology that reflects the networked co-operative interactivity of individuals.\footnote{\cite[p. 29 \& p.294]{Hardt:2001jl}} A new post-modern subjectivity will necessarily displace the modern conceptions of the same and will demand a rethinking of the related concept of sovereignty. Any post-modern reformulation of subjectivity and sovereignty must, Hardt \& Negri argue, take into account the communicative nature of (re)production in a post-modern global economy.\footnote{\cite[p. 34]{Hardt:2001jl}} I now turn to the question of how the modern political categories of state, subject and sovereignty are transformed in the passage into Empire.

\paragraph{}Browning, in his critique of \textit{Empire}, notes that for Marx, human beings ``are constituted by their social relations and capital capitalizes on this sociality by extending networks of relations across the globe.''\footnote{\cite[p. 196]{browning:2005gi}} Similarly, much of contemporary globalisation theory is formed from the linking of political economy to a wider web of social relations ``to provide a comprehensive explanation of the present via wide-ranging synchronic and diachronic analyses'' demonstrating an affinity with the grand narratives of Hegel and Marx. \textit{Empire} replays many of the central motifs of globalisation theory, presenting a spatial-temporal reformulation of contemporary global society. Hardt \& Negri argue that the hyper-global inter-networked interdependence of economic, political and legal structures diminishes the status of the nation-state. This has the effect of deterritorialising modern notions of sovereignty based on the concept of the nation-state. Therefore, Hardt \& Negri argue, the modern realist paradigm of international relations is no longer adequate describe the actual global order. The persistant, stable duality of sovereign nation-states vis-a-vis an anarchic international arena had conditioned international relations, both theoretically and in terms of \textit{real politik}, throughout the imperial age of history. Yet, in our contemporary world this duality is defunct, for all points on the political spectrum. The radical interconnectedness of global society demonstrates a degree of order, in contrast to the supposed anarchy, in the international arena. This is the point of departure for Hardt \& Negri: the central ``problematic of Empire is determined in the first place by one simple fact: that there is world order.''\footnote{\cite[p. 1]{Hardt:2001jl}}

\paragraph{}In \textit{Empire} the ``standard genealogies of modern sovereignty''\footnote{\cite[p. 128]{laffey:2005ri}} are rewritten, addressing three specific problems with the `sovereignty narrative'. Firstly, agency is no longer state-centric, it is now immanent in the multitude. Second, the ``intimate relation to racial subordination and colonization''\footnote{\cite[p. 114]{Hardt:2001jl}} crucial to the development of modern European conceptions of sovereignty is foregrounded, ``challenging the militant Eurocentrism of the sovereignty narrative.''\footnote{\cite[p. 128]{laffey:2005ri}} Thirdly, in rehabilitating the concept of `empire' Hardt \& Negri have debilitated the normalising tendency of the modern sovereignty narrative. 

\paragraph{}However, as Laffey \& Weldes argue, Hardt \& Nergi's symptoms of the passage from modern sovereignty to the imperial sovereignty of Empire exhibit a certain undertheorisation. These redefinitions are insufficient to fully displace the modern sovereignty narrative. Indeed, Hardt \& Negri ``remain indebted, both negatively and positively, to the sovereignty narrative.''\footnote{\cite[p. 128]{laffey:2005ri}} Firstly, although the passage to the imperial model of Empire is not defined in ``purely negative terms''\footnote{\cite[p. 13]{Hardt:2001jl}} the authors of \textit{Empire} do have a negative dependency on the decline of modern political categories. Secondly, whilst the Eurocentrism of these modern political categories is weakened by the concept of Empire, the question of ``what do we do about the United States''\footnote{\cite[p. 130]{laffey:2005ri}} remains unsettled: this question is crucial to many criticisms of the passage to Empire and made all the more important in the light of US `imperialist' military activity. That the US was instrumental in preparing the ``political forms and terrain of Empire'' and now ``sits atop the imperial pyramid of power''\footnote{\cite[p. 131]{laffey:2005ri}} that is Empire and yet is not the `new Rome' is, for Laffey \& Weldes, a contradictory and unsatisfactory conclusion, however precarious and temporary the position of hegemon in this new paradigm of sovereignty.

\paragraph{}According to Hardt \& Negri, in ``the passage from modern to postmodern and from imperialism to Empire there is progressively less distinction between inside and outside.''\footnote{\cite[p. 187]{Hardt:2001jl}} Laffey \& Weldes respond with the criticism that Hardt \& Negri have not paid ``careful attention to the actual relations of rule through which European empires governed'' at the expense of the ``diverse relations not explicable in terms of the simple inside/outside logic of modern sovereignty.''\footnote{\cite[p. 132]{laffey:2005ri}} In normalising Empire as the post-modern tendency that emerges from modern sovereignty, the former in effect legitmises the dubious constitution of the latter. Perhaps this is the case, but what is more important here, in the transformation of imperialism to Empire, is that the ``modern dialectic of inside and outside has been \textit{replaced} by a play of degrees and intensities, of hybridity and artificiality.''\footnote{\cite[pp. 187-8, my emphasis.]{Hardt:2001jl}}

\paragraph{}This debate between our authors and Laffey \& Weldes essentially comes down to two differences. The first is in their different methodological approaches to theorising both the state and sovereignty. \textit{Empire} is a post-structuralist autonomist reading of the state whereas Laffey \& Weldes emphasise a more genealogical approach. A second difference is demonstrated in the authors different theoretical responses to globalisation. Laffey \& Weldes strengthen the state through `internationalization' as a response to globalisation. Hardt \& Negri, in contrast, diminish the state to refrains of the \textit{Internationale}.

\paragraph{}At this point we should go back a little in order to understand the trajectory of the \textit{Two Europes, Two Modernities}\footnote{\cite[Ch. 2.1 pp. 69-92]{Hardt:2001jl}} Hardt \& Negri describe in \textit{Empire}. The relationship between new paradigm of global sovereignty that is Empire and the global political subject, the multitude, is grounded in the history of struggle between capital and labour and intertwined with the growth of the nation-states of Europe as stable political entities. As Peter Green notes in his commentary \textit{The Passage from Imperialism to Empire}, Hardt \& Negri ``repeatedly emphasise that the emergence of the nation-state system and `European modernity' are inseparable from capitalism.''\footnote{\cite[p. 37]{green:2002em}} The relationship between labour, capital and state is theorised in \textit{Empire} as a revolutionary/counterrevolutionary struggle. The revolutionary European modernity ``destroys its relations with the past and declares the immanence of the new paradigm of world and life.''\footnote{\cite[p. 74]{Hardt:2001jl}} Hardt \& Negri continue\ldots

\begin{quote}
``The new emergence, however, created a war. How could such a radical overturning not incite strong antagonism? How could this revolution not determine a counterrevolution? There was indeed a counterrevolution in the proper sense of the term: a cultural, philosophical, social, and political initiative that, since it could neither return to the past nor destroy the new forces, sought to dominate and expropriate the force of the emerging movements and dynamics''\footnote{\cite[p. 74]{Hardt:2001jl}}
\end{quote}

\paragraph{}The revolution of the multitude declares immanence the new paradigm of world and life. The counterrevolutionary modernity, serving the interests of the bourgeois ``poses a transcendent constituted power against an immanent constituent power, order against desire.''\footnote{\cite[p. 74]{Hardt:2001jl}} That the transcendent constituted power of the nation-state is intimately and inextricably intertwined with capital is the principal reason for Hardt \& Negri rejecting the `internationalisation of the state' that Laffey \& Weldes propose. Indeed, any anti-globalisation or anti-capital resistance ``founded on the identities of social subjects or national or regional groups'' tends towards the ``\textit{localization of struggles}'' and is rejected by our authors.\footnote{\cite[p. 44, original emphasis]{Hardt:2001jl}} The lyrics of the \textit{Internationale} expressed this sentiment long before Hardt \& Negri:

\begin{quote}
\textit{There are no supreme saviours, Neither God, nor Caesar, nor tribune; Producers, let us save ourselves, Decree the common welfare; That the thief return his plunder, That the spirit be pulled from its prison; Let us fan the forge ourselves, Strike the iron while it is hot; This is the final struggle, Let us stand together, and tomorrow; The Internationale, Will be the human race} \footnote{This is the literal English translation of the second stanza, sourced from Wikipedia. Accessed Apr 17th 2008}
\end{quote}

\paragraph{}The workerist hymn of the global proletariat becomes the post-workerist hymn of the global multitude through our authors' conceptualisation and reconciliation of immaterial labour with material labour, uniting all biopolitical producers. Hardt \& Negri recognise in the \textit{Internationale} the radical response to the sovereignty of global capital: the end of the insider / outsider division of peoples, that which was enforced by the transcendent power of the nation-state, through a revolutionary humanism. As capital becomes increasing pervasive as a global force, so too must the subject of Empire, the multitude, respond, resist and realise its unity as a global force. Empire is the response, the counterrevolution, to proletarian internationalism.\footnote{\cite[p. 51]{Hardt:2001jl}} With the continued globalisation of capital and the structures of law that govern in its interest comes the possibility of further revolutionary change. The deterritorialization of modernity's structures of exploitation and control is a positive benefit for the multitude, creating the conditions for further - final? - liberation.

\paragraph{}Hardt and Negri's hope for emancipation is found in the immanent power of the multitude, the agent with ``\textit{the will to be against Empire}.''\footnote{\cite[p. 210, emphasis in original]{Hardt:2001jl}} The figure of the multitude displaces the Marxist subject of liberation, the proletariat, in the age of Empire through the connection of immaterial labour with material labour, thus uniting all biopolitical producers.

\paragraph{}Browning is critical of Hardt \& Negri's figure of the multitude.\footnote{}He argues that the ``emancipatory potential of the multitude, like that of the proletariat, is shaped by preceding diachronic conditions that have opened up the global possibilities of freedom, while Hardt and Negri maintain, like Marx, that the logic of emancipation precludes determination of the course of revolutionary action.''\footnote{\cite[p. 195]{browning:2005gi}} The critical question here is `what of the politics'? Browning highlights that the ``unity amidst diversity that is characteristic of [E]mpire and its antagonist, the multitude, \textit{admits no clear mode of discrimination}.''\footnote{\cite[p. 198]{browning:2005gi}, emphasis added} So what does resistance, or any sort of politics for that matter, look like in the age of Empire? Politics, social struggle, resistance - each require objectives as a basis for action. However, against the totality of Empire, as described by the authors, these objectives are made parochial, any strategy self-ghettoising. Is \textit{Empire} anti-political as Laclau charges?\footnote{\cite{laclau:2005ss}}

\paragraph{}Schmitt's expression of a modern conception of `the political' is one of the purest.\footnote{\cite{schmitt:2007cop}} His friend / enemy distinction mirrors the inside / outside logic of modernity: each admits clear politically determined discrimination. For Marx, as Hardt \& Negri remind us, ``the relationship between the inside and the outside of capitalism's development is completely determined in the dual standpoint of the proletariat, both inside and outside capital.''\footnote{\cite[p. 209]{Hardt:2001jl}} The relationship between the multitude and Empire appears to be a continuation of the dialectic between labour and capital. Yet, Hardt \& Negri's key argument is that the hyper-global inter-networked interdependence of people and of global capital has deterritorialised the boundaries of modernity. The radical variation in the universal unity of the multitude and of Empire forms a mesh of difference, a networked non-place - ``the dialectic between productive forces and the system of domination no longer has a \textit{determinate place}.''\footnote{\cite[p. 209]{Hardt:2001jl}} Remember, the ``modern dialectic of inside and outside has been \textit{replaced} by a play of degrees and intensities, of hybridity and artificiality.''\footnote{\cite[pp. 187-8, my emphasis.]{Hardt:2001jl}}

\paragraph{}For Laclau this is presents serious gaps in our authors' argument. Firstly, the hybridity and artificiality of the multitude presents a ``proliferation of a plurality of identities and points of rupture [making] the subjects of political action essentially unstable.''\footnote{\cite[p. 7]{laclau:2005ss}} How are these subjects going to articulate action when they are radically unstable? A second criticism from Laclau is that the deterritorialization and virtualisation central to \textit{Empire} undermines the political program, articulated as the rights and demands of the multitude, put forward by Hardt \& Negri towards the end of the book. He writes that ``both demands and rights have to be recognized, and the instance for whom that recognition is requested cannot be in a relation of \textit{total exteriority vis-a-vis the social claims}.''\footnote{\cite[p. 9]{laclau:2005ss}, emphasis added} The recognition of the rights of the multitude, the fulfillment of the demands of the multitude, the multitude becoming a political subject, each raise questions that Hardt \& Negri do not adequately address in the book: the task of the multitude remains rather abstract.\footnote{The task of the multitude became the point of departure for their follow-up publication, \textit{Multitude} (\cite{Hardt:2005zt}). See also \cite{brown:2005cs}, \textit{What Is The Multitude?}}

\paragraph{}Returning to the question of Hardt and Negri's contention that we now live in the age of Empire. My position is that of a qualified agreement with the authors. With their description of a tendency towards the deterritorialisation, decentralisation and destabilisation of the sites and structures of power by the actions of both global capital and the multitude around a fulcrum of counterrevolution/revolution I am in agreement. This aspect of their argument, I find, is a compelling explanation of our present conjuncture. To this extent then I agree with our authors contention, demonstrated by various symptoms of a passage, of a tendency towards an age of Empire. What concerns me, leading to my qualified agreement, is that the relationship of subjugation between the multitude and Empire, and the possibility of the multitude's liberation appears at the same time both overdetermined and undertheorised by Hardt \& Negri.
\bigskip
\paragraph{}Words : 3143
\newpage
\singlespacing
\harvardyearparenthesis{round}
\bibliography{/home/robd/projects/robdyke/goldsmiths/Bibliography/globalbib}

\end{document}