\paragraph{}It will help if I first breakdown and clarify the three terms I'm using; governance, network and society. In asking `who governs' I am enquiring into the protection and reproduction of order and meaning for a community. By using the ambigous subject `community', I seek to demonstrate the multiple and varied sites of governance, from concrete neighbourhoods of physical co-presence to more abstract associations and aggregations of interest such as the modern concept of the state. By network, I am refering to interconnected groups or systems. To be in a network implies there being `more than one' and expresses an interface for some kind of communication. By society I am refering to the myriad of associations and groupings, institutions, regulatory bodies and governments for, from and of our communities. Most crucial to my purpose here is the movement of information between these complex systems, hence `networked society'.

\paragraph{}I'm being deliberate with my words here in my choice of `community' as I am trying to avoid deploying the concept of the nation-state this early on. Not out of any prejudice to the nation-state, rather that in its contemporary configuration the nation-state is somewhat diminished in status when compared to its imperial age. bounding concept, delimites realm of soverignty, static (relatively) At the same time, `community' reflects the fluid nature of membership of and communication between social systems.

\paragraph{}Modern liberal-democracy comes under increasing pressure in a networked society. The balance between those hyphenated concepts is delicate, leaving little but breathing space. Political liberalism's freedoms for the self-governing individual are curtailed by democratic governance mechanisms, mobilised by those same individuals, now sovereignly constitutedin political communities. Yet democracy's 'people', liberated from being cogs, now appear as little more than figures watching the machine they have been freed from continue its work without them. What was perhaps once ideal now seems like a bad idea. Worryingly it appears that these weakly linked concepts, liberal-democracy, are the only ideas left. I will critically examine the nature of govenance in liberal-democracy, most crucially considering the practical meaning of the hyphen that usually links them in our contemporary context. Closure characterizes the liberal-democratic political culture, a closure to anything prior to the acceptance of the liberal-democratic consensus \cite{Mouffe:2000fk}. The liberal individual is a universalising concept, as is the expansive language of rights. There is no `outside' to liberal-democracy.

\paragraph{}So I will examine the nature of modern political liberalism and democracy, considering the constituency of `the people' with regards to the principle of popular sovereignty and the diminishing of demos through universalized rights. I will examine the tension in the hyphen between  liberal-democracy using the discourse analytical method of \citeasnoun{Mouffe:2000fk}. Drawing on her work in the \emph{The Democratic Paradox} with that of \possessivecite{Keenan:2003tw} \emph{Democracy in Question} I will look for opportunities for radical `democratic openness in a time of political closure'.

\paragraph{}Returning to Dahl's examination of democractic government in New Haven. What Dahl observed in the 1960s was that the governance of liberal-democratic political cultures had shifted gradually towards plurality, gradually leaving behind the earlier oligopolistic modes of the recent history. If we commence this narrative further back in history and in Europe rather than America, we would see the shift from the monarchical order towards the aristocratic and much later oligopolistic society of the turn of the century. In tandem with these changes in modes of governing come changes in the modes of production.

\paragraph{}\textit{Recent examples of such perverse prioritisation can be seen in one weeks `news': The USA Federal Reserve Bank steps in to the `free market', providing billions of dollars of cheap money for private interests; I wonder how much support the Tibetans or the Sudanese or even Americans without health insurance will be given this coming week, and how pittiful it will appear in comparison to the bonuses paid to the Executives of American banks, rewarded for failure. Commercial and corporate interests are not, despite rhetorical and legal moves to the contrary, individuals; they are, at most, aggregations of private interest, not public interest. The intrinsic self-serving nature of global capital must not be forgotten!}

\paragraph{}Empire seeks to protect the interests of global capital so it is the logic of global capital which governs this supranational sovereignty. It is through the rubric of Empire that we can see the politicisation of areas of the social world at the sites where NSMs/NGO's activities take on a new status, providing the moral legitimation for the use of force in the paradigm of Empire.\footnote{\textit{Ibid}., pp. 34-36 \& pp. 312-314} These NGO's apparent altrustic and humanist crusades against the ills of global society take on a new function as the moral conscience of the supranational sovereignty in Empire. These humanitarian organisations are, for Hardt \& Negri, among the ``most powerful pacific weapons of the new world order''\footnote{\textit{Ibid}., p. 36} forming the broad base of the new global constitutional order.\footnote{\textit{Ibid}., pp. 309-314} They write that in Empire NSMs/NGOs represent the ``vital force that underlies the People, and thus they transform politics into a question of generic life, life in all its generality\ldots Here, at this broadest, most universal level, the activities of these NGO's conincide with the working of Empire ``beyond politics,'' on the terrain of the biopower, meeting the needs of life itself.''\footnote{\textit{Ibid}., pp. 313-314. Biopower is Hardt and Negri's Foucaltian term for the social and the biological, that is to say human social reproduction in Empire.}