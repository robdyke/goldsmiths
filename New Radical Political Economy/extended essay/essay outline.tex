\documentclass{article}
\usepackage{geometry}
\usepackage{helvet}
\usepackage{harvard}
\usepackage{setspace}
\renewcommand{\familydefault}{\sfdefault}
\geometry{a4paper}
\title{To what extent do social, peer-2-peer or participatory financial models democratize capital?}
\author{Rob Dyke, No. 22164733}
\date{\today}

\begin{document}
\begin{doublespacing}
\citationstyle{dcu}
\bibliographystyle{kluwer}
\maketitle

\paragraph{}What are the principle themes that are problemitised by this question that I will examine in this essay? In terms of new radical political economy I am considering the anti-capitalist criticism of capitalism as being inherently undemocratic and dehumanising. What do I mean by social, peer-to-peer and participatory models? How are these approaches different to dominate financial services? Are we seeing a resocialisation of banking back to the good old Baily Savings and Loan model\footnote{James Stewart's \emph{It's a Wonderful Life}}?

\paragraph{}Contemporary financial models are criticised for a lack of transparency, impersonal relationships, heavy systematisation and structural exclusion from services through credit scoring. These criticisms of banking institutions and debt money are magnified at the international level.

\paragraph{}I will examine the model and the nature of captialism and economic democracy with specific refernce to banking institutions. To what extent can capitalism be democratic? To what extent can any economic system be democratic? In terms of democracy I am thinking both in terms of government regulation and in terms of access to capital and finances services, for example bank accounts and finance capital. I will explore the crucial purpose of banking - that is, to create money from debt, the making of money from money - and will examine the possibilities for and consequences of the (re)socialisation of these transactions. As a contrast to private banking models I will explore traditional social lending models; co-operatives, mutual socities, building socities. These fading models, some long since capitalised and floated on stock-markets, will be contrasted to contemporary peer-to-peer models.

\paragraph{}I will present two case studies of `social' financial models. The first will be of Zorba, a peer-to-peer bank, a virtualised savings and loan, providing banking as a service (credit scoring and factored/mediated risk) with the decisions, the risk/reward speculation, being made by customers. Being a peer-based model it is customer driven and horizontal in structure. I will constrast the relationships configured by this model to those of traditional banks.

\paragraph{}My second case study will look at Open Capital. Chris Cook's ideas demonstrate that participatory economics can scale to infrastrucutre projects of a significant size.  Limited Liability Partnership model for community infrastructre projects

\paragraph{}I may also draw on secondary examples of participatory financial models such as Local Economic Trading Schemes and community currency projects, micro-credit projects for aid and development and also participatory budgeting projects in UK.

\end{doublespacing}
\end{document}