\documentclass[12pt,a4paper,titlepage]{article}
\usepackage{geometry}
\usepackage{helvet}
\usepackage{harvard}
\usepackage{setspace}
\renewcommand{\familydefault}{\sfdefault}
\geometry{a4paper}
\title{``There is no such thing as a `communitarian' critique of liberalism for such `critiques' are in reality only expressions of liberal dissatisfaction with the excesses of liberal individualism'' Discuss.}
\author{Student Registration No. 22164733}
\date{\today}

\begin{document}
\harvardparenthesis{none}
\citationstyle{dcu}
\bibliographystyle{dcu}
\maketitle
\doublespacing
\paragraph{}After an eleven year absence, the manifestation of all that was 1990s Girl Power, the Spice Girls, reformed recently for a world tour. Once again the emancipatory message of `girl power' and the undefinable yet absolutely essential `zig-a-zig-ah' was heard blasting out of stadiums. Although this time to much smaller, and now older audiences. The tour was cut short due to `personal reasons.' In the course of this paper it will become clearer why the Spice Girls, `girl power' and `zig-a-zig-ah' are apt cultural references to begin a paper on political theory. For the moment, please indulge me.

\paragraph{}Michael Walzner's introductory remarks to his retrospective of communitarian criticisms of liberalism describes the brief but reoccurring lives of intellectual fashions.\footnote{\cite[p. 6]{walzer:1990cc}} The initial intensity of the early communitarian critiques has subsided, a demonstration, perhaps, of the resilience of liberal political philosophy. Yet criticisms of liberal individualism resurface from old and new perspectives. This paper will consider the various arguments with liberal theory and liberal practice that communitarian critics hold. My view is that there are indeed a number of strong communitarian critiques of liberal individualism, critiques that are more than an expression of dissatisfaction by disaffected liberals. That said, like Walzner, I recognise the often contradictory relationship between different communitarian critiques. I also acknowledge that to a certain degree communitarian critiques are dependent on the liberal individual ontology, the very subject being criticised, to frame arguments. Yet I refute the assertion of Cohen who considers this ``a deep contradiction in communitarian thought.''\footnote{\cite[p. 283]{cohen:2000cr}} Furthermore, such communitarian critiques of liberalism are not framed constructively; that is to say, certain communitarian critiques do not have a better solution, only an clear sight of `the problem' which I find both problematic and disappointing.

\paragraph{}This is not, however, true of all communitarian critiques. The call to a virtuous, civic republican life from Alasdair MacIntyre\footnote{\cite{MacIntyre:1981lr}} will be considered, as will the radical democratic political theory of Chantel Mouffe who also draws somewhat similar conclusions, though from different premises, and in pursuit of different goals.\footnote{\cite[Chapters 2, 3 \& 4]{mouffe:2005rp}}

\paragraph{}My first task is to sketch an outline of the different communitarian critiques of liberal political theory. The reoccurring intellectual `fashion' of `communitarian critiques' has taken a number of different turns over the course of the 20th century. Liberalism and some of its critics exist symbiotically and the early communitarian criticisms, published in the 1970s, reflect this relationship.\footnote{For an introduction and bibliography of early communitarian writing, see \cite[pp. 1-20, esp fn 14]{bell:1996cc}} The grounds for these early communitarian critiques was epistemological. The twin threads of these critiques were the ontological basis of the liberal individual and the liberal individualistic morality.

\paragraph{}The concern of thinkers critical of the ontological grounds of liberal theory is the primacy given to the individual over and above that of society.\footnote{\cite[Scene 1]{bell:1996cc} \& \cite[p. 8]{walzer:1990cc}} This ontology is highly individualistic; the liberal subject is liberated from social bonds that are seen as restrictive of freedom. This particular philosophy of the individual holds that membership of a community is not ascriptive of identity. The individual is primary and stands before the social order. This liberal individual is unconstrained, represented as egoistic and autonomous over the all decisions and actions that are part of the course of living one's life. The epistemology of the liberal individual is if not chaotic, then certainly arbitrary. A communitarian critique of this liberal individual is concerned firstly with the isolated state that such an unconstrained being finds his or herself in and secondly with the disintegration of society that results from such atomised individuals. Yet against this charge of increasing entropy in the social world, the liberal ontology can account for the existence of the communities of association humans create and sustain for ourselves. The liberal individual recognises and chooses his or her membership of a community, their consent gives the association legitimacy.

\paragraph{}A communitarian critique of liberal morality centred in the individual is that this liberalism threatens the actual social order by its categories. Liberalism espouses freedom, yet it can not tell us what this freedom is for. The inheritance is rich when it comes to rejecting subjugation, fighting absolute and arbitrary power and in universalising freedoms. Yet liberal political theory is impoverished when it comes to describing what this freedom is for. The `zig-a-zig-ah' of the Spice Girls, melts into air: they really, really want this `zig-a-zig-ah' but its instrumentality, its utility to them can be neither questioned nor explained. Liberal individualism is criticised by communitarian's for prizing something greatly yet not knowing its value. The communitarian critique has a valid point here.\footnote{See \cite[Ch. 11]{bellah:1985hh}} We, as liberal individuals, can not make discursive sense of our lives. This inability to give voice to the purpose of our lives tends, if not towards nihilism, but at least towards a precarious social existence. Liberal political theory can elaborate rationally the requirement of freedom, autonomy, \&tc. Yet it falls silent when it comes to explaining what this liberty is for because of a rejection of any teleology.

\paragraph{}The liberal individual examined by these preceding critiques is a philosophical figure. In contrast, the real lived lives of individuals in liberal societies are socially situated, context dependent and possessed not only of the formal freedoms of liberal theory but with families, communities and ties of association. Again the Spice Girls are helpful. The concert tour was cancelled for personal reasons, their zig-a-zig-ah now made secondary to their social ties.

\paragraph{}The first two critiques presents liberalism as a triumphant failure, the first because it leads to the atomisation of society due to rampant individualism, the second becuase of the precarious existence these individuals live without a meaningful vocabulary to articulate their social bonds. Yet both of these critiques are in effect praising liberal individualism with their faint damns. These two critiques could be considered to be from those dissatisfied and disaffected with the consequences of the \textit{excesses} of liberal individualism, not liberal individualism \textit{per se}. Actual liberal societies are not as entropic as the first critique which is perhaps only highlighting ``the truth about the asocial society that liberals create''\footnote{\cite[p. 7]{walzer:1990cc}} Furthermore, the rejection of teleology by liberal theory, the problem for the second critique, does not extinguish other structures of meaning. The lived lives of individuals reflect the continuance of other socially situated structures of meaning and purpose, such as religious beliefs and political cultures outside of the liberal paradigm.

\paragraph{}However, as Walzer affirms, ``each of [these] two critical arguments is partly right.''\footnote{\cite[p. 11]{walzer:1990cc}} There has been a relative increase in dissociation and dislocation of individuals in liberal societies, as tracked by Walzer's `Four Mobilities'. His observation feeds into what neighbouring socio-political theorists have described as the rise of a post-materialist culture which has contributed to the emergence of New Social Movements (NSM).\footnote{\cite{Eder:1993fj} and \cite{Dalton:1990zr}} Liberal political philosophy freed individuals (and capital, although that's a different story\ldots) from various social, political, marital and geographic bonds. These individuals, once free to act, pursued and realised significant improvements in their material existence. The first communitarian criticism, that of an atomised society of egotistical individuals, expresses a dissatisfaction with the theoretical liberal individual. Yet the contemporary liberal individual, with their material well-being assured, now develops concerns beyond the self. The growth of New Social Movements concerned with ecology and peace, two examples of undeniably social concerns beyond that of the egocentric liberal individual, demonstrates the sociability of individuals.

\paragraph{}New Social Movements theories are built on the recognition of the historicity of individuals and the deeper social structure that liberal political theory denies in its ahistorical and autonomous individual. The perceived atomisation of society, and the rise of a post-material culture, at least in the developed west, has not dislocated each individual from their undeniably social roots. Individuals can talk to each other and make sense of each others lives. The growth of the memberships of different NSMs and the collaboration between such aggregations of communal interest, demonstrate an ability to share discursively experiences and to motivate individuals for shared goals.

\paragraph{}The extent of NSM theories ability to account for the real stability and cohesion in society is limited. Multiple and fluid memberships of multiple associations and communities to not make a stable and coherent society. However strong the motivations of liberal individuals, their moral and ethical language is still in a state of lack when in comes to describing the purpose of the individual and that of the society created by such associations and groupings. When conflict arrises within these associations we are ill-equiped to resolve it.

\paragraph{}The rejection of universal telos and morality in favour of a paradigm based on justice and universal rights as articulated by neo-Kantian libertarians such as Rawls\footnote{\cite{rawls:1972tj}} and later Nozick\footnote{\cite{Nozick:1974lr}} is the grounds of a communitarian critique of liberalism from Sandel.\footnote{See also \textit{Justice as a Virtue: Changing Conceptions} in \cite[pp. 244-255]{MacIntyre:1981lr} for a criticism of the modern liberal paradigm of justice and rights.} This is a crucial criticism of the more contemporary liberal theory which considers liberalisms philosophical figures and its categories as substantially flawed. The claim that the rights of individuals must stand in priority over the common good of society has, explains Sandel, a ``deep and powerful philosophical appeal.''\footnote{\cite[p. 82]{sandel:1984pr}} Yet, as he goes on to demonstrate, the ``failure'' of its philosophical figure, the ``unencumbered self'', and, indeed, the very claim of the priority of the rights of an individual over the common good of a community, has not prevented this vision of liberalism becoming ``the one by which we live by.''\footnote{\textit{Ibid.}} Sandel shows that liberal justice and rights, while rhetorically universal, are, in fact, local and dependent on a community for recognition and meaning. He goes on to demolish Rawls's `unencumbered self': if liberal justice and rights can only have meaning within a specific community, then this community must first exist and second have `goods' that it chooses to promote. Sandel's communitarian argument of the priority of `goods' over `rights' withstands the libertarian counter-ordering.\footnote{As GK Chesterton said `To have a right to do a thing is not at all the same as to be right in doing it'} His is a deeper reading of liberal theory, one that uncovers the aporias in its individualistic philosophy, shaking liberalism to its very foundations.

\paragraph{}This is a critique which has the ability to displace liberal political theory. Sandal's critique of the modern refoundation of liberal theory by Rawls demonstrates the categorical abstraction required by liberal theory - the liberal individual is dispossessed of their individual unique attributes and separated from their nature, their social world, in order for liberal theory to have any purchase from its description. The abstraction of the individual and the social order in Rawls' and Nozick's work dislocates individuals from themselves, from the choices they make and the communities they form. This has serious consequences for politics, as Chantel Mouffe as made clear.\footnote{\cite[p. 65]{mouffe:2005rp}} What these neo-Kantians fail to recognise in demanding the priority of individual rights over the common good is that this ``can only exist in a certain type of society with specific institutions and that it is a consequence of the democratic revolution.''\footnote{\textit{Ibid}., p. 64} That is to say that neo-Kantian liberals fail to recognise the historicity of liberalism, the context of the emergence of liberal political theory from the struggle against arbitrary and absolute authority.

\paragraph{}Before continuing with Mouffe and her call for a reformulation of the communitarian critique of liberalism, I will consider one further critique of liberal political theory. The critique of liberal political theory, indeed of the whole Enlightenment project of which liberalism is the favoured off-spring, of Alasdair MacIntyre goes further than any other. MacIntyre's criticism of liberalism is not a `communitarian' critique for, as we have seen, such `critiques' often only expressions of liberal dissatisfaction with the excesses of liberal individualism. MacIntyre argues that the political and moral standpoints which are typical of liberal modernity are the result of the failure of the `enlightenment project' to establish a grounds for morality in either reason or feelings. The rejection of both transcendental and normative grounds by the `enlightened Moderns' for their explanations of human nature is the cause of the decay and lack in contemporary liberal societies. Where the period we call the Enlightenment did accept normative grounds for morality and science, this was only in the sense that it gave primacy to the European world view of universal reason. For MacIntyre, the mistake of the theorists of modernity seeking to establish a rational and universal explanation of `human-nature-as-it-is' was declining to comment on `human-nature-as-it-could-be-if-it-realized-its-\textit{telos}.'\footnote{\cite[p. 53, orig. emph.]{MacIntyre:1981lr}} MacIntyre argues that this is becuase they could not. These theorists have inherited only fragments of Ancient traditions and other social structures of purpose and meaning, left behind by the revolutions of the Enlightenment and early modernity so was not available to commented on. The rejection of telos meant the rejection of ``the precepts of rational ethics as the means for the transition from one to the other.''\footnote{\textit{Ibid}.} This creates the moral problems of modernity, of the modern liberal individual and of liberal society. As MacIntyre writes,

\begin{quote}
``On the one hand the individual moral agent, freed from hierarchy and teleology, conceives of himself and is conceived of by moral philosophers as sovereign in his moral authority. On the other hand the inherited, if partially transformed rules of morality have to be found some new status, deprived as they have been of their older teleological character and their even more ancient categorical character as expressions of an ultimately divine law.''\footnote{\cite[p. 62]{MacIntyre:1981lr}}
\end{quote}

\paragraph{}This, MacIntyre argues, tends either towards utilitarianism in the first example and, in the second, towards the ideas of those like Rawls, who, after Kant, attempt to establish universal grounds for ethical action in reason. Both attempts, says MacIntyre, ``failed and fail.''\footnote{\textit{Ibid}.} He goes on to argue that it is not possible to prove universal and rational grounds as this rejects historicity, context and circumstance. MacIntyre is more Hegalisn than Kantian in this respect. There can be no presuppositionless rationality, just as there can be no `unencumbered self'. As for natural rights, or rights qua human, MacIntyre finds these concepts fictitious, fictions yet with ``highly specific properties''.\footnote{\textit{Ibid}., p. 70} The purpose of these such idioms is to ``at best provide a semblance of rationality for the modern political process, but not its reality.''\footnote{\textit{Ibid}., p. 71} The reality is a mock rationality which ``conceals the arbitrariness of the will of power at work'' in the political process.\footnote{\textit{Ibid}.}

\paragraph{}MacIntyre posits that certain pre-enlightenment political traditions hold a shared moral \& ethical language and recognises the boundaries of communities of understanding. This is a more complex approach than that of the authors of \textit{Habits of the Heart} who recognised the presence of fragments of earlier moral languages in contemporary society. MacIntyre calls for a synthesis of an Aristotelian model of political activity with the ethics and morality of St Benedict; for us to retrace our steps, before the `enlightened' turn which rejected human telos, and set out on our journey again. He urges us to establish ``local forms of community within which civility and the intellectual and moral life can be sustained through the new dark ages which are already upon us.''\footnote{\textit{Ibid}., p. 263}

\paragraph{}MacIntyre sees in the Ancients the nourishment that will remedy the lack at the heart of the Moderns. Yet I see no need for us to travel back with him in his attempt to revive classical political and ethical models. I am a liberal, a child of this age. I value the separation between the realms of morality and of politics that the enlightenment has bestowed on us. Yet I concede that this has led to an instrumentalist conception of politics, a politics devoid of ethics, and resulted in the normative concerns of the political subject, the liberal individual, relegated to the status mere values.

\paragraph{}So what then of politics in our liberal age? Do we really have to choose between the liberty of the Ancients and the liberty of the Moderns? As Mouffe argues, ``we should not accept a false dichotomy between individual liberty and rights'', i.e. the choice of the neo-Kantians, ``or between civic activity and political community'' as Sandal and, to a certain degree, MacIntyre would have us choose.\footnote{\cite[p. 65]{mouffe:2005rp}} 

\begin{quote}
``Our choice is not only one between an aggregate of individuals without a common public concern and a pre-modern community organized around a single substantive idea of the common good. To envisage the modern \textit{democratic} political community outside of this dichotomy is the crucial challenge.''\footnote{\textit{Ibid}., p. 65, emphasis added to draw attention to her project}
\end{quote}

\paragraph{}Perhaps Mouffe and MacIntyre are closer than I first thought. Perhaps his ``local forms of community'' are a manifestation of the radical democratic political community sought by Mouffe.\footnote{There is however insufficient space to explore this idea further within the scope of this paper.} While MacIntyre returns to a civic republicanism based on that of Athens, Mouffe posits a radical democratic citizen along the lines of an Oakeshotian \textit{respublica}.\footnote{\textit{Ibid}., p. 72} The civic republicanism proposed by Mouffe is that of a radically democratic republic, built from but not returning to the Ancients, a republic with a decidedly \textit{post}-modern character, which makes use of the ``symbolic resources'' of liberal political theory and the democratic tradition, rather than rejecting them. \footnote{\textit{Ibid}.} Whether this can be acheived without incoherance will perhaps be the subject of a revival for the intellectual tradition of `communitarian critique'.

\newpage
\singlespacing
\harvardyearparenthesis{round}
\bibliography{/home/robd/projects/robdyke/goldsmiths/bibliography/globalbib}

\end{document}