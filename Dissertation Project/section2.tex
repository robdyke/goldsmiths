\newpage
\chapter{The Free/Open Community.}
\paragraph{}The networked meshing of hackers, artists and activists has made public free/libre, open-source and creative-commons concepts.\footnote{I will use f/los when referring to both free/libre and open-source together in this paper. A more common acronym, FLOSS, demonstrates the genealogy of the term and its origins in computing.} This section will consider the hacker identity and hacker social and cultural relations. I posit that increasing contact with f/los \& creative-commons artifacts, technologies and economics and the wider dissemination of f/los \& creative-commons memes is raising popular consciousness of several emerging redefinitions for concepts such as freedom, property, production and ownership. Freedom and openness are two central concepts in hackerism and the contemporary social movements it has influenced. This section examines critically the political philosophies of hackerism in order to reveal the meaning of these emerging redefinitions of freedom and openness and the challenges they present to the corresponding dominant liberal notions of the same terms.

\paragraph{}To hack is to play, to learn, to survive.\footnote{\cite[pp. 48-53 and the prologue `Linus's Law']{himanen:2001he}} The hacker supports his/her own being through the application of his/her individual characteristics in relation to his/her social and material conditions and gains pleasurable satisfaction in doing so. As MacKenzie Wark puts it in \textit{A Hacker Manifesto}, the hacker ``produces the possibility of production, the possibility of making something of and with the world - and of living off the surplus produced by the application of abstraction to nature - to any nature.''\footnote{\cite[077]{wark:2004hm} There are no page numbers in Wark's \textit{Manifesto}, simply numbered points/demands. These appear in the place of page numbers in my references.}. The hacker disposition is demonstrated in the real, lived lives of humans and the development of all human civilisations. What I am suggesting by ascribing these characteristics to all humans qua human is that the hacker is the innate modality of humanity. Hacking, according to Wark, discovers ``the nature of nature, its productive - and destructive - powers. This applies as much in physics as in sexuality, in biology as in politics, in computing as in art or philosophy. The nature of any and every domain may be hacked. It is the nature of hacking to discover freely, to invent freely, to create and produce freely.''\footnote{\cite[076]{wark:2004hm}. In \textit{Empire} Hardt and Negri argue that capital has a similarly expansive character. Just as the hacker sees no boundary to what may be hacked, capital similarly has no boundaries, expanding into culture, science, sex.}

\paragraph{}To hack is also to subvert and resist. The desire to hack the information society emerges negatively, in opposition to the challenge of increasingly rapid enclosure of the immaterial and cultural commons and against the dislocation from history and society that virtual and networked property relations effect. Indeed, as both Wark and Himanen have argued, it is the hacker who is capable of answering the property question in capitalist societies by proposing an ``alternative spirit for the network society''\footnote{\cite[p. 12]{himanen:2001he}} creating ``new kinds of relation, with unforeseen properties, which \textit{question the property form itself}''\footnote{\cite[036, emphasis added.]{wark:2004hm}}. This presents a challenge to the dominant spirit of contemporary advanced capitalism. The hacker proposes free production and free access in contrast to the private property of capital. I will return to these new kinds of relationships later.

\paragraph{}Modern freedom, the liberty of contemporary liberals, emerged negatively, as a revolution against the dominant absolutism and authority of the pre-modern period. The freedom of free/libre also emerges negatively, a revolution against the widespread practice of enclosure of technologies and the encroachment of enclosure in the academy. In \textit{Revolutions OS}\footnote{\cite{moore:2002rv}} Richard Stallman, founder of the Free Software Foundation and computer hacker, describes his hostility the to enclosure of the `tools of his trade', in this case, software, in the guise of `intellectual property.'\footnote{For Stallman, ``The term “intellectual property” carries a hidden assumption - that the way to think about all these disparate issues [copyright, patents, trademarks] is based on an analogy with physical objects, and our ideas of physical property.'' \url{http://www.gnu.org/philosophy/words-to-avoid.html}} He felt his was being forced to ``sign a promise you won't share [software] with anybody else.'' He continues\ldots

\begin{quote}``And to me that was essentially a promise to be a bad person, to betray or cut myself off from society, from the cooperating community. So, I just wasn't going to do that. I felt this is wrong. I am not going to live this way.''\footnote{\cite[9m55sec]{moore:2002rv}}
\end{quote}

\paragraph{}This led Stallman and others to develop and share free and open tools. This work began in the 1980s under the banner of The GNU Project\footnote{GNU is recursive acronym that stands for ``GNU's Not Unix'', it is, says Stallman ``a hack'' \cite[11m35sec]{moore:2002rv}} and led to the foundation of the Free Software Foundation (FSF).  Stallman's political and legal work is part of the foundational philosophy of the free/libre and, later, open source movements.

\paragraph{}The freedom Stallman and the FSF espouse in the Free Software Definition (FSD)\footnote{See \url{http://www.gnu.org/philosophy/free-sw.html}} is free-as-in-freedom, that is to say positive freedom: `free-as-in-speech' not `free-as-in-beer', i.e. \textit{gratis}, no charge. The four freedoms of the FSD are:

\begin{itemize}
\item Freedom 0 - The freedom to run the program, for any purpose
\item Freedom 1 - The freedom to study how the program works, and adapt it to your needs. Access to the source code is a precondition for this.
\item Freedom 2 - The freedom to redistribute copies so you can help your neighbour.
\item Freedom 3 - The freedom to improve the program, and release your improvements to the public, so that the whole community benefits. Access to the source code is a precondition for this.\footnote{See \url{http://www.gnu.org/philosophy/free-sw.html}}
\end{itemize}

\paragraph{}Stallman's reasoning behind this radical stance is evident in the title of his collected essays \textit{Free Software, Free Society}.\footnote{\cite{Stallman:2002xy}. See \cite{Williams:2007uq} for a critical engagement with FSF philosophy and \cite{Stallabrass:2002kx} for a review of Williams's book.} He askes ``what kind of rules make possible a good society that is good for the people in it?''\footnote{\cite[Ch. 8]{Williams:2007uq}} The FSD goes some way to answering that question. The FSD describes ``the freedoms that enable people to form a community.''\footnote{\cite[15m30sec]{moore:2002rv}} The definition seeks to protect an individuals' ``right to cooperate with other people and form a community.''\footnote{\textit{Ibid.,}17m32sec} Stallman's point is clear: ``if you don't have all these freedoms [in the FSD], you're being divided and dominated by somebody''\footnote{\textit{Ibid.,}15m30sec} Some hackers, such as Hill and Coleman, contend that the ``four freedoms'' of the FSD are ``based in and representative of an extreme form of anti-discrimination resistant to categorisation into the typical ``left, centre and right'' tripartite political schema.''\footnote{\cite{coleman:2004fo}} Yet free/libre is not completely resistant to categorisation. I contend that the `freedom' of free/libre is based on an appeal to rights, an exclamation of a right of self-determination and self-selection of a community and of property. Therefore, in many respects, the definition of the ``freedom'' ideal-type for free/libre is markedly liberal and individualistic in ontology and political economy.

\paragraph{}The FSF is, however, only one voice among the multitude of f/los hackers, artists and activists; the radicalism of the FSD is only one expression of hackerism. The free/libre philosophy in transmission, as one set of symbols among many in a networked society, can be found in different practical articulations. That is to say that by ``recalibrating the broad meaning of freedom outlined in the FSD to align with their own philosophies and politics\ldots groups perceive [f/los] as a model of openness and collaboration particularly well suited to meet their own goals.''\footnote{\textit{Ibid.}} 

\paragraph{}Here the second part of the f/los acronym emerges, \textit{open-source}.\footnote{See Appendix 3, The Open Source Definition and \url{http://opensource.org/}}Open-source can be read as simply a technology development methodology that co-opts the advantages of free-software development - principally freedoms 1 and 3, open access to the source-code -  while playing down its freedom principle. Open-source can also be described as a social movement. The `cause' of this social movement is the adoption of open-source 
practices; its strategy is the neutralisation of free-software, `recalibrating the broad meaning of freedom' as put forward by Stallman in the FSD, in order to become more business-friendly. There is a third reading of open-source which shows its political implications. Opening the source code requires an attitude of openness towards modification, duplication and redistribution. Open-source, therefore, becomes a metaphor for transparency, for accountability, for democracy; a point to which I will return. Openness necessitates and configures new social relationships and creates new political spaces.

\paragraph{}Building on the central motifs of the networked society, the information economy and contemporary organisational theory, Raymond muses on the phenomenology of open-source in a number of essays, collectively published as \textit{The Cathedral and the Bazaar}.\footnote{\cite{raymond:1999catb}} He notes a shift from hierarchical models of organisational, that he refers to as `the Cathedral', to a more ad-hoc horizontal organisation model that he calls `the Bazaar'. The Cathedral model symbolises the paradigm of industrial production, the competing industrial giants of the 20th century, and the Bazaar model symbolises the paradigm of open-source production in a networked society. There are other parallels that can be drawn from his observation. The centralised and hierarchical Cathedral model can be viewed as analogous to the waining structures of modern government that are being destabilised and deterritorialised by the rise of an open-source networked society paradigm, Raymonds' rhizomatic Bazaar model. 

\paragraph{}Openness conditions new social relationships and 
practices with new technologies in a new political space, the networked society of the Bazaar. For Anderson the dynamic co-development of human societies and technologies conditions the social order.\footnote{\cite{Anderson:1991ic}} The `nation' is an imagined community manifested in `states'; the emergence of this political category in the 19th century is closely related to the spread of technologies that increased inter-personal communication in vernacular languages. The global networks that have extended and accelerated communications in our contemporary world have facilitated the multiplication of imagined communities which now scale globally. The idea of `community' is frequently evoked in f/los discourse and most f/los activity takes place within a group, with disparate and dispersed members. Hacker communities are the archetype of a networked community. The role of `the community' in maintaining and distributing knowledge and in inculcating a good f/los civic values is central to the hacker ethos. 

\paragraph{}I contend that the networked communities and the political cultures they condition provide useful examples for positive changes to our social order and 
practices. Many different New Social Movements have co-opted the symbols and 
practices of free-software as well as deriving benefits from the free software produced by hackers. An example of f/los in transmission to social-political activism, which is frequently presented, is IndyMedia.org. This is `` a collective of independent media organisations and hundreds of journalists offering grassroots, non-corporate coverage. IndyMedia is a democratic media outlet for the creation of radical, accurate, and passionate tellings of truth.''\footnote{Quoted from IndyMedia.org homepage, accessed 24th April 2008, \url{http://www.indymedia.org/en/index.shtml}. As mentioned by \cite{Prug:2007fs}, \cite{Boomen:2005uq}, \cite{coleman:2004fo}} The openness of open-source fits with the radical transparency of this network of movements and the free/libre software produced by hackers sits easily with the broadly anti-capital stance of IndyMedia.

\paragraph{}There are other more mainstream examples of the implementation of open-source methods at the organisational level, the use of free/libre software tools at a functional level and the general adoption of the principles of collectivism, active and open participation, and consensus decision-making and sharing through the organisation. The emergence of `social source' may be defined narrowly as the ``marrying [of] open source software development with social service and social change applications''\footnote{\cite{rosenblatt:2005mn}}, or, more, broadly as the marrying of open-source principles and free/libre production and distribution with the express goal of social change. The `civic hacking' of the mySociety organisation embodies in every way the positive socio-political potential of f/los, building technological tools to open up governmental activity and reinvigorate democracy.\footnote{See mySociety.org, \textit{What's it all about then, eh? - mySociety Frequently Asked Questions.} \url{http://www.mysociety.org/faq}, accessed 24th April 2008.}

\paragraph{}So, in summary, I have sketched the transition from hierarchy to rhizome, from the monopolistic monolithic structures of the industrial age to the distributed, diverse peer-nodes of networked society. There are further questions to ask: of a future f/los society of individuals participating in communities, producing and socially sharing with each other? Rhizomatic democratic networks overcoming the tyranny of both structure and structurelessness? What of the tendency towards transnational cosmopolitan `governance without government'? Before returning to the challenges for the contemporary political order, I will discuss the hack of the dominant economic system presented by free/libre philosophy.