Is it plausible to think that a representative democracy can be made more participatory?

In attempting an answer of the question posed, some deconstruction and evaluation of the terms used will be required. What is democracy? What is characteristic of the representative mode? What opportunities exist for participation, and who are they available to? These aspects, and others which will no doubt arise, will need to be considered in order to provide a vivid response to the question. First, a short account of the concept democracy, its history and development over time, will be offered by way of an introduction.

The term democracy has no fixed definition. The word is used the world over to describe political systems which are so varied they have little in common. Deriving from the Ancient Greek, the word democracy describes one of the political systems of that period in history. Its two component words demos and kratos have dual definitions; demos meaning either the citizens or the �lower orders� of a state, kratos meaning power or rule. Arblaster comments that the �ambiguity� of meaning in each of the component words is �of permanent significance in grasping [democracy�s] meaning and its history� (Arblaster, 2002:14).

In Ancient Greece, the city-states with populations of 50,000 or more were the largest concentration of peoples. In the city of Athens the system of government was a participatory direct democracy. This entailed the direct involvement of citizens in the governance of their city-state. Here democracy meant rule by the citizens.  Although that it not to say that the democracy of the Ancient Greeks was by any means an agreed set of principles; witness the internal dissent of one citizen, Plato, in his dialogue The Republic. Athenian democracy did not extend to all sections of the city-state society. There were a great number whose participation was not sought: women, foreigners, and slaves were excluded; the opportunity of participation in governance was for free males of the city-states only. For Plato, equality of political opportunity for participation was most important in a democracy. (Lee, 1987:311-320).

Jump forward 2500 years from Ancient Greece to present day we find a large number of political systems described as democratic. The variety of democratic form shows that democracy has moved far from the model of the Greeks; it has become, as Arblaster observes, a critical contested concept (2002:7). A common theme among modern democracies is that of government deriving its authority and legitimacy to govern the demos from the demos themselves. The mode of democratic government usually found is representative. Here the demos are no longer participating directly in the action of government, rather representatives carry this out on their behalf.

The modern Western-style democracy is representative. Populations have long expanded past city-state administration and the direct participation of that model. In the representative system elections, and the continued certainty of them, select representatives of the demos. A look at the characteristics of democracy in the Western world and an account of the opportunities for participation is required before any assessment of whether or not representative democracy can be made more participatory is given.

As mentioned, the certainty of popular elections is key in a representative democracy. Representative systems across the world enable citizens within their boundaries to participate in the governance of their nations through the extension of the franchise to its members. This entitles citizens to select through ballot their representatives in government. Furthermore, citizens are enfranchised to become candidates for election. However, these rights are not without restriction; we have already seen how the �model� democratic system of ancient Athens did not invite the participation of certain members of its society. Participation of citizens has gradually increased over time, developing from the sole enfranchisement of adult propertied males of a certain age to the inclusion of all citizens, male and female, propertied and landless, who have reached a minimum age of qualification. Modern Western democratic systems are much nearer the ideal espoused by Plato, with all citizens enabled to participate equally.

A feature of the Athenian democratic system was the allocation of civic positions by random lot. Trial by jury in Ancient Greece could be better understood as trial by the demos. Juries in trials numbered six thousand; participation in legal judgment viewed as crucial to maintaining social cohesion. Citizens of city-states were compelled by law to perform their civic duty of sitting on judicial council and were selected for jury service by random lot. We still see an element of this allocation of civic duty in the judicial systems of western democracies; jury service can be requested of any citizen. Arblaster views this as the �sole vestige� of direct participation by citizens that survives in modern democracies (2002:20).

A further opportunity for participation in representative democratic systems is found in referenda. In a referendum, all citizens are balloted to answer a specific question. This mechanism differs from an election, where the question asked is  �Whom do you choose to govern you�, as referendums are usually employed when the decision to be made is issue specific and of radical lasting importance to citizens. Referenda are a powerful opportunity for participation in a representative system because whatever happens as a result of the vote, the responsibility for the decision made rests with the electorate.

Returning to the theme of elections and the representative democratic model. A worrying trend across the democratic systems governing most Western nations is that of declining participation in elections. What this trend signifies and how best to reverse the decline is the matter of current political debate. Whatever the causes of lower and lower voter turnout, the consequence for representative democracy is damaging. Low participation decreases the legitimacy of those elected. Consider the mathematics; if 60% of an electorate of 10m turnout to vote with the winning candidate receiving 40% of votes cast, in what sense is the candidate representative of the citizens as a whole when only 2.4m will have voted for them?

I will now review briefly what I see as the three opportunities for participation that exist in the Western representative democratic model. Firstly, there are popular elections with the franchise extended to all those of a qualifying age, regardless of sex, gender, age, race, religion, and socio-economic indices. Second, there is the collective decision-making through referenda, which enables issue-specific participation. The third opportunity for participation is random: the subpoena of citizens to pass judgement as a jury of peers.

Reviewing the formal options for participation in representative democratic systems appears to show that the opportunities for direct involvement of the electorate are limited. Popular elections come infrequently, with every 4 years being the norm in British representative democracy. The cynical amongst us may say that referenda are only employed when those persons elected representatives of the people are afraid of making a decision in line with the responsibilities delegated to them. Additionally, the possibility of participation in the judicial system as a randomly selected juror decreases as the population of nation-states grows. 

What competencies are required for participation in democratic systems? In each of the formal participation options just reviewed, citizens are asked to make choices. Decision-making requires intellectual reasoning skills on the part of the citizen. Additionally, actions have consequences. Deliberation about possible consequences requires acumen. Historically, this argument on competency has been used to discourage participation. The demos have been viewed as being without the capacity to make informed rational judgements. This was the basis for Plato to support �philosopher kings� as rulers. If there exists a technocratic knowledge or experience barrier to participation in democratic systems the solution should be enable citizens through education and the dismantling of obfuscating methods. Educated, literate and informed citizens participating in a representative democracy would be able to judge for themselves the priorities and direction of government.

What informal opportunities for participation exist within a representative democracy? Are there forums � another word with its etymological roots in Ancient Greece � for widespread participation in the democratic process? Rao, writing on The Changing Context of Representation, highlights a number of mechanisms that are deployed within democratic systems to give voice to those represented, including �opinion surveys, referenda, citizens� ballots, consensus conferencing, deliberative opinion polls, citizens� panels and citizens� juries� (2000:3). I will briefly consider how some of these mechanisms precipitate participation from the sovereign body of the people and whether or not they could be used to make representative democracy more participatory.

Opinion surveys are a mechanism for soliciting the view of citizens on a given subject. Surveys enable the participation of citizens insofar as they create the opportunity for the views on any given subject to be captured and reacted to by elected representatives. However it is not plausible to think that because a voice has been heard it will be acted upon. Referenda and citizens� ballots could be construed as being employed only when elected representatives are not prepared to make a decision by the authority vested in them through popular elections. Representatives seek to make use of this consultative device when, as previously observed, the resultant action is of monumental consequence to those governed. In this example, it is clear that the participation being sought is binding, that is to say the outcome is directly affects the participants. However, this absolutist, majority rule scenario is not, as previously highlighted, without its mathematical drawbacks. Finally, citizens� juries, convened by random lot, are a plausible way to strengthen the �sole vestige� of participation in a representative democracy. 

Returning to the question posed, �Is it plausible to think that a representative democracy can be made more participatory?� I believe that the answer lies in achieving a balance of both representation and participation. The representative model needs to be strengthened by increasing the opportunities for participation. The representative model can precipitate citizen participation by providing the opportunities for it and strengthen participation by acting upon the �will of the people�.

There are practical problems to be overcome due to the growth of populations. There are few population groups small enough where the people as a whole can be convened to deliberate and vote: the last traces of Athenian democracy are visible in the Cantons of Switzerland. I believe that mass communications technology will enable a higher level of participation in representative democracies. The use of the internet as a medium for increasing participation has yet to be fully embraced. Virtual town square meetings, collaborative policy formulation, opinion polling and instant referenda are a reality made possible with internet based tools and protocols. These methods of making the representative mode participatory should be fully explored.

The UK Parliament has begun to wake-up to e-democracy and is beginning to use internet tools both internally, as part of the process of government, and externally, openly precipitating participation from the electorate. �TellParliament.net� is the internet based public consultation forum run by Parliamentary Select Committees. Recent open participation forums have considered Hate Crime, Constitutional Reform, the Modernisation of Parliament and Human Reproductive Technologies. Internet tools have also been the subjects of debate within Westminster Hall as Parliament seeks ways to scrutinise Bills before they are drafted.

Web logs or blogs are an excellent example of e-participatory democracy in action. Blogs differ from a website in that they feature an online journal and invite comment. Politicians in the UK have begun to take-up this new web software, seeing the technology as a way to engage with the electorate at a time convenient to them. A recent Hansard Society publication, Political Blogs � Craze or Convention, concluded that blogging was a �new media resource that can increase the transparency and accessibility of parliamentarians and their work� (2004:24). 

Representative democracy in the Western world has developed over the last hundred years to include the participation of more people through the extension of electoral franchise. But a system whereby the sovereign body of �the people� answer the question �whom do you choose to govern you� once every four or five years is not designed to require any greater participation from the citizens. Reforms leading to a government of the people, for the people, by people will be required in order to ensure that participation results in action, otherwise increased involvement will lead to dissatisfaction with the political process.

Is there a problem with representation? Can another person truly represent the interests of another? Representation is not reflection. With such diversity in society, do representatives need to reflect the demographic (ethnic, socioeconomic, gender &c) in order to represent interests? A discussion on the suitability of representation as a means of government is out of the scope of this essay, but the need for government to be relevant to and reflective of societies interests is essential. Participation ensures that government is relevant and reflective of society as it gives every citizen a stake in the process. In conclusion, I believe that it is plausible to think that the representative democratic model of government can and should be made more participatory.
Bibliography

Arblaster, A., Democracy Philadelphia, PA: Open University Press, 2002

Dahl, R. A Preface to Democratic Theory, London, The University of Chicago Press, 1956

Goodwin, B., Using Political Ideas, Chichester, John Wiley & Sons, 1992

Heywood, A., Political Theory, Basingstoke, Palgrave Macmillan, 2004

Lee, D., trans Plato The Republic, London, Penguin, 1987

Rao, N., editor Representation and Community in Western Democracies, Basingstoke, Macmillan, 200

Warburton, N. Philosophy, Routledge, London, 1999


Political Blogs � Craze or Convention? Hansard Society, 2004, London. http://www.hansardsociety.org.uk/programmes/e-democracy/blogging_report
Downloaded November 2004.
