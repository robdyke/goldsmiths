\documentclass[12pt,a4paper,titlepage]{article}
\usepackage{geometry}
\usepackage{helvet}
\usepackage{harvard}
\usepackage{setspace}
\usepackage{url}
\renewcommand{\familydefault}{\sfdefault}
\geometry{a4paper}
\title{To what extent do social, p2p or participatory financial models democratise capital?}
\author{Rob Dyke}
\date{\today}

\begin{document}
\harvardparenthesis{none}
\citationstyle{dcu}
\bibliographystyle{agsm}
\maketitle
\tableofcontents
\newpage
\begin{quote}
``What matters in the world is money, machines and people, in that order. Our political task is to reverse the order.'' Keith Hart\footnote{\cite[p. 6]{HartK2005thmd}}
\end{quote}
\doublespacing
\paragraph{}The long established savings and loan institutions are criticised at every point, from the financial models that they operate down to their de-personalised, off-shore call-centres. My argument in this paper is that banking services are anti-social, hierarchical and intensely private. These traditional banking services are, I contend, inherently undemocratic, even anti-democratic at the extremes. However, contemporary challenges to these established models are emerging, presenting the possibility of a more sociable, democratic alternative.

\paragraph{}I have spilt my paper into three sections. To commence I will sketch an outline anthropology of money. By using this perspective on the political economy of money I hope to separate money and markets from capital and capitalism. The purpose of this exercise is to demonstrate that there is nothing `wrong' with money per se, rather that it is the institutions and mechanisms that manipulate money that have distorted its purpose, de-coupling money from the social circuit it was once attached to.

\paragraph{}What do I mean by social, peer-to-peer and participatory models? How do these approaches differ from the dominant mode of providing financial services? To answer these questions, in the second section  I will draw on case studies of Zopa, Kiva and Open Capital to demonstrate these newly emergent business models, comparing and contrasting these models with their traditional correlate.

\paragraph{}My first case is Zopa, a peer-to-peer virtualised financial marketplace. My second example is Kiva, a peer-to-peer international microcredit for development facilitator. My third case study will look at Open Capital, a business model for participatory community infrastructure projects on a significant scale. My research for these case studies is based on interviews with key individuals in these organisations, in the case of Zopa and Open Capital. For all three case studies materials have been sourced from the public web presences of the organisations as well as from secondary sources.

\paragraph{}I will contrast the relationships configured by these new models with  those of traditional banks. In so doing I will expose and criticise the crucial purpose of banking - that is, to create money from debt, the making of money from money. I will examine the possibilities for and consequences of the (re)socialisation of these transactions. As a contrast to private banking models I will explore traditional social lending models; co-operatives, mutual socities, building socities. These fading models, some long since capitalised and now floated on stock-markets, will also be contrasted to contemporary peer-to-peer models.

\paragraph{}For each case I have considered the following three aspects: sociability, scalability and sustainability. In terms of sociability, I am interested in how `social' these models are and to what extent they are democratic in terms of participation, opportunity, access and governance. As to the scalability of these models, I have considered the size and duration of loans offered and the reach of services provided. With regard to the sustainability of these models I have been examining the long-term profitability of these models, where appropriate, and have assessed the ability of these models to withstand fluctuations in the global financial network of which they are an inseparable part.

\paragraph{}Drawing on my case studies, in the third section I will consider to what extent capital can be democratic. As will become clear in my examination, by democracy I am thinking in terms of decision-making and governance on the one hand and in terms of access to capital and financial services on the other.

\newpage
\section{Money: Global, Social and Virtual}
\singlespacing
\begin{quote}
``Most of the confusion existing in monetary theory [is] due to the separation of politics and economics, this outstanding characteristic of market society''\footnote{\cite[p. 195]{Polanyi:1957mi}}
\end{quote}
\doublespacing
\paragraph{}Money, according to anthropologist Keith Hart, is global, social and virtual.\footnote{\cite[Weblog entry, 9th October 2007]{HartKblog2007-2}} His ethnography of money describes it as a `memory bank', storing the value of transactions that humans wish to calculate. Conceptually, money is necessarily global. Specific currencies are local to a given nation-state, such as the Yen in Japan or the South African Rand, or local to a federation of states, such as the Euro in Europe and the Dollar in the United States of America. Money, however, is global insofar as it is a medium of exchange between localities. Currency exchange, speculation and trading, while certainly transactions in themselves, are not conducted with the same motives and outcomes as the purchasing of goods and services, where money is the medium of exchange, not the commodity exchanged. That money can take a material presence in the form of cash merely disguises the fact that money doesn't exist, only currencies exist physically. This may seem like an obvious distinction but it is a necessary one to make in order to de-couple money, a \textit{medium} of exchange, from capital and capitalism, a \textit{system} of exchange.

\paragraph{}Money is a social construction. The classical liberal economic narrative has it that it was brought in to being in order to resolve a specific need, the need to exchange commodities between producers and users indirectly. Money displaces the barter mode of the traditional marketplace by enabling transactions between multiple producers and consumers of different goods and services to take place over greater distance and over extended periods of time than would be possible by direct exchange. The classic economics texts of liberal John Locke and communist Karl Marx both contain this analysis of the social utility of money.

\paragraph{}To say that money is virtual in our contemporary society would be no exaggeration; there is insufficient currency in circulation to back the deposits in banks in the event of a run on a single bank, let alone the whole of the global banking system.\footnote{See report from the Bank of England's Monetary and Financial Statistics Division \& Monetary Assessment and Strategy Division on narrow money data.} Increasingly the cash currency used for daily transactions is being virtualised with smart-card technology.\footnote{RFID-enabled smart cards such as the Transport for London (TfL) Oyster system are facilitating the development of a cashless society. The partnership between Barclays Bank and TfL to converge micro-payment facilities, credit cards and Oyster information on the same physical piece of plastic demonstrates, perhaps, the direction in which virtual financial services are heading.} To say that money has always been virtual is to recognise the purpose of money in a global society. Money is virtual, according to Keith Hart, because it is a \textit{symbol}, the medium rather than the \textit{object} of exchange. That the accumulation of money, that is to say the creating and increasing of profit for wealth, has come to dominate exchange in a marketplace is due to the pervasive nature of capitalism. There is a second sense in which money is virtual: Karl Polanyi demonstrates that money is a `fictitious commodity' along with labour and land.\footnote{\cite[Ch. 6]{Polanyi:1957mi}} Each of these fictitious commodities has a fictitious price attached to it. For money this is interest; with labour, wages; for land, rent.

\paragraph{}Technology has increased the scale and scope of virtual money. Information processing and communications technologies have allowed money to travel further and faster through ever more complex circuits and abstractions of exchange. As technology becomes increasing integrated into the public and private lives of individuals and the operations of companies it contributes to an increased awareness of the variety of goods and services offered in a global marketplace, enabling both the instantaneous comparison and acquisition of goods or services. Furthermore, communications technologies mean that there is no longer a need for physical co-presence in order to access these goods or services. This last point is crucial to my argument that the re-socialisation and democratisation of money and financial institutions is possible, a point to which I will return.

\paragraph{}Stocks of money, that is to say capital, are considered a raw material for the capitalist market system as much as minerals mined from the land and the productive capacity of labour. Money is made from money in a capitalist system in two distinct ways. The first is the most obvious means and is intrinsic to capitalist market economies. Capital is consumed as a raw material to make goods or to create services which can be sold for a profit. Profit replaces the raw capital and adds to its stock.

\paragraph{}In all this talk of virtual money, goods and services and raw capital I've masked the social world, the people and their activities that these transactions describe: people, businesses, communities, institutions making decisions based on their aspirations, expectations, calculations. Paying for education, starting a business and building infrastructure often involve significant amounts of money. Borrowing money is one way of financing these transactions.

\paragraph{}The second method of making money with money is the deceptively simple process described by the economist J. K. Galbraith. It is to this institutional practice of the creation of wealth embodied in the banking industry, sanctioned and guarenteed by the coercive force of the state, that I now turn.

\newpage
\section{Institutions}
\singlespacing
\begin{quote}
`The process by which banks create money is so simple that the mind is instantly repelled. Where something so important is involved, a deeper mystery seems only decent.' John Kenneth Galbraith\footnote{\cite[p. 29]{Galbraith:1975kx}}
\end{quote}
\medskip
\begin{quote}
Les billets de banque sont prot\'{e}g\'{e}s par le droit p\'{e}nal.\footnote{``Banknotes are protected by the penal code.'' Printed on every Swiss banknote. Translation thanks to Elizabeth Trott.}
\end{quote}

\doublespacing
\paragraph{}Contemporary business models of the banking industry are criticised for a lack of transparency, impersonal relationships, heavy systematisation and structural exclusion from services. These criticisms of banking institutions and debt money are magnified at the international level. Against these institutions new models arise: is it possible that we are witnessing a re-socialisation of banking along the lines of a post-modern Baily Savings and Loan\footnote{The Baily Savings and Loan is the social lending institution in \textit{It's a Wonderful Life} (\cite{capra:1947wl})} or a new modality of capital?

\paragraph{}Banking institutions have created the vast majority of money in our societies from simple promises to pay them money. These promises, which are enforced by the coercive force of the state, are interest bearing. As Karl Polanyi describes, ``interest is the price for the use of money and forms the income of those who are in the position to provide it.''\footnote{\cite[p. 69]{Polanyi:1957mi}} The debt or credit, i.e. the capital sum, is increased by the levying of interest for the providing of the service of safe-keeping of deposits or for the facility of credit. The former promise bears less interest than the latter, reflecting, according to the rationale of the banking industry, the risk involved in extending that financial facility. This understanding of money is explored in the activist motion picture \textit{Money as Debt}.\footnote{\cite{grignon:2006}} However you don't need to be a money-crank to accept the most basic argument presented in the film: banks create money through debt. The net difference between those two rates of interest (that paid on deposits and charged on credit facilities) when understood as the money that it represents, rather than interest percentages is the profit made by the bank.

\paragraph{}As I have outlined, profit for institutional lenders is made by charging interest on credit services minus the interest paid to depositors. Yet, this is not the whole story; the business models of traditional banks are dependent on bad debt and the upselling of additional financial products to customers, such as payment protection insurance. Risk actuaries employed by the banks factor the profit that can be made from taking on sub-prime debt - debt with a greater probability of default than repayment. With promises to pay money ``protected by the penal code'' the true nature of interest-bearing business models is exposed: this is not a service but state-protected avarice.

\paragraph{}Banks are hierarchical in so far as they maintain a vertical structure in terms of tiered service provision for their customers (`basic', `silver', `gold', `platinum' services levels for example\footnote{The choice of precious minerals for the names of tiers of service is an insight to the history of banks as simply safe places to store these valuable commodities}) and in terms of competition against each other. Furthermore, banks operate on the basis of anonymity and the reductive categorisation of their customers, identifying depositors and creditors alike as impersonal account numbers and sort codes in vast computerised databases. The relationship between a customer and his or her bank is governed by a privately factored credit score, distributed through agencies such as Experian and Equifax. It could be argued that these ratings agencies hold more power over an individual than the institutions that use them! A poor credit score translates into a higher rate of interest for credit, if credit is offered at all; indeed, a poor credit score is most likely to result in the exclusion of an individual from banking services, even from the most basic of bank accounts. There are no personal, that is to say \textit{social}, relationships between the individual and his or her banking institution.

\paragraph{}A few further words on credit scoring are necessary before continuing. Credit ratings agencies operate on a number of different levels - from the rating of individuals, in the case of Experian and Equifax, and the credit worthiness of companies provided by international agencies such as Standard and Poor, to the credit scoring of entire nation-states in the case of the International Monetary Fund and the World Bank. Credit agencies gather information on individuals, companies and nation-states from a variety of sources. Whenever credit is requested a search is made with these agencies for data held on the applicant. This data probably includes information on prior credit requests, the status of existing or historical credit provisions - whether discharged or defaulted - and other indicators on the likelihood of repayment of any credit extended to the applicant. I say \textit{probably} because the topic of this data is not public: credit scoring depends on a shroud of secrecy.\footnote{For an insightful vignette on the affect on the individual of credit scoring see ``Speculating on Student Debt'', The \cite{committee:2007pb}} The opacity of credit scoring agencies and the private nature of the emphasis each individual bank places on a particular aspect of this data contribute to the charge of a lack of transparency in the banking industry. The `worth' of an individual for credit, when expressed on a scale, reduces an individual human life to a metric. A question then for the emerging financial models I will examine is this: is there more or less use of these mechanisms by social lending business models? How is the `worth' of an individual calculated by these models?

\paragraph{}Credit institutions have not always been anti-social, hierarchical, intensely private and motivated by profit. The Savings and Loan institutions in America and the Co-Operatives Mutual, Friendly and Building societies of the UK are examples traditional sociable credit services. These models are in decline, some extinct, unable to withstand the tide of global capitalism. However the themes of mutual aid, community and philanthropy central to these models are not lost, they are to be found in transmission, in new credit models, the social, peer-to-peer and participatory models that I will go on to examine.

\paragraph{}Against the hierarchical and secret nature of mainstream banking models there emerge new modes and practices emphasising horizontal and open structures. These new business models have three significant characteristics of difference from their mainstream counterparts: sociability, peer-to-peer and participation. As will become clear from the case studies I will present, each of these characteristics is understood in different ways in each of the business models, yet it is possible to tease out the aspects they have in common. These peer-to-peer, social and participatory financial models reflect autonomy, individualism and choice and the importance of post-material and post-modern values as a determinant on the economic activity of individuals and groups. Sociability is understood in terms of `community': an ambiguous concept at the best of times, which here relates to the social interactions of individuals and groups around financial services and engaging in economic activity. `Community' contrasts with the anonymous and alienated relationships embodied in mainstream banking. The term peer-to-peer (p2p) has its origins in the computer networks and the open source movement and describes rhizomatic communication between peers or points on the internet. In terms of these newly emerging models, what is described is a direct, `horizontal', person-to-person transaction. In traditional banks decisions are made in a closed loop in which the debtor or creditor has little to no involvement. In contrast to this privacy, these new models emphasis the participation of people, as customers, in the decision-making process. 

\newpage
\subsection{Case Study: Zopa}
\singlespacing
\begin{quote}
``Lenders get great returns and borrowers get low-cost loans. And with no bank in the middle, everyone get better rates and money becomes human again.''\footnote{Zopa website}
\end{quote}
\doublespacing

\paragraph{}Use of the internet has been the deciding factor in the success of challenges to traditional business models in a number of sectors. Internet-based businesses are lean, a low-cost competition to established businesses due to the near-total automation of business processing and because they conduct their business without an costly physical co-presence. Internet start-ups aim to capture a share of an already active and profitable market. With the market for unsecured personal loans in the UK worth around 20bn GBP and the profits in banking significant, the financial services sector is often challenged by new businesses offering an internet-based alternative. Zopa is one such challenger. Zopa is a virtualised savings and loan marketplace, facilitating peer-to-peer transactions. It differs from the high-street banking sector - it refuses the label of bank! - as Zopa holds no deposits for its customers and it does not loan directly from its own (ficitious) capital base to would be debtors.

\paragraph{}Zopa provides two peer-2-peer lending marketplaces. The first is a managed lending service where Zopa does all the work and the lender simply collects on the returns.\footnote{See \url{http://uk.zopa.com/ZopaWeb/public/about-zopa/how-it-works.html}. Accessed Apr 13th 2008} The second marketplace is much more `light-touch' in terms of Zopa's involvement. In this marketplace much greater control over rates and terms is devolved to the borrower and lenders to agree between themselves.\footnote{Zopa listings:  \url{http://uk.zopa.com/ZopaWeb/Listings/}} The Zopa marketplace(s) connects people with money (fictitious capital) to people who would borrow it, for a price (interest). Is this person-to-person transaction really making money human again? This model of a money market is certainly virtual; to what extent is this money market social? How democratic is this model? I met with Giles Andrews, Zopa's Chief Financial Officer, to discover more about the Zopa model.

\paragraph{}Zopa understands sociability in three key ways. Firstly Zopa emphasises that their business is about people helping people. While this assistance is within an interest-bearing framework there is an element of philanthropic truth in this argument; replacing the depersonalised anti-social system of banking available on the high-street, Zopa empowers real people to make individual lending and borrowing decisions. Secondly Zopa provide a 'safe harbour' for leaders and borrowers to conduct their business. The marketplace manages the risk involved in money-lending through the credit-scoring of applicants and by enforcing debts. Furthermore the Zopa marketplace is a secure electronic system regulated by the financial watchdog of the country of operation, increasing the security and safety of person-to-person lending. 

\paragraph{}The third aspect of sociability for Zopa is community and trust. The business model connects multiple lender's with multiple borrowers with each individual borrower borrowing from a number of lenders, thus reducing the risk of loss to a lenders capital.This configures inter-personal relationships between borrowers and lenders, building a community around Zopa. These (pre-qualified) inter-personal relationships are strengthened by trust; Zopa emphasises this element of its business. The default rate for Zopa facilitated loans is significantly lower than the industry average.\footnote{ZOPA runs a low level of bad debt at just 0.05\%. against an industry average of 0/5\%. Sources: \cite{Hawkes:2007} and \cite{Cattermole:2004}} Repayment is encouraged as borrowers feel a personal connection to the lender(s), that is simply absent in the comparable relationship between a borrower and a high-street bank.\footnote{Notes from interview with Giles Andrews and quantative analysis in \cite{hulme:2006sl}}

\paragraph{}The use of credit scoring is not diminished in this model as Andrews explained. Zopa make significant use of credit-scoring and `hand-check' all borrower applications and there is no greater likelihood of acceptance from Zopa than any other institution. In fact, Zopa are quite risk averse in this respect because its person-to-person business model and the very competence of Zopa as risk assessors would be critically damaged in the event of widespread default by borrowers. Zopa are, however, much more social than high-street banks because they do not depend on bad-debt to increase the profitability of their business.\footnote{See 'Zopa - How we make our money' \newline\url{http://uk.zopa.com/zopaweb/public/about-zopa/how-does-zopa-make-money.html}} For Zopa, however, the credit worthiness of an individual is not the sole factor on which a decision to loan money is made, because the lending decisions, the risk/reward speculation, is made by the lenders themselves. While a credit score does influence whether Zopa will accept a borrower into the marketplace, the credit-score is only a single part of a more complex decision-making process with lender(s) and the borrower engaging in dialogue over a transaction.

\paragraph{}This radical transparency, in contrast with the `computer says no' decision model of high-street lenders, does not end here. The crucial part of the sociability of the Zopa business model is precisely the inter-personal relationships that govern the loan decision. While a high-street bank bases its decisions on inpersonal metrics, Zopa lenders in the marketplace base their decisions on any number of other factors\ldots \textit{perhaps I will not loan you the money for a season ticket for your team because I support another team\ldots perhaps I will loan you money for a further education course because you intend to study a similar course or at the same institution that I did\ldots I'll loan you money for a new car but not not for a motorcycle as I think they are dangerous\ldots} In this sense Zopa is not masking the social world; the purpose of the loan and the readiness of the borrower to talk about themselves and the instrumentality the money will give them makes `money human again'\ldots if only partially.

\paragraph{}How scalable is the Zopa model? Andrews explained to me that Zopa's target market is that of the unsecured personal loan, which is short in duration and low in value. That said, there is nothing to prevent Zopa, in terms of regulation, from scaling up their business and facilitating larger transactions; the provision of mortgages, for example. Two factors for a p2p lending business to consider in moving into this market segment would be the appetite of small lenders to tie-up their money for a significant period of time on the one hand, and the number of lenders and the amount of capital that would need to be pooled to provide a mortgage on the other. In terms of global scaling, due to different regulatory regimes for the different local financial markets of the world Zopa money is not able to be global. Although Zopa operates in a number of different markets; directly in the US and UK and with partnerships and franchises in other countries of Europe and in Asia, these marketplaces are not linked and it would be illegal to do so. In practice then Zopa facilitates local transactions in so far as a lender from the UK is not permitted to loan to a borrower in another country. Zopa's scalability is then limited to the what new business it can capture within a specific country's market and what it can poach from other lenders.

\paragraph{}How sustainable is Zopa? If Zopa continues to grow its business at present rates it will gain a critical mass and become a serious challenge to the long-term viability of traditional high-street lenders. In the event of a widespread economic recession the challenge to Zopa would be different to that of traditional lenders. Zopa may even outperform others in the debt industry in the event of a liquidity crisis in the wider capital markets - remember Zopa \textit{facilitates} p2p transactions rather than extending credit directly. The challenge to Zopa's business would be to continue to make the money needed for its business to operate through its 0.5\% service charge: if people do not borrow, lend or can not afford to repay loans the Zopa model would foreclose on itself.

\subsection{Case Study:  Kiva}
\singlespacing
\begin{quote}
``You lend to a specific entrepreneur in the developing world - empowering them to lift themselves out of poverty.''\footnote{Kiva website}
\end{quote}
\doublespacing

\paragraph{}The Weberian `spirit of capitalism', that is to say that the accumulation of wealth as an end in itself which is demonstrative of virtue, commitment, skill and stability on the part of the individual, now takes an electronically facilitated, and decidedly post-modern, global turn. For Weber, with wealth came responsibility to others: your wealth is God given, now do Gods work with it. The Kiva project enables people to short-circuit international development finance. Kiva has created an online marketplace for lenders and borrowers to come together, much like Zopa. Kiva facilitates person-to-person loans; the `disposable' income of a (relatively) wealthy westerner loaned to an entrepreneur in the developing world. Kiva however is a radical departure from the trading of capital for interest; Kiva lenders are not loaning for profit, rather they are engaging in direct and global philanthropy. By choosing to loan on Kiva, someone is "sponsoring a business" to help the world's working poor towards economic independence. 

\paragraph{}The Kiva marketplace lists the profiles and business plans of groups and individuals in the developing world in search of capital. Business plans are produced and reviewed by local activists and existing micro-credit organisations - Kiva.org calls them `field partners'\footnote{See Kiva.org \url{http://www.kiva.org/about/risk__partnerRole}} - who are the critical on the ground link to entrepreneurs. These field partners work in support of the new or growing business, report back to investors and manage repayments.

\paragraph{}This model significantly challenges the dominant development models of large-scale interventions, the legacy of which, Highly Indebted Poor Countries, should be a stain on the conscience of the developed world. Kiva is not, however, charity - these are loans and not gifts. The capital that changes hands is a strategic investment in entrepreneurial activity which the investor expects to be returned. As with Zopa, some sort of risk assessment is carried out, although not of the same manner of credit checking in the developed world. And, as Kiva's website disclaims ``Lending to the working poor through Kiva involves risk of principal loss. Kiva does not guarantee repayment nor do we offer a financial return on your loan.'' That said, on \$4,328,735 of loans with completed loan terms, the default rate is just 0.1\%\footnote{Source: \url{http://www.kiva.org/about/risk/overview}}

\paragraph{}Inspired by the work of Muhammad Yunus\footnote{Founder of Grameen Bank microcredit institution in India who promotes access to credit as fundamental human right. \url{http://www.grameen-info.org/}} and Amartya Sen,\footnote{\cite{sen:1999df}} Kiva has grown from an activist project to one of the world's largest microfinance peer-to-peer marketplaces, connecting entrepreneurs in the developing world to millions of dollars in loans from tens of thousands of lenders around the world. Kiva.org is helping the poor get access to credit. The relatively small-change from wealthy west is pooled and directly invested as micro-finance in entrepreneurial activity in the developing world. No banks, governments, agencies, nepotism nor corruption between lender and borrower. The amounts of money are small yet the positive effects on the lives of the world's poor are great.

\paragraph{}How scalable is the Kiva model? Globality and the inter-connectedness of our social existance are central to Kiva model. Kiva money is global in a way that Zopa is not - Kiva loans do not require `protection by the penal code' in the way that mainstream debt does so Kiva money need know no borders. Yet, however great the significant change in the circumstances of borrower, Kiva money is \textit{micro}-finance and does not yet scale to major community infrastructure projects.

\subsubsection{On Philanthropy\ldots}
\paragraph{}Why are these peer-finance models successful? What is the attraction? These emerging models are, perhaps, representative of a `New Social Movement' in economics, made possible by the wealth of the post-material and post-industrial developed `west'.\footnote{For a recent review of New Social Movements see \cite{Porta:2006kx}.} With individual material needs to a greater or lesser degree satisfied, concerns beyond the material can develop; social and communitarian concerns. Zopa and Kiva have built on this growing holisticism, establishing participatory projects with philanthropic motivations. The mainstream corporate banking models I've discussed practice a `strategic' philanthropy, expecting a return for their investment, whereas organisations like Kiva facilitate `altruistic' philanthropy, characterised by the direction of funds, with Zopa sitting somewhere between these two poles.\footnote{For an extended discussion of Strategic and Altruistic Philanthropy as it relates to social lending see \cite{hulme:2006sl}.}


\subsection{Case Study: Open Capital}
\singlespacing
\begin{quote}
``The incorporation of Co-operative Principles gives us a Co-operative Corporate Partnership: this entity is capable of underpinning a truly Co-operative Society.''\footnote{\cite[p. 2]{cook:2004cp}}
\end{quote}
\doublespacing

\paragraph{}Chris Cook's model of `Open Capital' demonstrates that participatory economic models can scale beyond that of Kiva to capital projects of a significant size. His models, based on using the legal status of Limited Liability Partnership (LLP), represent a radical departure from other corporate or charitable forms. Open Capital is contrasted to the closed, often contradictory and antagonistic mode of capital in other incorporated bodies. The Limited Liability Partnership legal form allows for new modes of economic activity. In the Open Capital mode different economic actors become members of a legally defined grouping in order to pool their resources, meaning that members of an LLP are `on the same side' – there is no `profit' or `loss' between the Members of a partnership, merely the creation, exchange and accumulation of economic value - in whatever form, be that resources 'to the value of' money or indeed money, that members choose.

\paragraph{}Open Capital attempts to solve a ``paradox of the modern world: that humans have never been more inter-dependent in our needs, or more individualist in our outlook.''\footnote{\cite[p. 6]{cook:2004cp}} The two innovative concepts that characterise the Open Capital model are the alignment of interests of partners with other stakeholders and the use of proportional shares to a) represent value and b) constitute a form of capital, `Temporary Equity' for use between partners. Chris Cook believes that the phenomenon described as `Temporary Equity', which has the ability to raise capital without borrowing at interest, confers the emerging `Open Capital' LLP model with a `property' which may render it superior to others.

\begin{quote}
A transaction entered into in 2002 by the Hilton Hotel Group serves as an example of how this `Temporary Equity' may operate in practice. Hilton Hotel Group sold a portfolio of 10 hotels for some 350m GBP to an LLP in which Hilton (the Occupier) hold 40\% and the balance of 60\% is owned by another LLP linking the three Investor Members. For 27 years, investors receive 28.8\% of the gross revenues from these hotels plus a further 33m GBP p.a.,  all subject to a floor of 317.5m GBP p.a. or 5\%. Note firstly that there is no Debt and no Interest in this structure and the risks and rewards are shared as in all true partnerships. The outcome is therefore to create `Temporary Equity' with a 27 year term. \textit{Most crucially, it is in the interests of both Occupiers and Investors to co-operate in order to create the maximum flow of revenue over the period of the agreement.}\footnote{\cite[p. 10]{cook:2004cp}. My emphasis.}
\end{quote}

\paragraph{}This Labour government's flagship Public Private Partnerships (PPP) and Private Finance Initiatives (PFIs) have been criticised for enriching private interests at the expense of the tax-payer. These PPPs are based on `partnerships' that are in reality `joint ventures' in business transactions in which at least one of the partners holds a 'for-profit', i.e. capital gain, motive. It is difficult for this author to see how the interests of government spending on civic services can be reconciled with the logic of private capital. As Ruth Sunderland comments, the ``bailout of Metronet, which maintained nine tube lines, exposes the utter fiction that Public Private Partnerships and PFI projects transfer risk and debt \textit{away} from the public sector.''\footnote{\cite{Sunderland:2008gn}}

\paragraph{}In contrast to PPPs and PFIs, the Open Capital / Limited Liability Partnership model offers a possitive possibility for public capital projects of the scale of Metronet. This posibility is presented in the transcendence of both profit and loss by the positive step of the redefinition of these concepts. In a departure from the ``competitive economy based upon shareholder value and unsustainable growth results from a transfer of risks outwards, and the transfer of reward inwards''\footnote{\cite[p. 17]{cook:2004cp}}, Limited Liabilty Partnerships neutralise the conflict created by the centrifugal/centripedal strategy. In place of `profit' and `loss' are `risk' and `reward', as defined by all participants in the partnership to result in a mutually satisfactory exchange of value, rather than the simple exchange of capital. This has dramatic consequences for public spending. Firstly, money would have no “cost” when issued. Second, the `public' does not need to borrow to invest. Thirdly, a ``National Equity'' formed from the value of the assets can be valued, as well as a National Debt.

\paragraph{}In terms of socialbility, Open Capital configures co-operative relationships around the economic activity. Indeed, the social purpose of the economic activity is central to the LLP model. As a LLP organisation can be created without the involvement of solicitors or accountants it is certainly accessible to many different economic actors and different purposes. Chris Cooks gives a number of examples of sociable LLPs, from community land partnerships to the London 2012 Olympics, even private housing projects, on the Open Capital website.\footnote{\url{http://www.opencapital.net/co-ownership.htm}}

\newpage
\section{To what extent can capital be democratised?}

\paragraph{}I am considering two aspects of democracy when evaluating the democratising potential of these peer-2-peer and social models. I am thinking of decision-making, in terms of transparency, governance and accountability, on the one hand and in terms of access to capital and credit services on the other.

\paragraph{}Thinking in terms of control of decisions, Zopa is certainly democratizing capital. In terms of governance and accountability, Zopa's radical transparency in all aspects of its business model and its openness in its use of online public forums are a marked departure from the private models of mainstream services. In the Social Futures Observatory\footnote{\cite{hulme:2006sl}} survey of Zopa clients, 81\% of lenders felt that Zopa offered them ‘significant’ control, whereas only 4\% felt that mainstream financial services offered ‘significant’ control. Similarly, whilst 70\% of borrowers felt that Zopa offered ‘significant’ control, only 1\% thought that mainstream financial services offered ‘significant’ control. Zopa's has a thriving community who participate in the online forums, offering peer-support to new-commers and seasoned members. Key members of the user community are regularly invited to meet with the Zopa team in person and have a voice in the development of the business. However the Zopa model does not `open up' credit to a greater number of people, rather, it is an innovative \textit{alternative} to the mainstream models. 

\paragraph{}Kiva, in contrast, is radically opening up access to capital. Micro-finance projects target the problem of access to credit and Kiva, through its use of the internet, is a very effective agency in co-ordinating this altruistic philanthropy. 

\paragraph{}The most promising hope for democratising capital is found in Open Capital and the co-operative society that this flexible and scalable model presents us with a glimpse of. Open Capital LLP enterprises can operate with charitable, voluntary, social and commercial aims and objectives. With Open Capital modes of economic activity, the social world is rescued from being subsumed by the economic world because relationships between the owners and the users of capital, however so defined, are no longer conditioned by interest and competition, but by value and co-operation. 

\paragraph{}I began this paper with this challenge from Keith Hart\ldots ``What matters in the world is money, machines and people, in that order. Our political task is to reverse the order.''\footnote{\cite[p. 6]{HartK2005thmd}} Zopa, Kiva and Open Capital are reordering our world, each in different ways. For Zopa, the rhetoric prioritises people, but people are not placed before money in practice: capital must be rented with interest. Kiva certainly prioritises people before money, then money before machines. The linking of a Kiva capital marketplace to fund infrastructure projects using the emergent phenomenon of Open Capital presents us, perhaps, with a model with does prioritise people, machines and money, in that order.
\medskip
\paragraph{}Words : 5748
\newpage
\singlespacing
\bibliography{/home/doitsysadmin/Documents/GoldsmithsCourses/bibliography/globalbib}
\end{document}