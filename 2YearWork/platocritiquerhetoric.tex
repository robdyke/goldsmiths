\documentclass{article}
\usepackage{geometry}
\usepackage{harvard}
\usepackage{helvet}
\usepackage{setspace}
\renewcommand{\familydefault}{\sfdefault}
\geometry{a4paper}
\title{Is Plato's criticism of rhetoric relevant today?}
\author{Student Registration No. 22164733}
\date{\today}

\begin{document}
\begin{doublespace}
\bibliographystyle{kluwer}
\maketitle
\begin{quote}In the history and philosophy of rhetoric, which overlaps with political theory or simply �philosophy�, the question of truth applied to the sphere of public deliberation, the �polis�, the social contract � whatever term is used �, is not new. Politics, rhetoric and truth have been linked ever since democracy took shape.
\begin{flushright}\cite[pp.13]{Salazar:2004fk}\end{flushright}\end{quote}

\paragraph{}In this essay I will show rhetoric as both an art and a science; a knack to cultivate and knowledge to acquire. I will examine Plato's dual layered criticism of rhetoric (a criticism that is not without criticism) and demonstrate the lessons that we should take from Plato that have relevance today and those that can safely remain in classical antiquity.

\paragraph{}To comment on the rhetoric of a person's language, or to call a person sophistic, has negative connotations. Such, apparently, is strength of condemnation in Plato's critique. Initially I too was of this way of thinking; condemning the rhetorician, slighting the sophist. But in reading for this question and around the subject of rhetoric in general I am persuaded of the purpose and value of rhetoric in political life. I follow in the footsteps of others who have come to this position and I will make reference to them in my essay.

\paragraph{}\possessivecite{Cassin:2005uq} etymological analysis of the word rhetoric was the first instrument in my conversion. Her definition of rhetoric also neatly summarizes Plato's critique\begin{quote}...the proper object of rhetoric is not truth: it is called pithanon, ``the persuasive''. ``Rhetoric is the ability to theorize in every (or  any) case the appropriate persuasive''
\begin{flushright}\cite[pp. 858]{Cassin:2005uq}\end{flushright}\end{quote}

\paragraph{}Plato is interested in truth. As a metaphysician his method is the reasoned discovery of truth; philosophic enquiry using structured argument to discover knowledge or \emph{episteme}. For Plato these truth and knowledge were universal concepts that enabled humans to define the existence of something.\footnote{A simple example. For humans to recognise an object for sitting on as a 'chair' from any of the other objects in the world there has to be a universal concept of a chair. This universal concept would distinguish the characteristics of a chair over, say, a fish - as a result we recognise that we sit on chairs and eat fishes and not the other way round.} The search for the universality of concepts such as justice, truth, equality, government was Plato's goal.

\paragraph{}Plato is concerned with finding universal truth, truth's \emph{essence}, in order that citizens might know truth and be virtuous and therefore be good citizens. Plato is not interested in anything that gets in the way of truth, so his critique of rhetoric, as taught by the sophist Gorgias and others, is an attempt to sweep away this method of debate because ``this sort of proof is valueless in the pursuit of truth.'' \cite[pp. 35]{Plato:1952lr}

\paragraph{}Why was it necessary for Plato to attack rhetoric? Political life in Athens centered around the \emph{polis} where matters of the city state were decided by the enfranchised few.\footnote{Athenian democracy was not the universal suffrage of the modern democratic society, with women and slaves excluded by political power} This assembly was both profoundly democratic and highly structured with frequent elections for civic positions and strict terms of office, in addition to participatory democratic decision-making. Returning to the quotation at the beginning of this essay; perhaps it was Plato who first linked rhetoric, politics and philosophy - linking good, virtue and truth - in \underline{Gorgias}. Truth, for Plato, was universal and not subjective. Universal `truth' is valued above subjective `opinion', which is a matter of belief, or \emph{pistis}. Plato's critique of rhetoric in \underline{Gorgias} is also,therefore, a criticism of Athenian democracy.

\paragraph{}Athens democratic `weakness' was that she accorded equal status to the votes of those at the \emph{polis} and therefore accorded equality to the opinion of each voting citizen. This weakness has two dimensions. Firstly Plato concludes that the opinions of the uneducated and uninformed, i.e. the non-professional citizenry, should not be accorded equal status with a professional on any given topic. Secondly, as rhetoric is not concerned with truth the adoption of rhetorical teachings by the political citizen is of grave concern for Plato. Practitioners of rhetoric heard in the \emph{polis} were not seeking a true vote from citizens. By persuading citizens of one course of action over another rather than seeking the \emph{true} action, the \emph{virtuous} decision, good government was being sabotaged. The assembled ``ignorant'' \cite[pp. 18]{Plato:1952lr} citizens are making the wrong decisions having been persuaded of a false action by rhetoric.

\paragraph{}So Plato's first criticism of rhetoric is that it is not concerned with \emph{universal} truth and therefore has no place in politics which is concerned with governing us all. The argument of Gorgias the sophist and of political theorists post-Plato is that politics is concerned with making truth from opinion. That is to say, truth - the right decision - in politics comes from the competition between relative opinions. Rhetoric, responds Gorgias, is but a training for that competition.

\paragraph{}A second aspect of Plato's criticism of Athenian democracy addresses the problem of scale. The scale of Athenian democracy, the numbers of participants in the processes of government, conditioned Athenians, in Plato's view, into irrational decision-makers.\footnote{The irrational paradox of democracy is discussed by \citeasnoun{Graham:2002vn} Chapter 2, ``Politics and Reason''} Yet Plato's alternative, his philosophical approach to government, is not suited to the competitive, majoritarian \emph{polis}. Indeed Plato concedes as much to Gorgias: ``For I know how to produce one witness to the truth of my assertions, the man himself with whom I am holding the argument (the others, the mob, I can dispense with); and I do know how to put the vote to one man at a time, though I will not hold conversations with a crowd.'' \cite[pp. 38]{Plato:1952lr} Rhetoric on the other hand is perfectly suited for persuading an assembled group of one course of action or another, as Gorgias tell us \cite[pp. 18]{Plato:1952lr}.

\paragraph{}Is Plato's \emph{inventio} sound? If I accept his argument that truth is absolute then I must therefore accept his conclusion that to persuade people of a falsehood is unjust, and so support Plato in his attack on rhetoric. In part I agree with Plato - the concepts he is interested in; justice, morality, truth \&tc, are universal. In a sense the well-travelled sophist Gorgias is agreeing with Plato too. Although having witnessed many cultures the sophist tends towards relativism rather than absolutism \cite[pp. 39]{Herrick:2005lr}. Relativism was embodied in the democratic culture of Athens, indeed relativism is embodied in the plural \emph{demos}.

\paragraph{}It is precisely because political decision making is not concerned with acting in accordance with absolute truth that persuasion is a necessary a part of politics. Politics is concerned with the mediation of subjective opinions/truths. Political life is a social, communal life. As \citeasnoun[pp. 194]{Latour:1997lr} askes; ``Are these not \ldots the normal conditions of the Body Politic? Is it not precisely to deal with these peculiar situations of number, urgency, and priority that the subtle skills of politics were invented.'' \footnote{Plato says, through Socrates, ``Rhetoric, in my account, is a reflection of a branch of politics'', a knack, a subtle skill \cite[pp. 24]{Plato:1952lr}.}

\paragraph{}The structures within which mediation takes place need to take into account the `peculiar situations' and democracy, rule of the \emph{demos}, is one such structure. In a democracy opinions are voiced and the \emph{demos} agree on a course of action through a vote.\footnote{I recognise that this is a basic, normative definition} The structure of mediation, the democratic assembly in both Athens and in modern states, represents agreement on the means of mediating opinions, between relative truths, rather than agreement on the ends, on absolutes to be pursued. An agreed mediation structure reduces conflict; political philosophers have observed that the want of one creates conflict.\footnote{Social contract theorists demonstrate a conflict between individuals that is permanently resolved by a rational group agreement on an external mediation mechanism. Although, as \citeasnoun[pp. 194]{Latour:1997lr} demonstrates \ldots ``The mythology of the war of all against all that threatens to engulf civilization if morality is not enforced is told only by those who have withdrawn from the people the basic morality that sociability has imposed for millions of years on animals in groups''. ``Rationality in the choice of means'', writes \citeasnoun[pp.45]{Habermas:1996qy} ``accompanies avowed irrationality in orientation to values, goals, and needs''. Plato had his own rational solution to the mediation problem; \underline{The Republic}}

\paragraph{}To recapitulate then, (before demonstrating Plato's relevance today) rhetoric is a skill used to persuade others of the (relative) truth of an opinion rather than to seek absolute truth. Plato thinks that rhetoric corrupts politics, more so in a  democracy, the arena of plural value systems. For Plato it is irrational to govern a society this way. 

\paragraph{}Remember here that Plato is writing this `dialogue' between Socrates and the assembled Sophists; not one word is wasted. Plato's warning to us is very clear. Rhetoric is concerned with words and words are in everything; e.g. medicine, mathematics, \emph{politics}\ldots\cite[pp. 7-9]{Plato:1952lr} Words are used to persuade us of truth, in both Gorgias's rhetoric and Plato's philosophy. The relevant lesson to take from Plato's criticism of rhetoric is his cold warning to watch for the manipulation ideas/truth by words: be wary and aware of the rhetoric of those who will pander to our prejudices for their own pernicious purposes.

\paragraph{}Plato highlights the rhetoric in mathematics and the sciences provides me a helpful argument. In Plato's time the earth was \emph{truthfully} flat. Mathematics has later proved the earth to be \emph{truly} a sphere. No rhetoric would have been able to convince the \emph{demos} otherwise in Plato's time. What I am trying to demonstrate by this is that the power of rhetoric, although strong, is limited by other factors, specifically it is limited by the current stock of accepted opinion, `truth', in a society. A more contemporary example would be the subject of global warming.\footnote{I challenge Plato to philosophically discern a universal truth that can instrumentally answer this pressing environmental issue.} Words are used to communicate the science, ideas and opinions on this subject and discerning the relative \emph{truth} of one argument over another and transposing this into action is the pluralist soup of modern government.\footnote{For further reading on science, the scientization of politics and a discussion of different models of decision-making processes, see \cite[Chapter 4]{Habermas:1996qy}}

\paragraph{}Having demonstrated Plato's criticism of rhetoric and the aspects of relevance to contemporary politics I now turn my attention to Plato's own rhetoric. In my introduction I stated his criticism was dual layered. I have shown the first layer of his criticism in the previous section. Yet there remains an aspect of Plato's critique that can not go unchallenged. The dialogue \underline{Gorgias} conceals Plato's contempt for the \emph{demos}.\footnote{See \cite{Latour:1997lr} for a full challenge to Plato and a defense of the collective morality of the \emph{demos}.} His dialogue establishes rhetoric as a skill used to conceal or reveal an opinion with words, used to persuade another of an opinion. He then demonstrates his skill with rhetoric, concealing his opinion that the \emph{demos}, ``the mob'', the ``ignorant'', and not capable of discerning and sustaining a shared morality. Gorgias is the democrat of the dialogue, equipping people with the skills for participation in mass politics. Polus and Callicles are concerned with power \cite[Ch. 3]{Herrick:2005lr} over the \emph{demos}, desiring perhaps a feudal system of competing aristocrats. Plato, through his mouthpiece Socrates and concealed in masterful rhetoric, slyly agrees with Calliciles; the only dispute is on the means to negate the \emph{demos} - Plato's solution to the problem of organising the people out of politics is his \underline{Republic} governed by enlightened philosophers. \citeasnoun{Latour:1997lr} staunchly defends the morality and the capacity of the \emph{demos} to self-govern\ldots
\begin{quote}``If there is one thing that does not require an expert, and cannot be taken out of the hands of the ten thousand fools, it is deciding about what is right and wrong, what is good and bad. But the Third Estate [\emph{demos}] has been turned, by Socrates and by Callicles, into a barbaric population of unintelligent, spoiled, and sickly slaves and children, who are now waiting eagerly for their pittance of morality, without which they would have ``no understanding'' of what to do, what to choose, what to know, what to hope.''
\begin{flushright}\cite[pp. 217]{Latour:1997lr}\end{flushright}\end{quote}

\paragraph{}So what, then, of rhetoric today?Our modern states are based on a rational agreement on democratic `means'. Modern political life is conditioned by belief, \emph{pistis}, rather than knowledge, \emph{episteme}. Mass-politics is truth, \emph{episteme}, through rhetoric rather than \emph{dialectic}.\footnote{See \possessivecite[Ch. 4]{Herrick:2005lr} table of the differences between dialectic and rhetoric, between knowledge (\emph{episteme}) and  belief (\emph{pistis}) as presented by Aristotle.} Because in a democracy the \emph{demos} must choose between the truth of relative opinions I conclude, like Cassin, that we, the \emph{demos}, ``need more rhetoric to help us against rhetoric''. \possessivecite[pp. 858]{Cassin:2005uq} illustration is Colin Powell's powerful demonstration of \emph{his} truth when addressing the United Nations, seeking a resolution to authorise military action in Iraq. We, the \emph{demos}, knew then there were no WMDs in Iraq. Later, the truth was revealed during the military action, and the rhetoric of Bush and Blair was proved false. As in modern politics \emph{truth} can be understood as what is concealed from the \emph{demos} by rhetoric, the \emph{demos} need more rhetoric to help against rhetoric.

\end{doublespace}
\newpage
\bibliography{globalbib}
\end{document}