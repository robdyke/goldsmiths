\documentclass[12pt]{article}
\usepackage{geometry}
\geometry{a4paper}
\usepackage{harvard}
\usepackage{helvet}
\usepackage{setspace}
\renewcommand{\familydefault}{\sfdefault}

\title{Politics and Rhetoric Group Speech Task}
\author{Course PO52011A : Group 1}
\date{}
\begin{document}

\bibliographystyle{kluwer}

\maketitle

\section{Candidate Numbers}
\begin{itemize}
\item 33009633
\item 33010016
\item 33012490
\item 33006643
\item 33000367
\item 33004641
\item 22164733
\end{itemize}

\begin{doublespace}
\section{Addressing our scenario}
\begin{quote}
``Justify ten years in office, despite being forced out in a wave of hostility from your own party and the public over failed policies''
\end{quote}
\paragraph{What was the argument of the speech?}
Our inventio was ``The King is Dead, Long Live The King!'' We were determined our Prime Minister would leave with his legacy properly defined, rising above the temporary strife, to be remembered as a great leader. We wanted those who had forced the PM to resign to feel foolish for engineering a coup d' etat. Our argument relied on demonstrating the steadfastness of the PM, his commitment to his values and a unity of values with the British people. We found our evidence in the popular support of the people given to our PM and his party over the last 10 years. A second piece of evidence to support our argument is our PM's commitment to democracy.

\paragraph{}From the outset we took this scenario as an opportunity for valedictory rhetoric, ending a political career on a high note. We would use the resignation speech to persuade our audience of the successful aspects of our Prime Minister's term of office in order to justify the past ten years in power. Our argument was wholly subjective - the personal character of our PM, his ethos was appealed to. We were to remember the last 10 years and not let the `last ten minutes', the temporary `wave of hostility', engulf his legacy.

\paragraph{How was this presented in terms of style?}We set our resignation speech in the House Commons. This scenario dictated a certain form of speech; the ceremonial conventions for speech in the Commons would have to be followed. We could have set our speech at a Downing Street press conference, scheduled, stage managed, delivered to world's media and perhaps even pre-leaked. This setting would have modified our style somewhat insofar as the ceremonial speech fitting of the Houses of Parliament would not have been required. As our speech was set in the House of Commons 'the people' were, on the whole, indirectly referred to, if we had delivered our speech direct to camera then 'the people' would have been addressed directly.

\paragraph{}Our PMs resignation speech battles to define the contents of the empty signifiers of `history' and `justice'. Our PM drew this audiences attention to periods and actions in that he felt would secure a positive definition of  `justice'. The issue of the day, the catalyst for resignation, is only indirectly referenced. In doing so we are demonstrating only the positive actions of our PM leadership and not drawing attention to negatives. We are focusing the attention of our audience on the topics our PM choosing, rather than addressing the substance of the present ``wave of hostility''.

\paragraph{}Our speech did not deal in intricacies of the policy or the details of the accomplishments; this was no ordinary speech, delivered at Prime Ministers Questions, where statistics have to be on hand to score political points. This was a battle for a legacy - was the present circumstance of hostility going to tarnish the unblemished record of commitment? The wider context, the bigger picture, was represented in order to refocus the memory of the audience, to define the history and win for our PM a positive legacy in history.

\paragraph{}Our scenario stated that our PM was being "forced out in a wave of hostility". The wave of hostility over a failed policy is referred to by our PMs statements on the purpose of government. "Tough decisions" had to be taken and somebody had to do it. Some of the feedback on our speech was that is was severe in parts; certainly an indignant tone crept in to some sentences - our Prime Minister is being forced to resign and he perceives this as unfair treatment.

\paragraph{}Our Prime Minister made frequent references to the values; his own and those of the people of Britain. He has a faith in democracy and in democratic culture, criticism and debate. We use this to draw a positive conclusion out of the negative situation. Our PM believes he is maintaining his commitment to democracy in his resignation, in this he is vindicated. We illustrate a unity of purpose with his policy and the purpose of the British people and that his history is one of successful delivery of policy that gained and maintained popular support. In a sense, our PM is genuinely shocked at having to resign, bewildered as to the loss of his political support. He demonstrates his simple inductive argument - `if you look at all of these things you'll see I have done a good job'.

\paragraph{}Personifying the scenario given to us created a challenge in that we discussing ideas for the speech our discussions sometimes centered on whether or not the person we envisaged would or would not speak in a certain way. Personification was a strength for our speeches style, it grounded us on a person and gave us a narrative, sounds bites and quotes to reuse, e.g. ``history will be my judge''. We modeled our style on the clipped iambic pentameter of Tony Blair. There were short sentences. Punctuated with abstract nouns. And pauses. Declaratives.

\paragraph{}This style is best demonstrated by the two sections of our speech on the evidence to support our Prime Ministers claim to a positive legacy. The first, where our PM defines his history, answering the unspoken rhetoric question: ``Just what has this Prime Minister ever done for us?'' His list of declaratives attempts to prompt a positive recollection of a benefit that any member of the audience has derived as a result of his leadership. In second section, where our Prime Minister secures credentials as a social democrat, he first makes an appeal to his ethos via a metonymy of the leader of a whole nation. The anaphora (principles\ldots) and epistrophe (\ldots values) used in the following two paragraphs further secures our PMs ethos.

\paragraph{}We understood from the principle character in our scenario that the audience for this speech was made up of many constituencies. There was the multi-dimensional political party, behind the Prime Minister: all members of the party \emph{physically} behind him due to the layout of the House of Commons; some behind him to support him, others to stab him. Opposite the Prime Minister would be The Opposition parties; eager to make political capital from the Prime Minister's difficulties. Of course the media; pundits, commentators and lobby hanger-ons\ldots The British public are the most significant constituent of the audience. 

\paragraph{}We wanted to keep a good momentum in our speech and build to a crescendo. We admired and sough to recreated what we loosely coined `the Thatcher effect' (\cite{Thatcher:1990fk} ``I'm enjoying this!'') - we wanted our PM to enjoy the speech he was giving as it would be his last as PM. We sought the right peroration, attempting to renew enthusiasm for our Prime Minister from the audience. We achieved this by not revealing the subject of the speech until the final paragraph - this is perhaps why some of our feedback from fellow students mentions that our speech looses coherency at the end. This is because our audience are shocked by the magnitude of the announcement - until the resignation is revealed as the subject of the speech the audience are waiting. In real life of course, perhaps they would be aware of the 'wave of hostility' and the audience might be expecting a resignation, but it is possible that we could have announced another subject. The revelation at the end is the culmination of the tension and anticipation that we have built.
\begin{quote}
``The King Is Dead! Long Live The King!'' 
\end{quote}

\section{Sources}
\paragraph{What sources did you use and how did you make use of them?}
\begin{itemize}
\item \cite{Wikipedia:fk} article \underline{Resignation Speech} gave us links to notable political resignation speeches
\item Notes from Lectures and Seminars (specifically\ldots)
\begin{itemize}
\item Plato, Aristotle, Cicero and Quintillian
\item Populism
\end{itemize}
\item \possessivecite{MacArthur:1999yu} collection of speeches
\item \possessivecite{Fairclough:2000lr} \underline{New Labour, New Language?} 
\item The Parliamentary Record as published in \citename{Hansard:rm}. We used this resource to research the cerimonial forms of speech in the House of Commons.
\item BBC Today Program Podcasts of interviews with Tony Blair \cite{interview:2007qf,interview:2007jk}
\item \possessivecite{Yeltsin:1999qv} resignation speech 
\item \possessivecite{Blunket:2005ty} resignation speech
\item \possessivecite{Nixon:1974sf} emotional resignation speech. Perhaps due to differences in political culture, but also more likely to the specific circumstances of his resignation, we felt this style of address was inappropriate for our Prime Minister. We also borrowed from Nixon his ``constitutional purpose''.
\item Ms Thatcher's speech to the House of Commons
\item Major's ``put up or shut up'' \cite{Major:1995xy}.
\end{itemize}

\paragraph{}The BBC Today Program interviews with Tony Blair \cite{interview:2007qf,interview:2007jk} were particularly useful as a source. Tony Blair was pressured by John Humphrys on recent government policy `failures', each attack was deflected and redefined. In terms of rhetoric, that is, seeking to persuade, these interviews were a rich resource of the language and style used in such a hostile situation.

\section{Working as a group}
\paragraph{What difficulties did you come across in writing the speech?}Initially working as a group held out a challenge to us due to preconceived difficulties. When we actually began working we found that we rapidly built a good working relationship and our fears were allayed. A challenge in group work is avoiding the ``freeloader problem''. Our group worked well together and we did not experience this issue.

\paragraph{}We had a positive experience as a group. We communicated well, on the whole, using electronic tools to discuss our task and work on our speech, extending our working from the face to face sessions to other times. Our group had good attendance from all members. Our organisation of 7 people to complete this task, against backdrop of other pressures, was strong and we completed the task within time and without rush.

\paragraph{How were these overcome?}A significant benefit for us a group was the use of an online collaboration tool (a wiki\footnote{A wiki is a website that allows visitors to add, remove, edit and change content. This ease of interaction and operation makes a wiki an effective tool for mass collaborative authoring. The term wiki also can refer to the collaborative software itself (wiki engine) that facilitates the operation of such a site, or to certain specific wiki sites such as Wikipedia. \cite{wiki:fk}}). This tool enabled us to share a concurrent master version of the speech, without having to exchange attachments in email or worrying about how each individual contribution would fit with the contributions of others in the group. That the tool was online made us more productive as a group - we were able to work together without needing to physically meet. The electronic tools fostered intra-group communication, allowed us to work independently and yet as a group at the same time.

\paragraph{}The wiki gave each of us the opportunity to 'hold the pen' at the same time. While drafting and redrafting, editing all of it together as a group was made possible by the wiki tool. The tool helped us to capture the large volume of source material we generated. The hard and brutal task of revising and reducing a body of content/ideas to a coherent draft of a speech was helped as we could all see all of the edits all of the time.

\bibliography{globalbib}
\end{doublespace}
\end{document}