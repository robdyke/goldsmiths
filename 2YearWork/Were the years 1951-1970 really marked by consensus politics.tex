Were the years 1951-1970 really marked by 'consensus politics'? 
Harold Wilson, the Labour Prime Minister at the end of the 19 years framed in this 
question, once said that 'a week is a long time in politics'. The period 1951-1970 is a long 
time and to claim ascendancy for a particular characteristic over others risks limiting 
debate. Yet political scientists, journalists, politicians and academics commenting on the 
post-War pre-Thatcher period of British politics can be roughly divided into those who think 
that the period can be described  as being marked by 'consensus politics' and those that 
do not. For me, the nays have it. I will attempt to wrestle a definition of the term out from 
the literature and I will show that 'consensus politics' does not apply to this period of British 
political history. 
The term 'consensus politics' is difficult to define because it has been used in many 
different ways by different agents at different times. In his typographical analysis of 
consensus, Kavanagh1 lists four definitions: consensus on ends, consensus on style, 
consensus on agenda and consensus relative to other periods. This typography might be 
helpful in teasing out a definition of 'consensus politics' from the wide debate and I will 
briefly discuss and give examples of each of these types in search of a synthesis of the 
idea of 'consensus politics'. 
Many of those who believe that the period in question is marked by 'consensus politics' 
give examples of an apparent consensus on the goals of government from one 
government to the next. The 'consensus of ends' is usually supported with evidence that 
the goals of the Labour government of 1946; full employment, a welfare state and a mixed- 
economy, managed within a Keynesian economic framework 2; were not discontinued by 
the incoming Conservative government of 1951. The argument for 'consensus politics' 
based on a 'consensus of ends' is supposedly reinforced by the analysis that these policy 
goals were continued into the second post-war Conservative government, elected in 1955. 
Those holding the view that a  'consensus of style' marked this period are doing so, in the 
view of Young3, in order to support a revisionist agenda. By painting the 19 years under 
examination as consensual, these commentators are able to make a contrast with the 
supposed socialist radicalism of Atlee and his government in the immediate post-war 
period and the success of later conservative governments. For Kerr, the consensus thesis 
1 Kavanagh, (2002) 
2 Hickson, (2004), p. 148 
3 Holmes, (1989)
is an �heuristic framework designed to highlight specific contrasts between Thatcherism 
and its historical antecedents�4. 
'Consensus of agenda' is explained by Kavanagh as the organising of topic out of the 
political agenda. This is a negative understanding of 'consensus politics'. I find this type of 
consensus the most difficult to form into a cogent argument. Parsons5 understands the 
policy making process as always reflecting the organisation of issues from a plurality of 
interest groups into or out of the policy making arena. To define the period 1951-70 as 
being marked by a 'consensus of agenda' therefore is not saying much of use: the policy 
making agenda is continually reshaped and redefined as a response to issues as they 
arise. Any perceived 'consensus of agenda' appears to me to be politics as usual rather 
than an exception worth marking. 
The final typographical distinction is of consensus relative to other periods in history. 
'Relative consensus' has weaknesses similar to those found in the arguments for a 
'consensus of style' and can be deployed, as noted above, to support specific political 
agendas that become all to apparent on examination. Drawing contrasts or highlighting 
similarities between the intricacies of one period with and another in order to support a 
theory of consensus or otherwise can often tell us more about the author than it does 
about the periods contrasted. 
So far, examining these threads in 'consensus theory' has not got us any closer to a 
definition. If anything Kavanagh's typography has changed the parameters of the debate 
on 'consensus politics'. In the place of rigorous investigation into the substance of the 
arguments of those who see 'consensus politics' as marking this period is the examination 
of the type of consensus debate. While his typography is a helpful prism, its continued use 
risks a de facto consensus on 'consensus politics'. 
In order for a term to be of use in describing a period it needs to accurately describe a 
period. My argument against 'consensus theory' is based on the limited utility of term. Yet 
'consensus politics' continues to have currency (despite public floggings6) and the era in 
question continues to be examined. Would another term better describe the period 1951- 
70? I think that 'osmosis' better describes the politics of the period. My synthesis of some 
of the ideas in the 'consensus politics' debate follows in support of my observation. 
4 Kerr, (1999), p. 73, emphasis added 
5 Parsons, (1996) 
6 After Butler, (1993)
Taking 1951-70 in isolation discounts contrasts with other periods. Yet taking these years 
in the context of the socio-political landscape that preceded them is necessary in order for 
us to understand the importance of the political themes of the Fifties and Sixties. World 
War had ended in 1945 and the Labour Party won the General Election with a massive 
majority on a manifesto based around the immediate implementation of welfare measures 
that were proposed during the war-time coalition Government, principally the Beveridge 
Report of 1942. That the Labour Party won this election is significant. Pre-war 
governments were usually formed by the Liberal Party. The Conservatives were the major 
party in the war-time coalition. A significant point in 'consensus theory' literature is the 
Labour victory and I draw attention to it now to highlight not the 'radicalism' of the Atlee 
government of 1945-51 � (as the Conservatives would have implemented Beveridge 
Report recommendations sooner or later)7 � but a centre-left swell in the nation.  Did 
Labour win on the back of a realignment of politics to the left-of-centre in part due to the 
war-time experience8 of the nation? Kavanagh argues that this observation is exagerated9, 
but the strength of electoral support for a social-democratic party of Labour with a 
manifesto of nationalisation and greatly increased state provision of services, indicates this 
this is the case.10 
The Nation had voted ostensibly for social democracy by endorsing 'Beveridge Now' in the 
1945 general election. This was a significant shift to the left for the state and its apparatus: 
for example, in establishing the various departments needed so administer the welfare 
payments and the National Health Service. The Civil Service was traditionally conservative 
and was geared towards the administration of a nation at war; the agenda of the Labour 
government of 1945-50 was  a major shift for it. 
With this background of significant change in society and government, let us return to the 
period in the question. In 1950 the electorate decided to return Labour to government. The 
Conservatives were at this time unable to define themselves in the prevailing order of 
society. Later the Conservatives pledged to end the 'queuetopia' of Britain under Labour 
7 Kavanagh, (1997), p.31 
8 See Frazer (2000) p. 348 
9 Kavanagh, (1997), p.31 
10Election results sourced from Goldsmiths VLE
and won the election in '5111. They wanted to run Labour's socialism more efficiently12, and 
initially only minor changes to domestic policy in practice were made. This continuation of 
policy signals to me an 'osmosis in politics' rather than 'consensus (in) politics'. There was 
no consensus developing from Churchill's second term in office. Besides, many of the 
policies around which some authors see a consensus began to take shape within 
Churchill's wartime coalition government; the best example of this is Beveridge report from 
1942: 
�The argument between Tories and Socialists is not whether the welfare 
state is a good or bad thing; the scramble is for the credit of having invented 
it. No British politician would now suggest abolishing it, any more than he 
would suggest abolishing state-paid policemen or firemen.�13 
Perhaps there is no 'consensus' to be found in this early period, rather simply an 
acceptance and a pragmatic continuation of 'radical' Atlee policy? If the Conservative 
continuation of policy signals agreement with it, this is because they were involved in the 
initial formulation of policy. I find the argument that agreement on domestic policy signals 
consensus has weaknesses for this reason. 
The party of social democracy, Labour, were to be kept out of power by the electorate for 
the following twelve years, through two general elections, and by a significant margin each 
time. The Conservatives continued the expenditure commitments of the welfare state and 
the  National Health Service and began to diminish the 'queuetopia' of post-war Britain, 
ending rationing on various goods in their first government of the 1950s. Rationing was a 
war-time measure limiting the supply of goods. Continuing it in to the 50's was a necessity 
for the economic growth of the nation, but it was never intended as permanent or the 
austerity measure it became. Government would have ended rationing; no party can claim 
credit. I make this observation to show that in some aspects of domestic policy, when 
'consensus' is understood as simple agreement on practical measures, 'consensus politics' 
can be seen in the period 1951-70. 
The economy and economic policy is a area where 'consensus theory' makes significant 
11Time Magazine   �Osmosis in Queuetopia�: The Tory campaign manifesto tried to sound as Laborite as possible and 
the Labor manifesto tried to sound as Tory as possible. This political osmosis, familiar in U.S. politics, is relatively 
new to Britain. Unless the Tories can break out of the "me-too" pattern and wake the voters to a sense of Britain's 
economic and political perils, it is hard to see how the Tories can win.� This observation was made 1950, by 1951 
the Conservatives were returned to power.  
12Ibid 
13Ibid
steps towards coherence. An observation was made during the early 1950s to the effect 
that whichever party was in government  there was little change in handling of the 
economy. The term 'Butskellism' was coined at this time to show the small difference 
between the outgoing Labour Chancellor Gaitskell and the incoming Conservative Butler. 
The apparently indistinguishable approach of these Chancellors leads some authors to 
present this as  consensus in these early years. 
Other issues arose to test the Conservatives, and therefore the 'consensus politics' thesis, 
during the 1950s. Early in the first government denationalisation of the industries aken into 
public ownership by Labour happened. De-nationalisation shows a major departure from 
the economic policies of Labour and is also a marked contrast with Labour's philosophy of 
the role of the state in the market. Yet not all nationalisations were reversed: this 
demonstrates to me not 'consensus' but 'osmosis': the Conservatives and Labour both 
practising politics that gave good example of their respective philosophies of government 
and the state, but within the boundaries of the policy arena and with pragmatic support for 
policies from 'the other side' that were working. 
Related to industrial policy was labour policy and the goal of full employment. The 
Conservatives had to deal with the rise and rise of union power during their three terms in 
office. Early in the 51-55 government Churchill adopted a conciliatory approach to 
unionised striking rail workers, appearing to be in agreement with Labour on the remedy 
for industrial unrest. Whatever consensus this agreement signifies was diminished by the 
end of the third Conservative government (1959-64) when industrial strife and rising 
unemployment again dominated the political agenda. These challenges were to prove too 
great for the Conservatives: times had changed and any consensus on union relations had 
broken down. Labour, the traditional  party of organised labour, was returned to 
government in the May of 1964. 
Time in opposition had been time for introspection for the Labour party. Indeed it appeared 
that a new strain of British Socialism was emerging, one that had reconciled itsself with the 
market as the best means for creating wealth which could then be redistributed14. That 
said, the steel industry was renationalised when Labour came to government. For some 
though this was a symbolic gesture to the Left of the party. Gesture or otherwise, this sole 
nationalisation project does not amount to a wholesale reversal of the Conservatives round 
of privatisation; and as Labour had arguably revised its approach to democratic socialism 
14 For a discussion on the 'Future of Socialism' and the redefinition of Labour during the 1950s see Crosland, A. For 
an assessment of the revision this represented see Briggs, A. in http://tinyurl.com/y7lzgp
through market mechanisms neither does it indicate 'consensus politics'. 
The relationship between government and unionised labour continued to be of importance 
for the Labour governments of 1964-70. The relationship between government and labour 
was, as noted above, difficult for the Conservatives and continued to be turbulent for 
Labour. By the middle of the sixties full-employment appeared to have been dropped from 
the agenda, perhaps by a 'negative consensus', but more likely due to the frequent 
clashes with trade unions on economic policy, (for example, wages policy in 1968), in 
times of pressure on the economy. Regulation of trade union activity became seen as 
increasingly necessary. In 'In Place of Strife'15 the Labour government proposed a series of 
measures which would have restricted trade union activity. The proposal in the White 
Paper never became legislation: Labour left government in 1970 and the Conservatives, 
who perhaps had needed the party of labour to reform labour relations, did not take this 
proposed legislation forward when entering government in June.  
Another area of interest to consensus theorists and myth-busters alike is British policy 
towards Europe and European cooperation projects. The 1950s saw healthy debate and 
disagreement both between and within the two major political parties on the role of Britain 
in relation to the project of economic and political integration on the continent. The 
Conservatives toyed with both participation and alternative multilateral approaches, such 
as the European Free Trade Association (EFTA) agreements made in competition with the 
European Economic Community (EEC). Labour during the 1950s was critical  of 
applications to the EEC, yet later, in the 1960s, became more pro-European. Governments 
of both parties made applications to join the EEC during the period in the question and 
both were rejected. A place in Europe was finally secured by the Conservatives in 1973, 
later renegotiated by Labour in '74 and with a referendum on membership proposed and 
won in 1975. Here we can see the long time line of European relations and see that both 
parties had times of confidence and times of crisis � these are, perhaps, the only 
similarities, between these actors over the period. 
Osmosis can again been seen in debates on Europe. There were shifts in the policy of the 
two major political parties, showing a gradual process of absorption of ideas and politic 
from each other and from society at large. EEC membership was not secured until after 
the period in the question.  That both political parties played the roles of advocate and 
detractor over this whole time line shows a lack of 'consensus politics'. 
15Government White Paper Cmnd 3888
In my introduction I stated that I did not find 'consensus politics' as marking 1951-70. I 
have demonstrated the utility of understanding this period with the prism of 'osmosis' - 
although this description is limited as it concentrates on the point where the political parties 
meet, the centre ground of British politics, and therefore downplays the spectrum of 
opinion within political parties. There is, after all, great complexity in all political periods 
and whatever term is used must reflect this. 'Consensus politics' is blunt and is limited in 
describing the contradictions and conflicts in politics. For Kerr 'consensus politics' needs to 
be abandoned completely as he see it as a constraint on understanding the complexity of 
British politics: �In its place, we need to begin to reappraise the dynamic and constantly 
evolving relationships and conflicts which structured the development of postwar public 
policy. This would enable us not only to develop a more sophisticated understanding of the 
earlier postwar period, but also to generate a better conception of the Thatcher years and 
beyond.�16 
Word count 2734 
16Kerr, (1999), p. 85
Bibliography 
Butler, A., The End of Post-War Consensus: Reflections on the Scholarly Uses of Political 
Rhetoric, in Political Quarterly, 64, 4 (1993), pp. 435-446 
Fraser, D., The Postwar Consensus: A Debate Not Long Enough? in Parliamentary Affairs, 
53, pp.347-362 
Hickson, K., The Postwar Consensus Revisited, in Political Quarterly, 75, 2 (2004) 
Kavanagh, D., The Postwar Consensus, in Twentieth Century British History, 3, 2 (1992), 
pp. 175-190 
Kavanagh, D., (1997), The Reordering of British Politics, Oxford, Oxford University Press 
Kerr, P., (1999), The Postwar Consensus, in Postwar British Politics in Perspective, 
Cambridge, Polity Press 
Parsons, D. W., (1995), Public Policy, Worcester, Edward Elgar Publishing 
Young, M., & Holmes, M., Controversy: Health's Government Reassessed in 
Contemporary Record, 3, 2 (Nov. 1989), pp24-27 
Time Magazine, 'Osmosis in Queuetopia', February 6th 1950, 
http://www.time.com/time/magazine/article/0,9171,811812-8,00.html , Accessed 
December 2006