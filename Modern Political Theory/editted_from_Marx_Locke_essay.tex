
\paragraph{}collective ownership. the stupification of capitalism has led to a situation when an object can only be appreciated when a exclusive annexation of it has been made. This is an illusion, not all enjoyment is related to ownership. for example - the view of nature's horizon, public art.

\paragraph{}Centrality of labour to Locke and Marx. Locke's highest human activity is to labour and to cultivate the earth. However, since Locke's time the compacting of labour as a force alienated from humanity through waged employment and the division of labour has tended toward drudgery for the proletariat. Valuing labour like a commodity and trading it in a consensual market system, devalues labour \cite[pp. 12]{Marx:2004yu}. Humanities explicit consent to the alienation our themselves is something that frustrated Marx and this can be seen in the urgency of his writing. Under the conditions of capitalism the value of the human as an individual is increasing found though their labour, through the identity of 'labourer'. Through the open competitive market place for labour the worker is disposed from their worth, by the purchasing of labour at less that its true value - the labourer must earn less per commodity than the final sale price of the object in order to the capitalist, the employer, the owner of the means of production, to make a profit.

\paragraph{}majority people don't have own property - they hold a right to it be can not exercise this right. a worker does not own what they produce. only a minority own property. there is, therefore, in Marx's view structural inequality. Tendency in capitalist system, tension in private property ownership, increasing entropy; this is caused by the impoverished conditions of the workers both in the material and socio-political senses. class struggle.

\paragraph{}Locke calculates that cultivated land yeilds more than uncultivated land, an observation that would not be disputed by Marx. This very transition from foraging to the organised cultivation of land to sustain human existence demonstrates Marx's example of a revolution in the mode of production. Paid employment, the hiring of labour to cultivate these lands is a good thing for Locke - upholds the as much and as good proviso. Labour can be bought and sold. Hired labour produces more objects that the owner needs. Selling objects for money, purchasing objects for money - legitimate property transfer - prevents objects spoiling, upholding Locke's other acquisition proviso. For Marx however this entire system only sustains and reinforces the power of the owners of the cultivated land.

\paragraph{}paid employment and the money system estranges the labourer from the object to which his or her labour is mixed. the creation of commodities (Labour+Materials= Commodities) commodifies also the labour of the labourer. increasing commodification of labour and the competition for employment of labour 'devalues the world of men, increasing the value of things'\footnote{\cite[pp. 71]{Marx:1844qf}}

\paragraph{}As \citename[pp. 160]{Nozick:1974lr} points out, the production of wealth/property and the distribution of wealth/property are not separate and independent questions. Objects to not appear out of nothing - they ``\ldots come into this world attached to people having entitlements over them.'' The is a relationship between entitlement (to nature and to property) and Lockean property theory. not increasing the entropy. Marx agrees and argues that the entitlement is with the worker, the one who mixed their labour to create the object.

\paragraph{}Marx observes a paradox in appropriation. The worker can create nothing without the raw materials of nature, either directly in the case of a coal miner or indirectly in the case of a 21st century call centre operator. Nature provides labour with the means of life. Appropriating from nature deprives labour of means to life.

\paragraph{}Do private property rights backed by a theory of justice in acquisition and transfer legitimate greed in society? Nozick p. 179's Lockean justice based model. The lassie-faire free market model with its built in mechanisms of price control through scarcity value \footnote{invisible hand, p. 179-181,this is all the intervention the market needs}prevent the accumulation of all of a substance.

\paragraph{}In Locke's own writings do not see scarcity as an issue, indeed he makes great use of the abundance of natures provision. Locke's 'as much and as good' proviso would allow for the accidental private accumulation of all of a commodity. If by misfortune an individuals private property came to be the last example of that object, then so long as all acquisitions and transfers were just, private monopoly of enjoyment is legitimate.

\paragraph{}Nozick principles of distributive justice
\begin{itemize}
\item current time-slice
\item matrix based distribution
\item historical principles of justice
\item pattern-based distribution, criteria for determining who gets what and when such as by IQ, moral worth, usefulness...
\item end-result principles of justice, for example socialism. regulating the workers entitlements from productive process. require continual interference in the lives of individuals in the socialist society
\end{itemize}