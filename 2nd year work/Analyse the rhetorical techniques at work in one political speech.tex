\documentclass{article}
\usepackage{geometry}
\usepackage{harvard}
\usepackage{helvet}
\usepackage{setspace}
\renewcommand{\familydefault}{\sfdefault}
\geometry{a4paper}
\title{Analyse the rhetorical techniques at work in one political speech of your own choice.}
\author{Student Registration No. 22164733}
\date{\today}

\begin{document}
\begin{onehalfspace}
\citationstyle{dcu}
\bibliographystyle{agsm}
\maketitle
\paragraph{}The political speech I have chosen to analyse for its use of rhetorical techniques is the Personal Statement given to the House of Commons by Mr. Robin Cook M.P. (Livingston). This speech was Mr Cook's statement to parliament on his resignation from the position of Leader of the House of Commons. He had resigned in protest against his governments decision to go to war in Iraq without a second United Nations mandate.  In order to analyse this speech I will consider both the eluctio present in this address  and the inventio of his argument. 

\paragraph{}His address was delivered to colleagues in the House of Commons, yes, but it was also directed at the population of the nation. The speech retraces the recent history of the debate around military intervention in Iraq and seeks to persuade those pro-war of the folly of their convictions and the inconsistencies in their arguments. He begins his speech with a little light humour and uses the usual ceremonial forms of address relevant to the situation, before introducing his theme: why [he] cannot support a war without international agreement or domestic support. 

\paragraph{}Paragraph 5 of his speech features a number of rhetorical devices. First using antistrophe (more time for inspection) to stress the importance of the work of United Nations weapons inspectors. His use of the expletive indeed here places great emphasis on the following sentence. Also, he repeats the word time in order to show the antithesis present in the contrast between the desires of our neighbours and usual coalition partners and that the unilateral approach being favoured. The final sentence of this paragraph concludes with polysyndeton device, not to further emphasis the isolation of the UK's position.

\paragraph{}In paragraph 6 parallelism is deployed to good effect. Here multilateral agreement and a world order governed by rules are made parallel. This equality is contrasted with unilateral  action. His unspoken message here is unilateralists do not support government by rules. He also uses the metaphor of 'reverse'  to describe the disintegration of coalition relationships: the unilateral course does not take us forward to our goal. 

\paragraph{}Paragraph 7 shows procatalepsis. Mr. Cook's argument is moved forward and he attacks an argument of his opposition; the parallelism made between military action in Kosovo and in Iraq. He uses anaphora (It was support by\ldots) in explaining how the parallelism is undermined. 

\paragraph{}Here he undermines a major premise of the pro-war argument: I hope that Saddam, even now, will quit Baghdad and avert war, but it is false to argue that only those who support war support our troops. It is entirely legitimate to support our troops while seeking an alternative to the conflict that will put those troops at risk. (paragraph 9) Robin Cook's experience in foreign policy is not disputed, but to remind us of his authority in this subject he asks us to recall his four years of service as Foreign Secretary. He has specific knowledge of the western strategy of containment. His ethos is reinforced by this. He tells us of the success of this strategy and induces our agreement in his assessment: a successful weapons destruction program means there is no threat. 

\paragraph{}In paragraph 11 there are two occurrences of irony. The first is as an expletive to begin the paragraph. The choice of ironically as the expletive is deliberate. It signposts the real irony: We cannot base our military strategy on the assumption that Saddam is weak and at the same time justify pre-emptive action on the claim that he is a threat. The term 'weapons of mass destruction' had taken a great deal of discussion to define. Mr Cook reminds us of this debate when he makes a distinction at the beginning of paragraph 12: Iraq probably has no weapons of mass destruction in the commonly understood sense of the termnamely a credible device capable of being delivered against a strategic city 
target. 

\paragraph{}Paragraph 13 contains the most systematic deconstruction of his oppositions arguments. 

\begin{quote}
Only a couple of weeks ago, Hans Blix told the Security Council that the key remaining disarmament tasks could be completed within months. I have heard it said that Iraq has had not months but 12 years in which to complete disarmament, and that our patience is exhausted. Yet it is more than 30 years since resolution 242 called on Israel to withdraw from the occupied territories. We do not express the same impatience with the persistent refusal of Israel to comply.
\end{quote}

\paragraph{}Here Mr Cook attacks the syllogism of the argument that 'defiance of United Nations resolutions leads to military action; Iraq has defied the UN; there must be military action in Iraq to impose the will of the UN'. Mr Cook asks us to reach our own conclusions about the 
inconsistencies in that argument while he continues: I welcome the strong personal commitment that the Prime Minister has given to middle east peace, but Britain's positive role in the middle east does not redress the strong sense of injustice throughout the Muslim world at what it sees as one rule for the allies of the US and another rule for the rest. 

\paragraph{}These last few paragraphs have thinly veiled from us who Mr Cook sees as being the primary actor in the moves towards was in Iraq. His metonymy of 'Washington' is teasing us. But in the erotesis posed in paragraph 15 leaves us with no doubt that he sees the President of the United States as responsible: What has come to trouble me most over past weeks is the suspicion that if the hanging chads in Florida had gone the other way and Al Gore had been elected, we would not now be about to commit British troops. In his closing sentences Mr Cook makes an appeal for his argument to be heard by the British people, with their good sense and collective wisdom and sound mood, not to 
provoke an uncritical response, but to demonstrate that he is speaking up for a popular opinion. 

\paragraph{}His final paragraph uses fragments of the appropriate ceremonial speech for addressing the House of Commons before going on to challenge his fellow MPs: It has been a favourite theme of commentators that this House no longer occupies a central role in British politics. Nothing could better demonstrate that they are wrong than for this House to stop the commitment of troops in a war that has neither international agreement nor domestic support. I intend to join those tomorrow night who will vote against military action now.

\cite{Cook:2003nr}

\end{onehalfspace}
\medskip
\bibliography{/Users/robdyke/Documents/GoldsmithsCourses/bibliography/globalbib}
\end{document}
