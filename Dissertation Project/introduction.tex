\chapter{Introduction.}
\paragraph{}The dynamic between free-software and open-source is often misunderstood by social and political theorists. As a consequence it is also under-theorised within socio-political theory. In this paper, I show how philosophies of free/libre, open-source and commons regimes have engendered new forms of sociopolitical consumption and new political economies of meaning. My emphasis on the interplay between the local and the global/structure and agency, shows new ways of `thinking' the cosmopolitan, sedimented in the interconnected networks of the technical age. My thesis is concerned with our present moment of opportunity. I believe that positive possibilities for politics and political economy are presented in the philosophies of free/libre, open-source and commons regimes. In this paper I will demonstrate the contribution of these new sociopolitical categories and the new politics that is being made public because of free/libre hacking.

\paragraph{}My thesis explores two themes: networks and movement. I recognise in the literature around social movements, global politics and government a similar interest in networks that develops in tandem with advances in physical and technical networks, such as transportation, computing or utilities. I, too, am interested in networks in both social and technical senses of the concept. According to Parsons the network concept was attractive to social movement and public policy theorists of the 1950 and 1960s because this model was flexible enough to describe the fluid and complex interplay between both the formal and informal political and social relationships that condition political culture.\footnote{\cite{parsons:1995}} During the 1980s the network concept was developed by a number of French sociologists. I have been particularly interested in the work of Bruno Latour in this area. He is associated with the Actor-Network-Theory (ANT), a sociology which claims that ``modern societies cannot be described without recognizing them as having a fibrous, thread-like, wiry, stringy, ropey, capillary character that is never captured by the notions of levels, layers, territories, spheres, categories, structure, systems. [ANT] aims at explaining the effects accounted for by those traditional words without having to buy the ontology, topology and politics that goes with them.''\footnote{\cite{latour:1998ant1}}

\paragraph{}In terms of global society, the network metaphor has been widely deployed in the field of international relations, economics, anthropology and socio-political geography. Most prominent is the geographer Manuel Castells whose seminal work \textit{The Rise of the Network Society}\footnote{\cite{Castells:1996ns}} considers the social and political dynamics created by the flows of people, goods and services, and capital around the world in terms of networks. More recent examples of the use of the network concept can be found in the ethnographies of \citename{Appadurai:1996lp},\footnote{\cite{Appadurai:1996lp}} the political/cultural research of \citename{Terranova:2004ly},\footnote{\cite{Terranova:2004ly}} and the socio-political geography of \citename{Barry:2001ff}.\footnote{\cite{Barry:2001ff}} The network concept has helped these authors to move beyond the `ontologies, topologies and politics' of world systems theorists, hyper-globalist free-marketeers and models of dependency and development.

\paragraph{}Most crucial in the ANT approach is that this sociology includes the material; that is to say it recognises ``the facts manufactured by natural and social sciences and the artefacts designed by engineers.''\footnote{\cite{latour:1998ant1}} Some may be critical of this approach, fearing a technological determinism or a cyborg future. However, I agree with Benkler who, in arguing that strict technological determinism\footnote{Understood as ``if you have technology `t,' you should expect social structure or relation 's' to emerge.'' \cite[pp. 11-12]{benkler:2006wn}} is false, states that ``different patterns of adoption and use [of a certain technology] can result in very different social relations.''\footnote{\textit{Ibid}., pp. 11-12} Benkler finds wealth in the network and not in nations, as Adam Smith did. His argument in \textit{The Wealth Of Networks} considers both the technological and social aspects of relational (re)production in networked society. 

\paragraph{}The closely related ideas of free/libre, open-source and commons regimes have developed in a similar time-frame to the network concept. These ideas have developed symbiotically with the rise of networks in the technological sense too, most crucially the global information-computation network, the internet. Computer hackers and political activists Richard Stallman and Eric S. Raymond along with lawyer Laurence Lessig are the people most strongly associated with free/libre and open-source philosophies and commons regimes, respectively.\footnote{\cite{Stallman:2002xy}, \cite{raymond:1999catb} \& \cite{Lessig:2004pi}} These individuals are recognised as pioneers in their areas and their work is widely cited. Stallman's work is focused on articulating and developing the free/libre philosophy and curating several `constitutional' documents. For Raymond, amateur anthropologist and `accidental revolutionary', it is the socio-economic manifestation of free/libre philosophy as open-source in a networked society that is of importance. In his book he explores what he perceives as a shift from organisational hierarchy, which he describes as a Cathedral, to the contrasting horizontal networks of a Bazaar. Lessig's work focuses on the problematic relationship between culture and property in advanced capitalist societies, concentrating on commons regimes.

\paragraph{}A second aspect of my dissertation is concerned with movement and movements, again considering social and technical aspects. I am interested in social movements, for example the socio-political organisations considered by New Social Movement (NSM)\footnote{Two excellent surveys of New Social Movement discourse are \cite{daltonetal1990} and \cite{Porta:2006kx}} theorists, and in technical movement, considering the flows of signs and symbols in transmission. Both movement and movements are considered through the rubric of global networked society. The groups and communities around free/libre and open-source philosophies and commons production regimes form what has been described by many as a New Social Movement. I am interested in the movement of concepts and practices from free/libre and open-source groups to other social movements and political organisations in our networked society.

\paragraph{}My approach in this thesis is to pay attention to both the interconnectedness of nodes in networks, be they biological or techno-sociological, and the movement of signs and symbols in those same networks. I have taken a hyper-networked attitude in my thesis. There is nothing but networks or, to use Latour's words, ``there is no \textit{aether} in which the networks should be immersed.''\footnote{\cite{latour:1998ant1}} In this sense the network concept is both reductive and relativist. Yet, when considered in parallel with the political philosophy of free/libre, open source, and commons ideals, the network concept takes us towards an ontology that is relational and not reductive. I acknowledge the atomic, i.e. indivisible, nature of each individual being. Yet, my ontology of this being also recognises the multi-dimensional inter-connections, that is to say the networks that each individual is a node in.\footnote{Latour describes the network concept as  ``a change of metaphors to describe essences: instead of surfaces one gets filaments (or rhizomes in Deleuze's parlance).'' \cite{latour:1998ant1}, citing Gilles Deleuze et F\'{e}lix Guattari, \textit{Mille plateaux. Capitalisme et schizophr\'{e}nie}, Minuit, Paris, (1980).}

\paragraph{}My dissertation has four parts. Firstly, I pose the question `Who Governs in a Networked Society?', considering the effect of global dynamics on modern conceptions of subject, governance and sovereignty. The post-modern history of our contemporary geospatial and political present presented by Hardt and Negri in \textit{Empire}\footnote{\cite{Hardt:2001jl}} has been a critical influence on my thoughts on networks and movement in this regard.

\paragraph{}Freedom and openness are two central concepts in hackerism and the contemporary social movements it has influenced. In the second section of this paper I explore in depth the political philosophy of free/libre and open source ideas that have arisen from hacker culture. In examining the epistemology of these emerging redefinitions of freedom and openness I demonstrate the challenges and opportunities they present to the corresponding dominant liberal notions of the same terms.

\paragraph{}The third section of this paper engages with the question of production and reproduction under the conditions of `communicative capitalism'\footnote{After \cite{dean:2005cc}, \textit{The Networked Empire: Communicative Capitalism and the Hope for Politics}} and the `positive possibility'\footnote{After \cite{Hardie:2005px}, \textit{Change of the Century: Free Software and the Positive Possibility}} presented by free/libre and open-source conceptions of the same. 

\paragraph{}The ``contemporary proliferation of political spaces and the multiplicity of democratic demands''\footnote{\cite[p.17]{Mouffe:2000fk}} needs a new politics, a politics that works at the global and the local level, the universal and the particular. New relationships of interconnectedness between people, communities, identities, artifacts, locations and events need to be reflected in the political culture of our institutions. What are the dynamics of these relationships? How are these new political movements creating spaces in our networked society? Furthermore, with ``advocates of freedom in the new digital society [\ldots] decried as pirates, anarchists, communists''\footnote{\cite{moglen:2003dcm}} by the those who dominated the imperial/industrial economy - those who have the most to loose - the stakes are clearly high. We are witnessing, as Moglen announces in his work \textit{The dotCommunist Manifesto}, ``the arrival of a new social structure, born of the transformation of bourgeois industrial society by the digital technology of its own invention.''\footnote{\textit{Ibid}.} This networked remixing and retransmission of free/libre philosophy, directed back at the forces of capital in liberal democracies that (re)produced and conditioned its emergence presents, perhaps, the challenges and the new opportunites for radical political communities and political economy. The effects of free/libre, open source and commons in transmission will conclude my dissertation.
