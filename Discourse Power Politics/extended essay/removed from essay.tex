\paragraph{}In \textit{Empire} Hardt \& Negri argue, following Marx, that capital, like the multitude, is immanent.\footnote{\cite[p. 326]{Hardt:2001jl}} Both forces are corrosive to modernities trancendent sovereignty. Democracy, the rule of the people, erroded the absolute power of modern sovereignty `from below', with capital disolving the territory of modern sovereignty `from above'. The emphasis placed on the deterritorializating power of capital by Hardt \& Negri is echoed by many commentators. ``The passage to Empire is the informatisation/modernisation of the economy that leads to the dominance of the world-market over the nation-state as the latter has little ability to contain the amorphous capacity of the dynamics of capital. It is this intensification that is indiciative of immance.''\footnote{\cite[p. 632]{alhluwalia:2004ei}}

\paragraph{}Browning's critique of Empire notes that for Marx, human beings ``are constituted by their social relations and capital capitalizes on this sociality by extending networks of relations across the globe.''\footnote{\cite[p. 196]{browning:2005gi}} Similarly, much of contemporary globalisation theory is formed from the linking of political economy to a wider web of social relations ``to provide a comprehensive explanation of the present via wide-ranging synchronic and diachronic analyses'' demonstrating an affinity with the grand narratives of Hegel and Marx.

\paragraph{}In Multitude Hardt and Nergi outline four logical categorical stances on globalisation, distinguishing those who consider that ``the present form of globalisation'' extends the powers and possibility of democracy across the world from those who don't and between those broadly from the `left' or the `right' of the political spectrum: social democratic, liberal cosmopolitan, neo-liberal and traditional values conservative.\footnote{\cite[p. 232-236]{Hardt:2005zt}}

\paragraph{}Yet while communicative capitalism transforms production and deploys technologies as a means of control, the same technologies can apparently be mobilised by the multitude in resistance. Can this paradox be resolved? This question is dealt with by Dean\footnote{\cite{dean:2005cc}}

\paragraph{}The multitude of the flesh is the ontological figure which points to the real question, unanswered in \textit{Empire}, as to the ``constituent political tendency within and beyond the spontaneity of the multitude's movements.''\footnote{\cite[p. 398]{Hardt:2001jl}} The multitude, politically constituted,


The `realist' paradigm of the territorial sovereignty of nation-state has been superseded with supranationality and a new form of sovereignty called Empire.

The nation-state is challenged by a new form of transcendental political sovereignty; Empire.

The notion of sovereign, delineated nation-states that dominated the imperial age is displaced by Empire. 

 and on the other hand of social and cultural relations

to yet not by way of the Cartesian grid structures of modernity but with rhizomatic and ad-hoc relationships, of flows between places, of movement.



I will consider the basic tenets of globalisation theories and consider to what extent globalisation is an ideology rather than an ideological perspective, deploying the critical work of Browning.\footnote{\cite{browning:2005gi}} Drawing on the work of Tiziana Terranova\footnote{\cite{Terranova:2004ly}, Chapter 2, \textit{Network Dynamics}} and Arjun Appadurai\footnote{\cite{Appadurai:1996lp}, Chapter 2 \textit{Disjuncture and Difference in the Global Cultural Economy} \& Chapter 3 \textit{Global Ethnoscapes:Notes and Queries for a Transnational Anthropology}} I will justify the hyperglobalism in \textit{Empire}.

However the hyperglobalist stance can be criticised on a number of points. Have Hardt and Negri perpetuated any of  the \textit{Follies of Globalisation Theory} discussed by Rosenberg?\footnote{\cite{Rosenberg:2000fg}} Do the authors displace, dispute or disregard historical and materialist Marxist accounts, such as those from Meiksins Wood?\footnote{\cite{Wood:2002fu} \& \cite{wood:2003de}}






\paragraph{}Thirdly, I will examine the figure of the multitude, Hardt and Negri's agent class with ``\textit{the will to be against Empire}.''\footnote{\cite[p. 210, emphasis in original]{Hardt:2001jl}} Empire is framed by the emancipatory struggle of the multitude who is, in turn, (re)configured by Empire; each in a relationship of power and resistance with the other. What or who is this figure of multitude and how does this figure displace the Marxist subject of liberation, the proletariat, if at all? This question is one of several causes of fierce critical assaults on Hardt and Negri's contention in \textit{Empire} and I will broadly review the debate. Hardt and Negri's second book, \textit{Multitude}, builds the theoretical framework of multitude advanced in \textit{Empire} and I will draw on this publication as well as two interviews the authors have given.\footnote{\cite{Hardt:2005zt}. \cite{brown:2005cs} and \cite{dumm:2005te}.}

\paragraph{}Hardt and Negri's hope for emancipation is found in the immanent power of the multitude. So what does resistance, or any sort of politics for that matter, look like in the age of empire? Politics, social struggle, resistance - each require objectives as a basis for action. However, against the totality of empire, as described by the authors, these objectives are made parochial, any strategy futile. Is \textit{Empire} antipolitcal as Laclau charges?\footnote{\cite{laclau:2005ss}} This final though will conclude my essay, weaving together the conclusions from my thematic examination of \textit{Empire}.