\documentclass[12pt,a4paper,titlepage]{article}
\usepackage{geometry}
\usepackage{helvet}
\usepackage{harvard}
\usepackage{setspace}
\renewcommand{\familydefault}{\sfdefault}
\geometry{a4paper}
\title{Do you agree with Hardt and Negri's contention that we now live in the age of Empire? Justify your response}
\author{Course Code PO53018A, Student No. 22164733}
\date{\today}

\begin{document}
\harvardparenthesis{none}
\citationstyle{dcu}
\bibliographystyle{agsm}
\maketitle
\doublespacing
\begin{quote}
``There is a spectre haunting Empire, the spectre of multitude\ldots ''\footnote{With apologies to \cite[p.2]{Marx:2004yu}}
\end{quote}
\paragraph{}My reading of Hardt and Negri's \textit{Empire}\footnote{\cite{Hardt:2001jl}} has taken me on a personal political journey. The text has provoked a critical examination of my own politics, bringing about a new consciousness of subject and sovereignty, of agency in a hyperglobalised world and of resistance to global capital. To a greater or lesser extent I do agree with Hardt and Negri's contention that we now live in the age of empire.

\paragraph{}Michael Hardt and Antonio Negri's post-modern narrative erects a theoretical scaffold relevant to contemporary global neoliberal and international capitalism. It attempt to break from the teleology of historical and materialist Marxisms. Empire presents a `grand narrative' of the history of the present so it is not without good cause that their book as been hailed as a post-modern \textit{Das Capital}. \textit{Empire} is also a manifesto, expressing the authors' hope of emancipation for their subject of liberation, the multitude. Three broad themes of the book stand out, that of global informational networked capitalism, of subject, sovereignty and supranationality and of a class politics of resistance in post-modernity.

\paragraph{}In order to defend my position of qualified agreement with Hardt and Negri I will explore the three broad themes outlined above. Firstly, I will consider the hyperglobalist position of the authors. Globality is central to Empire; the localism of nation-states and imperialism is superseded with supranationality in Empire. I will consider the basic tenets of globalisation theories and consider to what extent globalisation is an ideology rather than an ideological perspective, deploying the critical work of Browning.\footnote{\cite{browning:2005gi}} Drawing on the work of Tiziana Terranova\footnote{\cite{Terranova:2004ly}, Chapter 2, \textit{Network Dynamics}} and Arjun Appadurai\footnote{\cite{Appadurai:1996lp}, Chapter 2 \textit{Disjuncture and Difference in the Global Cultural Economy} \& Chapter 3 \textit{Global Ethnoscapes:Notes and Queries for a Transnational Anthropology}} I will justify the hyperglobalism in \textit{Empire}. The age of Empire is based on a spatio-temporal reformulation of contemporary global society, of total networked interdependence yet not by way of the Cartesian grid structures of modernity but with rhizomatic and ad-hoc relationships, of flows between places, of movement. However the hyperglobalist stance can be criticised on a number of points. Have Hardt and Negri perpetuated any of  the \textit{Follies of Globalisation Theory} discussed by Rosenberg?\footnote{\cite{Rosenberg:2000fg}} Do the authors displace, dispute or disregard historical and materialist Marxist accounts, such as those from Meiksins Wood?\footnote{\cite{Wood:2002fu} \& \cite{wood:2003de}}

\paragraph{}With the description of the contemporary character of capitalism as informational, leading to immaterial production\footnote{\cite[p. 289-300]{Hardt:2005zt}} I am in agreement with authors. Communicative capitalism\footnote{\cite{dean:2005cc}} and its perceived effects, the deterritorialisation, decentralisation and destablisation of imperial power leading to Empire, is a key part of Hardt and Negri's argument. The great role of telecommunications and information technologies in the transition from material to immaterial production, from industrial production to knowledge production, can not be downplayed. However this informationalist political economy is based on relatively recent developments in computational power and tends towards a techno-determinism.\footnote{\cite{panitch:2003de}} And while communicative capitalism transforms production and deploys technologies as a means of control, the same technologies can be mobilised by the multitude in resistance. Can this paradox be resolved?

\paragraph{}Secondly, I will explore the post-modern reformulations of modern political categories presented by Hardt and Negri in \textit{Empire}. Here I will describe the post-modern subjectivity that both creates and is shaped by empire. There are several questions posed by the post-modern subject and idea of supranational sovereignty in \textit{Empire}. How are some of the core ideas of modernity, those of state, subject and sovereignty transformed in the age of empire? How is the post-modern subject understood and how much post-modern deconstruction do I need to subscribe to in order for this definition to have any purchase?\footnote{I will draw on essays in \cite[Part 1]{finlayson:2002ps}} How is communicative capitalism active in the shaping of subjectivities? How has the deterritorialisation and decentralisation of communicative capitalism changed the state? What, if anything, remains of sovereignty in the age of empire?

\paragraph{}Thirdly, I will examine the figure of the multitude, Hardt and Negri's agent class with ``\textit{the will to be against Empire}.''\footnote{\cite[p. 210, emphasis in original]{Hardt:2001jl}} Empire is framed by the emancipatory struggle of the multitude who is, in turn, (re)configured by Empire; each in a relationship of power and resistance with the other. What or who is this figure of multitude and how does this figure displace the Marxist subject of liberation, the proletariat, if at all? This question is one of several causes of fierce critical assaults on Hardt and Negri's contention in \textit{Empire} and I will broadly review the debate. Hardt and Negri's second book, \textit{Multitude}, builds the theoretical framework of multitude advanced in \textit{Empire} and I will draw on this publication as well as two interviews the authors have given.\footnote{\cite{Hardt:2005zt}. \cite{brown:2005cs} and \cite{dumm:2005te}.}

\paragraph{}Hardt and Negri's hope for emancipation is found in the immanent power of the multitude. So what does resistance, or any sort of politics for that matter, look like in the age of empire? Politics, social struggle, resistance - each require objectives as a basis for action. However, against the totality of empire, as described by the authors, these objectives are made parochial, any strategy futile. Is \textit{Empire} antipolitcal as Laclau charges?\footnote{\cite{laclau:2005ss}} This final though will conclude my essay, weaving together the conclusions from my thematic examination of \textit{Empire}.

\newpage
\singlespacing
\bibliography{/Users/robdyke/Documents/GoldsmithsCourses/bibliography/globalbib}
\medskip
\paragraph{}Words : 835
\end{document}