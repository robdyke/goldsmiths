Does Hobbes give the sovereign too much power? 
'Perhaps the most enduring criticism of Hobbes's political philosophy is that it provides for 
an absolute sovereign that poses a great threat to individual freedom.'1 
The question highlights a significant topic of discussion in the considerable body of 
Hobbes scholarship: power. Does Hobbes give the sovereign too much power? How is his 
absolutism to be understood? What are the constraints and freedoms for both subject and 
sovereign in Hobbes's commonwealth? These questions continue to be debated because 
Hobbes has been understood as an authoritarian because of the absolutism of his political 
philosophy in Leviathan and his other writings. 
Hobbes is relevant today because he defined the need for a sovereign to govern a 
commonwealth, because of his theories on the obligations and duties of subjects or 
citizens of that commonwealth and his apparent endorsement of the democratic 
institutionalisation of  sovereignty within the commonwealth.  His assessment  and 
recommendations have cast  a long shadow over western political philosophy. 
My argument is that Hobbes gives excessive power to the sovereign. In demonstrating my 
argument I will show the great threat posed by this absolute governor. To show this I will 
outline Hobbes's thought experiment in Leviathan2 into the natural state of humanity and 
his account of the need of a sovereign. The perpetuation of the relationship of sovereign 
and subject in a social contract will be briefly examined. In doing so I will highlight the 
arguments of those who seek to rescue Hobbes from the charge of authoritarianism by 
constructing a liberalised, almost sanitised, account of his position. I will then examine 
aspects of the power available to the sovereign according to Hobbes in order to support 
my argument that Hobbes's Leviathan is given too much power.  Throughout this essay I 
will aim to complement the argument of Tarlton3. 
Hobbes wrote Leviathan in order to define the �matter, forme and power of a common- 
wealth�4. To do so he dissolves socio-political constraints on the individual in a thought 
experiment. Hobbes was a humanist and atheist so he conducts a rational thought 
experiment rather than constructing a political theory based on religious appeals. Hobbes's 
1 Van Mill, (2002), p 21. Van Mill wrote to rescue Hobbes by defending civil liberties in Hobbes's Commonwealth. 
2 Throughout this essay I will reference pages in Leviathan to the Penguin Classics Hobbes, T. (1985) edition of the 
book in simple parenthesis. 
3 Tarlton (2001) provides a through examination of what he terms 'The Liberalization of Leviathan' in Part 1 of 'The 
Despotical Doctrine of Hobbes' 
4 The subtitle of Leviathan as shown on the title plate of Hobbes, T. (1985)
psychology of the human condition, the 'matter' of a commonwealth5, forms the first part of 
his book Leviathan. In summary, Hobbes is a pessimist in his view of humanity. While he 
says that humans have a desire for peace6, Hobbes believes that we have an innate 
capacity for violence and murder and a paramount desire for self-preservation7. His 
sociology is one of individualism, with humans having �no pleasure, (but on the contrary a 
greate deale of griefe) in keeping company� (p185) with each other. 
Hobbes's purposes in writing Leviathan was to show the need for 'government', so his 
thought experiment removes external restraint on an individual human being in order 
ascertain if there is a need for what he takes away. Without government, humans live, 
according to Hobbes, in a 'state of nature', where all live at complete liberty. For Hobbes, 
complete liberty, his 'state of nature', is rationally undesirable for human beings. This is 
because without constraints humans live with a perpetual fear of experiencing violence at 
the hands of another human in competition for the resources needed to stay alive. 
Hobbes's thought experiment concludes that human beings will rationally choose to divest 
themselves of the complete liberty to murder and be murdered that they have in the 'state 
of nature' to a sovereign which has the power to provide security and protection for them: 
�For by this Authoritie, given him by every particular man in the 
Common-Wealth, he hath the use of so much Power and Strength 
conferred on him, that by terror thereof, he is inabled to forme the wills 
of them all, to Peace at home, and mutuall ayd against their enemies 
abroad.� (p.227-8) 
A communities consent in partially giving up their liberty to the sovereign binds them and 
the sovereign in a social contract. Both consent and obligation in the contract can be 
unspoken. An explicit declaration by a group of people in a state of nature to institute a 
sovereign need not have ever occurred � Hobbes's Leviathan could have evolved from 
convention into mass-practice, arriving at the sovereign ruler of a territory he describes. 
That the contractual obligation is assumed and not codified in a constitution is particularly 
concerning, given the power the absolute ruler has at his/her disposal. 
Now that I have considered the instituting of a sovereign in Hobbes's commonwealth, I will 
outline the power available to this office. In doing this I will demonstrate that Hobbes's 
5 I have understood the �matter, forme and power� Hobbes speaks of as psychology, sociology and political philosophy 
respectively. 
6 �That every man, ought to endeavour Peace� is one of Hobbes's Natural Laws (p. 190) 
7 �The Second, the summe of the Right of Nature; which is By all means we can, to defend our selves.� (p. 190)
sovereign is given too much power and that there are no practical constraints on it, leaving 
his Leviathan prone to abuse of power leading to tyranny. 
The sovereign described in Leviathan is the sole and absolute authority in the area he/she 
governs8. The sovereign is so defined in order that there be a final arbiter, to keep the 
peace, and provide the security Hobbes deems necessary for humanity to flourish, 
allowing for  trade and agriculture, travel and education, arts, letters, sciences9. Talton 
observes that many authors 'see' the absolutism in Leviathan but explain it away as a 
technical  necessity, 'commonplace and unproblematic': �'Absolute'  became only an 
inoffensive and indistinct adjective for modifying 'sovereignty', 'power' and 'authority'�10. I 
find Hobbes's doctrine as problematic as Talton, and will now go on to show how this 
absolutism gives the sovereign too much power. 
The sovereign, as arbiter of justice, is entitled to promulgate any legislation11. Furthermore, 
the sovereign is placed by Hobbes outside of the framework within which final justice is 
obtained12. This is the principal example of how the sovereign is given too much power: 
Hobbes's sovereign is beyond the law. While Hobbes attempts to provide some 
constraints on a sovereign beyond the reach of law; such as the requirement for 
promulgated legislation to be 'just'13, and Hobbes's expectation that the sovereign will act 
for  the common good14; the sovereign is primarily guided by Hobbes to preserve his 
power and authority. This provides the potential for the abuse of power and a pathway to 
tyranny. 
It is possible to counter this claim with Hobbes's insistence that natural, i.e. pre-political, 
law continues into civil law in a governed society. Yet I think that this 'constraint' is flimsy: 
the only relevant natural law that remains in Hobbes's civil society is that of self- 
preservation and this only comes into play when the sovereign attempts to punish a 
subject for disobedience. Furthermore, the continuation of natural law into civil law 
provides an opportunity for tyranny to arise: the natural law in the state of nature is violent 
and brutal. 
Some authors do not see the positioning of the sovereign outside of the rule of law as 
8 'Soveraign Power ought in all Common-wealths to be absolute' (p. 260) 
9 The in-commodities of the �state of warrre� prevent humanity from reaching its full potential (p. 186). 
10Talton, p. 599-600 
11'The Soveraign is Legislator' (p. 312) 
12'And not subject to Civill Law' (p. 313) 
13'By Instruction & Lawes' (p. 376) and 'Good Lawes what' (p. 388) 
14'The Procuration of the Good of the People' (p. 376)
problematic. Indeed it can be argued that the sovereign must be outside the rule of law in 
order that law can be promulgated and final arbitration achieved. But by not challenging 
this core principle of Hobbes's commonwealth, or by accepting this idea and by some 
'sleight of hand' sanitising the clear and present danger of a sovereign beyond direct 
accountability to his/her subjects or by highlighting weak 'constraints' on the sovereign, 
these academics are contributing to what Talton calls 'the liberalisation of Leviathan'. 
This crucial placing of the sovereign beyond the law is related to one of the duties of the 
sovereign: education. Hobbes places a duty on the sovereign to educate his/her subjects. 
Hobbes counsels sovereigns to promulgate not  only  just legislation,  but  also 
understandable legislation. This duty is derived from Hobbes's conclusion that an 
educated subject will understand and uphold the social order within which they live and 
enjoy their existence out of the 'state of nature'. An educated subject will also understand 
the reason for the political and judicial framework and will accept justice from it.15 
The provision of education goes hand in hand with the right of censorship Hobbes gives 
his sovereign. In Hobbes's commonwealth the sovereign is entitled to control the doctrines 
in public discourse: �it is annexed to the Soveraignth, to be Judge of what Opinions and 
Doctorines are averse, and what conducing to Peace; and...who shall examine the 
Doctorines of all bookes before the be published.�16  This right can be utilised in a number 
of ways. It can be used to silence dissent in the commonwealth, and, more positively, it 
can be used to protect subjects of the commonwealth from persecution on the basis of 
their sex, race, gender, etc. Control of 'hate-speak' has been a method by which western 
liberal democracies have pursued an anti-discrimination agenda, most recently with the 
Racial and Religious Hatred Act 200617 passed by the sovereign government of the United 
Kingdom. 
Van Mill argues compellingly that the 'Boundaries of Freedom' in Hobbes's commonwealth 
are wide and secure. He agrees with Hobbes that education is the duty of the sovereign 
and highlights the 'virtues of character that the state has to inculcate'. He argues that 
Hobbes's censorship is a light touch and points out that the banning of books, even those 
likely to challenge the rule of monarchy, is not advocated. In place of prohibition, Hobbes 
would have us educated by the publishing of 'masterful corrective' works.18 
15(p. 378-9) 
16(p. 233) 
17The full bill can be found at http://www.opsi.gov.uk/acts/acts2006/20060001.htm and for commentary on  the impact 
of this bill on freedom of speech see http://politics.guardian.co.uk/homeaffairs/story/0,,1698850,00.html 
18Van Mill, D. p23-24
I make the link between these two duties, education and control of doctrine in order to 
draw attention to the potential for tyranny by an absolute sovereign. In George Orwell's 
1984 these two duties are shown taken to there most extreme conclusion. In the case of 
the first, the eduction of children is used as means of control. Children are educated by the 
sovereign, 'Big Brother', to spy on and accuse their own parents. In the case of the 
second, Newspeak, the official language of Oceania - devised to meet the ideological 
needs of Ingsoc, or English Socialism - is a means of controlling the doctrines in Big 
Brother's commonwealth. The publishing of 'masterful corrective' works falls to the Ministry 
of Truth. 
�When Newspeak had been adopted once and for all and Oldspeak 
forgotten, a heretical thought � that is, a thought diverging from the 
principles of Ingsoc � should be literally unthinkable, at least so far as 
thought is dependent on words.� 19 
Orwell gives an example of absolute power wielded by a tyrannical government in his 
satire. My choice of 1984 as an example of the excesses a Leviathan is able to impose on 
his subjects may seem a little unfair. After all Hobbes conceded that his preference for 
monarchy with absolute government in the hands of a single passion-led human was open 
to this office being abused: �And though of so unlimited a Power, men may fancy many 
evill consequences� Hobbes writes. �...yet the consequences of the want of it (the state of 
nature) are much worse� (p.260) he continues, providing his own defence to this charge. 
In the introduction to Leviathan Hobbes appears only too aware of the challenging aspects 
of his discourse: �I know not how the world will receive it, nor how it may reflect on those 
that shall seem to favour it. For in a way beset with those that contend on one side for too 
great Liberty, and on the other side for too much Authority, 'tis hard to pass between the 
points of both unwounded.�20 Let us see how he did. As discussed, there are those who 
see in Leviathan and Hobbes's other writings an absolute sovereign consensually 
instituted with only sufficient power as to maintain peace and security through the 
guarantee of justice. For those authors, attributing a liberal 'balance of power' notion to 
Hobbes is sufficient. The relationship between sovereign and subject is harmonious and in 
balance because the former desires to maintain his/herposition and governs the latter, who 
themselves do not desire the alternative 'state of nature', equitably and justly. For Van Mill 
19Orwell, G. (2000) p.917 
20(p.75)
the point is that �political power is necessary but because of this it is also necessarily 
dangerous. Hobbes thought that the ill-use of such power will usually end in a disruption of 
the peace and hence is bad for all involved.�21 
For others, such as Talton, who's view of the power available to Hobbes's sovereign is that 
it leads directly to tyranny and arbitrary despotism22, Hobbes should not be considered a 
'proto-liberal'. The power given to the sovereign by Hobbes is too great and, as I have 
shown with the graphic example from Orwell, absolute power creates the fearful 'state of 
nature' that Hobbes was trying to rescue us from. 
Word count 2351. 
21Van Mill, p.36 
22Talton, C.D., p. 587
Bibliography 
Hobbes, T., (1968), Leviathan, London, Penguin Books 
Orwell, G., (2000), The Complete Novels, London, Penguin Books 
Talton, C. D., (2001), The Despotical Doctrine of Hobbes, Part 1: The Liberalisation of 
Leviathan, in History of Political Thought, Vol XXII. No 4, Winter 2001 
Talton, C. D., (2002), The Despotical Doctrine of Hobbes, Part 2: Aspects of the Textual 
Substructure of Tyranny in Leviathan, in History of Political Thought, Vol XXIII. No 1, 
Spring 2002 
Van Mill, D., (2002), Civil Liberty in Hobbes's Commonwealth, in Australian Journal of 
Political Science, Vol 37, No 1, pp 21-38