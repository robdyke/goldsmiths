\documentclass[13pt]{article}
\usepackage{geometry}
\usepackage{harvard}
\usepackage{helvet}
\usepackage{setspace}
\renewcommand{\familydefault}{\sfdefault}
\geometry{a4paper}
\title{Compare and Contrast Locke's and Marx's approach to property}
\author{Reg No 22164733}
\date{\today}

\begin{document}
\begin{onehalfspace}
\bibliographystyle{kluwer}
\maketitle


\paragraph{}In this essay I will show the different relationships between the approaches to property of the philosophers Locke and Marx. Initially I will place each thinker within their historical context and elaborate the conception of history that each has. Then I will consider how each conceives of property before looking deeper at these conceptions and the links between themes in their writings. My argument will show Marx as the truth of Locke: that is that through economic history the trajectory of private property is towards gross inequality. 

\paragraph{}Locke (1632-1704) wrote to define and defend the liberty of the individual in his age, and his writings were influential in the development of liberal political philosophy. The liberty of the individual to ensure his or her survival was protected from absolute and arbitrary power by, Locke argued, a minimal state, one which protected private property rights. Locke defined a historical theory of acquisition and the foundations of later liberal theories of justice are based on on this history. Locke's writing were revolutionary in his age - here are political tracts advocating a specific social, economic and political order. A Lockean liberal perspective was later enshrined in US constitution. Locke defined the utility of money and its relationship to property and in doing so, advocated a laissez-faire approach to exchange: the capitalist economic system of free markets. 

\paragraph{}Marx's (1818-1883) conceptions of property are the antithesis of Locke's and were influential in development of the political philosophies of anarchism, communism and socialism. To be influential on both libertarian (anarchism) and communitarian (communism and socialism) approaches to the distribution of the natural resources for humanity's survival shows how radical Marx's interpretations of political economy and his political philosophy are. Contrary to Locke's concept of liberty, Marx shows how private property makes individuals unfree. With the benefit of historical hindsight he demonstrates the development of private property revolutions. Marx's theory of history is one of class struggle. The condition of this class struggle is capital, that is to say, `private property'. Two classes homogenize as a result of the development of the capitalist economic system; the Bourgeoisie, the capital owners, and Proletarians, the wage labourers. History therefore is the history of revolution, expressed as changes in the economic, social and political orders to the benefit of the Bourgeoisie and the detriment of the Proletariat.

\paragraph{}Each of these thinkers makes use of a theory of history in their discussions on property. Locke's theory of history is one of historical justice. Marx's materialist conception of history describes injustice and, consequently, class struggle.\footnote{See opening proclamation in\cite[Chapter I]{Marx:2004yu}} I will return to these conceptions of historical justice later in this review as they are crucial to my main argument.

\paragraph{}Locke talks about private property; the exclusive ownership of a thing. Unowned things are nature's abundance and are held in common by all humans. How does Locke account for a thing ceasing to be part of the common wealth and becoming a piece of private property? His philosophy holds that humanity as a whole, and humans as individuals, have a natural right to make an exclusive claim of ownership over things. This right is given to humans by god. This right is self-evident to Locke and justified with two claims.\footnote{See \possessivecite[Chapter V.]{Locke:1988lr} \emph{Of Property}} First, it is the natural proclivity for humans to produce, cruically for survival. Second, without a right of private property there would be no protection of our activities, of our labour invested in nature to preserve our lives. Humanity is placed at the pinnacle of the natural order of life on earth by Locke. He founds an individualistic approach to property; property rights are an expression of our right to self-preservation.

\paragraph{}Whilst humans are entitled under these natural property rights to appropriate from nature Locke places limits on these entitlements. He observes an abundance in natures provision of commonly held raw materials. He recognises also that each individual equally posses two things; firstly a right of property, secondly a right in natures abundance for self-preservation. His limits are stated as two provisos on entitlement to property, viewed as self-evident in nature and thus natural law. The act appropriation, taking something out of the commons, must ensure that 'as much and as good' remains in nature for others. This proviso is stated as self-evident by Locke: over acquisition harms others and therefore goes against the laws of nature. The second proviso states that whatever is appropriated from nature must not be allowed to spoil.\footnote{See \cite[pp. 288 \& 290]{Locke:1988lr}} Later thinkers, such as Nozick have elaborated on Locke's theory of property based on the appropriation from nature by the mixing of labour. Nozick questions is not leaving enough and as good left ensuring that "the situation of others is not worsened"\footnote{\cite[pp. 175]{Nozick:1974lr}} the same as the principle as non-wastage?

\paragraph{}The crucial aspect of Locke's property theory is the legitimacy of the acquisition. If there is justice in the acquisition and in any subsequent transfer then the property claim, the ownership of the object by an individual, is just. Justice in acquisition is found when Locke's provisos - i.e natural law - have not been broken. 

\paragraph{}Locke as a writer of his age, places humans above nature, viewing them as masters and nature as subordinate to their will. Humanity cultivates the earth, mixing labour with nature. Mixing our labour with nature increases its value and its utility to our survival. This value is to be protected. Humanity has a natural entitlement to land, a natural entitlement to fashion tools and a natural proclivity for development and progress. For Locke then, humanity's natural mode of appreciation of the natural world is to be found in the valuing of exclusive ownership of things. The mixing of ones labour with unowned raw materials is the act of appropriation that makes a private property claim: ``Thus \emph{Labour}, in the Beginning, \emph{gave} a \emph{Right of Property}''.\footnote{Emphasis in original. \cite[pp. 299]{Locke:1988lr}} Private property for Locke then is arrived at though mixing human labour with nature. This bond between labour and labourer is central to both Locke's and Marx's approach to property.

\paragraph{}For Marx, self-conscious self-determination is the natural liberty of humanity. Any system that estranges humanity for this natural state alienates humanity from nature and each person from each other. Property rights, the employment of labour and the use of money all estrange humanity from their natural state. These social constructions, these conditions of society, are therefore expressions of humanity's willingly consenting in its own enslavement. Property, labour, money and the resulting inequality of that mix are discussed in Marx's early writings.

\paragraph{}In one sense Marx's analysis of economic history can be understood as picking up the narrative where Locke finishes. Locke's writes of an initial acquisition of the title to some land, the `homesteading' of a farm through labour and later he advocates the employment of labour. A farm owner, as an example of an early Lockean landed private property owner, provides employment, giving labourers payment in money. In providing employment thus ensures `as much an as good' for others. This agrarian labour is the later urbanised labour of the industrial revolution.  Paid employment marks the beginning of new forms of human relationship: employer and employee. Marx's picks up Locke's agrarian employment, writing

\begin{quote}``Labour appears at first only as agricultural labour, but then asserts itself as labour in general.''\footnote{\cite{Marx:1844xy}}\end{quote}

\begin{quote}``Just as landed property is the first form of private property, with industry at first confronting it historically merely as a special kind of property � or, rather, as landed property's liberated slave � so this process repeats itself in the scientific analysis of the subjective essence of private property, labour.''\footnote{\cite{Marx:1844xy}}\end{quote}

\paragraph{}I have elaborated the notions of property of Locke and Marx. Now I will demonstrate comparable and contrasting aspects of their approaches as I draw the relationships between each thinker. I will consider first the relationship between property and political power then I will consider how the `money system' leads to inequality. I will complete my demonstration with a consideration of justice and injustice in distribution.

\paragraph{}For each thinker there is a strong relationship between property and political power. For Locke in individual liberty there is necessarily private property. Without individual property rights humans would be collectively enslaved to living in a world where our labours would be unprotected, leaving us in precarious position with regard to sustaining our life. The protection of property is the purpose and limit of the state. This is in a sense a negative freedom, whereby there is freedom from intervention from the state and from others for the individual. The legal system of Locke's society defends the interests of private property. But in collective liberty there is no private property; property is slavery for Marx, who is interested in a positive definition of the liberty of humanity, the freedom to self-determine being crucial to Marx. In a comparable position to Locke, Marx too sees the political order of a society (the legal system, the apparatus of the state \&tc) as defending the interests of property (and not of the interests of labour)\ldots

\begin{quote}``The executive of the modern state is but a committee for managing the common affairs of the whole bourgeoisie.''\footnote{\cite[pp. 5-6]{Marx:2004yu}}\end{quote}

\paragraph{}As Locke's right of property is universal yet not everyone can exercise this right of property. For example, in Marx's analysis a worker does not own what they produce. Only a minority own property, that is to say `the means of production'; the Bourgeoisie. Therefore, in Marx's view, capitalist society has it in a structural inequality. The tension between private property ownership and competing labourers increases entropy in capitalist society. Private property, indeed the capitalist system itself is the engine for this development as self-interested capital accumulation drives revolutions in the modes of production, compacting labour and increasing class conflict.

\paragraph{}Locke shows why humans give their explicit consent for a money system to be involved in the transfer of property. A `money system' is a social agreement on an object as a universal means of exchange. Money is symbolic of alternate and future value - they are, in the words of Eric Raymond, `scarcity tokens'\footnote{See \emph{Homesteading the Noosphere} in \cite{Raymond:2000lr}}. 

\paragraph{}The existance and use of money is central to Marx's argument\ldots\begin{quotation}``connection between private property, avarise, and the separation of labour, capital and landed property; between exchange and competition, value and the devaluation of men, monopoly and competition, etc; the connection between this whole estrangement and the money system''\footnote{\cite[pp. 71]{Marx:1844qf}}\end{quotation}

\begin{quote}``By possessing the property of buying everything, by possessing the property of appropriating all objects, money is thus the object of eminent possession. The universality of its property is the omnipotence of its being. It is therefore regarded as omnipotent. . . . Money is the procurer between man�s need and the object, between his life and his means of life. But that which mediates my life for me, also mediates the existence of other people for me. For me it is the other person.''\footnote{\cite[pp. 71]{Marx:1844qf}}\end{quote}

\paragraph{}Money leads to inequality, which is justified as far as Locke is concerned, because people deserve to acquire wealth from the investment of their own labour. More crucially this wealth can accumulated in ``a little piece of yellow metal''\footnote{\cite[pp. 294]{Locke:1988lr}} - symbolic of life preserving goods and labour to acquire them. Locke's use of money and paid employment for labour helps him to not breach his first provision of property ownership; money doesn't spoil.His second provisos is secured too, even after the last unowned object has been appropriated, as money can be used to buy the satisfaction of material needs.

\paragraph{}I have made reference to the justice or injustice that, for each thinker, flows from property. Justice is related to legitimacy, or the lack of it, for each; legitimate property holdings for Locke, illegitimate appropriation for Marx. Are there any legitimate property holding? Can Locke's account of property rights justify the capitalism system that Marx is so critical of?

\begin{quote}
``\ldots there are in fact very few, and in some large areas of the world \emph{no}, legitimate entitlements. The property owners of the modern world are not the legitimate heirs of Lockean individuals\ldots they are the inheritors of those who, for example, stole, and used violence to streal the common lands of England from the common people, vast tracts of North America from the American Indian\ldots''\footnote{\possessivecite[pp. 251]{MacIntyre:1981lr} chapter \emph{Justice as a Virtue: Changing Conceptions} discusses the theme of justice in both Marx and Locke. From MacIntyre's example alone it is possible to understand the anarchist approach to property, as inspired by Marx: ``property is robbery''\cite{Proudhon:1840lr}.}
\end{quote}

\paragraph{}Without legitimate acquisition shown to be weak, is there justice in distribution? Locke argues that there was no original distribution to individuals, only the original distribution to the whole of humanity. Humanity's equal right to mix labour and to create property maintains this original distribution. There is no inequality due to the equality of opportunity to own private property and abundance of nature to mix labour with to create private property.

\paragraph{}Marx however demonstrates injustice in the distribution of property. The situation of others is worsened by private property. By the time of writing the Communist Manifesto, `as much and as good' did not remain for others. Nozick acknowledges that Locke's proviso to prevent consolidated accumulation of property was fracturing under the historical weight of ``the original appropriation plus all the later transfers and actions''.\footnote{\cite[pp.190]{Nozick:1974lr}} Nozick acknowledges injustice in the `end state' of distribution. Yet the status quo is the sum of all individual property decisions. It is individual property justice make for the correct distribution, after all,``Will people tolerate for long a system yielding distributions that they believe are unpatterned? No doubt people will not long accept a distribution they believe is unjust.''\footnote{\cite[pp.158]{Nozick:1974lr}}

\paragraph{}Locke makes provisions to correct injustice which are elaborated by Nozick in his account of a historical view of distribution. Intervention to redistribute private property is permitted for the retification of injustices that have taken place. This is a justice based on what has actually happened. Marx's philosophies however, have been used to justify the intervention in society to address inequality.\footnote{Marx's own account of the redistribution arises from raising of class consciousness amongst a compact of workers resulting in the overthrow of capitalism and its replacement with communism. This would be a final revolution in the mode of production; this time however it would reflect the interests of the proletariat and not the bourgeoisie.} The political ideology of socialism and the pragmatic manifestation of this in social democracies use the instrument of the state to create a specific pattern of justice in property ownership. to create a pattern. Nozick illustrates that to continually ensure a specific pattern distribution requires great and perpetual intervention in the affairs of individuals.

\paragraph{}For Nozick our natural property rights ``do not determine a social ordering but instead set the constraints within which a social choice is made, by excluding certain alternatives, fixing others, and so on''.\footnote{\cite[pp.166]{Nozick:1974lr}} The justice that is found in private property acquisition and in justice of transfer are not ``end state principles''.\footnote{\cite[pp.181]{Nozick:1974lr}} Nozick's argument here is clear: appropriate actions affect others, but do not structure the results. In conclusion, Nozick demonstrates Marx to be the truth of Locke. Nozick however also demonstrated  the threat to liberty that intervention in distribution to create a specific distribution has.






\end{onehalfspace}
\newpage
\bibliography{globalbib}
\end{document}