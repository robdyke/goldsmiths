\documentclass[12pt,a4paper,titlepage]{article}
\usepackage{geometry}
\usepackage{helvet}
\usepackage{harvard}
\usepackage{setspace}
\renewcommand{\familydefault}{\sfdefault}
\geometry{a4paper}
\title{Evaluate the contemporary relevance of Schmitt's conception of `the political'.}
\author{Student Registration No. 22164733}
\date{Autumn Term 2007-08}

\begin{document}
\harvardparenthesis{none}
\citationstyle{dcu}
\bibliographystyle{agsm}
\maketitle
\doublespacing
\paragraph{}Schmitt's stark conception of `the political' serves to remind us of two significant yet often neglected aspects of political philosophy. Firstly, Schmitt's conception of the political draws our attention to the establishment and membership of a political community; who is outside our group, why, and what the relationship to outsiders is. Secondly, and most crucially, Schmitt reminds us that the methods of politics are contestable, that `the political' is not bound within any manifestation of ideologies, institutions and mechanisms. I intend to show the significant relevance of Schmitt's conception of `the political' through my examination of these two challenges to contemporary politics.

\paragraph{}Schmitt's definition of `the political' is forceful due to its simplicity. His concept of `the political' is the antithesis of friend and enemy. Schmitt's friend/enemy dualism is a ``definition in the sense of a criterion and not as an exhaustive definition or indicative of substantial content.''\footnote{\cite[p.26]{schmitt:2007cop}} Crucially this antithesis is independent of other antithesis, such as moral, religious, rational, legal or economic antithesis. The use of the word friend is in its down to earth, literal sense. The meaning and therefore the identity of friend in this antithesis is easily understood. So while non-friends may appear immoral, heretical, irrational, illegal, or uneconomic these criterion would not be enough to distinguish these non-friends as `enemy'; indeed these are as likely to be characteristics of our `friends', in Schmitt's antithesis, as of our `enemies'. An enemy is separated from the union of friends; friends associate with one another and are dissociated from their enemies. The enemy is stranger, other, different, alien. Schmitt is not concerned here with private confrontations and competitions. Both friend and enemy are public. It is the public definition of friend and of enemy that makes this antagonism \emph{political}. Schmitt's Concept of the Political then is this antagonism, the ``most intense and extreme antagonism \ldots [becoming] that much more political the closer it approaches the \emph{most extreme point}, that of the friend-enemy grouping.''\footnote{\cite[p.29, emphasis added]{schmitt:2007cop}}

\paragraph{}The friend-enemy polemical criterion, as an antithesis independent of other distinctions, gives other political concepts meaning. By grouping with friends and identifying our enemy we can understand the political antagonism manifest between proponents of the state vis-a-vis society, between classes, between different manifestations of power as a state. Without the clarity of the friend-enemy grouping Schmitt demonstrates the ``incomprehensibility'' of these other political concepts. The friend-enemy distinction identifies ``exactly who is to be affected, combated, refuted, or negated'' in political language and in political actions.\footnote{\cite[p.31]{schmitt:2007cop}}

\paragraph{}There are, as Schmitt points out, a variety of forms and degrees of intensity that the polemical nature of this antagonism could take.\footnote{\cite[p.31 fn.12]{schmitt:2007cop}} But in the extreme, decisive case, it is the state in its entirety, ``as an organised political entity which declares the friend-enemy distinction.''\footnote{\cite[p.29-30]{schmitt:2007cop}} The political for Schmitt is not is the administration of difference but the real and ever present possibility of conflict. At the `most extreme point' the friend-enemy concept holds the possibility of violence. The enmity in this antithesis leaves ever open the real possibility of conflict for ones survival, the real possibility of killing enemy human beings. Political power, that utilised by the state in its declaration of enemy, is the ability to mobilise humans to kill other humans, in its most extreme case. Friend-enemy combat may be war with an enemy external to a political entity or a civil war with an internal enemy of a political entity. Schmitt friend-enemy criterion is not a metaphorical duality; it is a real grouping. Schmitt's combat is equally real; war and civil war will involve the real ``existential negation of the enemy.''\footnote{\cite[p.33]{schmitt:2007cop}} The concept of war, like other political language, is given meaning by the friend-enemy distinction. War ``presupposes that the \emph{political} decision has already been taken as to \emph{who the enemy is}.''\footnote{\cite[p.34, emphasis added]{schmitt:2007cop}}

\paragraph{}This leaves us with a `chicken and egg' confusion. If the friend-enemy dichotomy is the antithesis which gives other political concepts meaning, yet the final declaration of friend-enemy is held by an existing political entity, namely a state, what is the meaning of the state in Schmitt's conception of the political? Schmitt is not interested in a \emph{general} theory of the state\footnote{\cite[p.820]{Frye1996}} in the same way as liberal political philosophy is, which, beginning with a narrative of inalienable rights arrives at a theory of state, constitutionally restrained by the people to govern for the people; or conceptions of the political similar to popular or democratic states. For Schmitt, the state is where the political is enacted; the state is the sovereign political entity, the only entity that holds the power over the decision which defines the identity of friend and enemy. ``The concept of the state presupposes the concept of the political.''\footnote{\cite[p.1]{schmitt:2007cop}} The empirical existence of nation-states, parliaments, liberal economies and democratic mechanisms is of no direct concern to Schmitt's conception of the political; these concepts are functional because of the political, not conditions of the political in its self.

\paragraph{}These concepts are the also the cause for Schmitt's writing \textit{The Concept of the Political}, \textit{The Crisis of Parliamentary Democracy}\footnote{\cite{schmitt:1985pd}}, \textit{Political Theology}\footnote{\cite{schmitt:2005pt}} and other texts. He is against conceptions of the political as understood through functional units such as state and society, against the political as parliamentary government, against liberal and democratic political conceptions: Schmitt is demanding that we recognise the political first, and in its self, before the prevailing political philosophy, institutions or constituent functions of a particular political community. 

\paragraph{}As noted earlier, Schmitt's conception of `the political' reminds us of two often neglected problems in political philosophy. Firstly, Schmitt's conception of the political draws our attention to the establishing and membership of a political community through the friend-enemy antagonism. The establishing of a political community is, for Schmitt, a sort of Hobbesian escape from a `state of nature', but in reverse. In order to know an enemy and to locate the threat of our own existential negation, humans create a political community. This explanation is not based on an anthropological or ahistorical proposition of either a negative or a positive conception of human nature in the former example, or in a constitutional framework based on liberal rights in the latter. Schmitt's political community is Hegelian in that his political communities are transformed into being through the dialectical negation of what has gone before, rather than birthed, fully formed, as liberal narratives of the arrival of constitutionalism maintain. This transformational establishment of a political community illustrates the contingent and circumstantial nature of any political community, its functional appearance, and of its borders.\footnote{In the sense of a frontier between friend and enemy, rather than in the sense of borders of states. I note this as the borders of state do not necessarily correspond to the borders of political communities in a post-modern transnational world.} Schmitt's friend-enemy antithesis, his concept of the political, comes first: the definition of enemy defines the people in Hobbes's state of nature, \emph{it defines with whom we are at war}.

\paragraph{}Understood as this antagonism, the political is isolated from the institutions of government and the mechanisms of state. Has Schmitt successfully separated the political from political philosophy; from liberal conceptions of inalienable individual human rights and from democratic notions of popular sovereignty? In conceptualising the political as he did, Schmitt demonstrates that the methods of politics are contestable and that the political is not bound by ideologies, institutions and mechanisms. Here Schmitt is of significant contemporary relevance as his conception of the political, as distinct from political philosophy, ideology \&tc, is made by a direct philosophical assault on both liberalism and democracy; two concepts, often conjoined by hypenation, which describe the culture and the practices of politics in many states today. Schmitt demonstrated what he perceived as an inherent instability in liberal democratic regimes and heralded the demise of parliamentary democracy. Although the increase of liberal-democratic regimes the world over since his writing may prove him wrong in empirical fact, his exposure of the necessarily consensual and antagonistic relationship between liberalism and democracy gives us the opportunity to glimpse at what lies behind the union of the liberal-democratic facade. Strauss and, later, Frye maintain that the principle recipient of criticism in Schmitt's Concept of the Political is liberalism. Others, such as Cristi, view democracy as being the main target.

\paragraph{}Turning first to Schmitt's engagement with democracy as it relates to his conception of the political. The circumstances contingent to Schmitt's philosophical attack on democracy were that of Weimar period Germany and an apparent \textit{Crisis of Parliamentary Democracy}.\footnote{\cite{schmitt:1985pd}} This `crisis' was that brought about by popular involvement in politics through a democratically constituted parliament which weakened, in Schmitt's view, the ability of the state to act, both in normative circumstances and most crucially in times of exception. Schmitt understood that democracy was a principle of legitimacy in a political order\footnote{\cite[p.285]{cristi1993ldc}} and was not suitable to be applied to day-to-day decisions of government. Most crucially, because the friend-enemy antagonism involves the real possibility of killing people it can not be decided democratically: enemies can not be declared by lottery. This is not to say that Schmitt is anti-democratic; Cristi demonstrates that Schmitt is advocating reform of parliamentarianism and maintaining democratic procedures in order to ensure the legitimacy of public officials.\footnote{\cite[p.283-4]{cristi1993ldc}} However, in order to preserve and protect the unity of the state, the purity of the political, Schmitt sought to restrain democracy.

\paragraph{}To restrain democracy is to restrain the rule of the people over themselves. Democracy's hegemony as a mechanism of legitimacy and of decision-making in political communities is evident; there appears a want for more democracy, not less.  So what then can we learn from Schmitt's engagement with democracy? There are three nuances to Schmitt's understanding of democracy as it relates to his conception the political. Firstly the forging of the \emph{demos}, the people in any democracy, is a crucial part of the conflict that the friend-enemy antithesis defines: this ``paradigmatic constellation is a group's contention of its `sameness' and `identity' against the `otherness' of a different group.''\footnote{\cite[p.156]{Preuss:1999cs}} This illustrates that democracy, or any institutional ordering, ``rests on `the political' as its preceding condition''\footnote{\cite[p.157]{Preuss:1999cs}} and as such membership of a political community, that \emph{demos} which is governed in a \emph{democracy}, is also contingent on the friend-enemy antithesis. Second, democratic equality is always curtailed by the political. Democratic equality, the equality of humans beings with each other as humans, is a universalising conception of humanity; yet this is divided by friend-enemy antagonism. Only members of a political community can be equal, enemies can not. Third, Schmitt thought that democracy had been made functionally subservient to the liberal acceptance of social pluralism as an empirical fact. This reduced democracy to a decision-making mechanism, a \emph{rational} way out of the multiple, equal and contending demands in politics.

\paragraph{}Schmitt's concerns from Weimar period Germany articulate a precarious and contingent description of the democratic political ideal. The \emph{demos} is shown as constituted through the political which defines the borders of who is to be treated equally, and who is unequal, `enemy'. With democracy bounded by the borders of a political community, its universal equality is revealed as being truly particular and unequal. This leads, in Schmitt's analysis of his own political time, to a crisis of legitimacy, to partisan politics and the collapse of the liberal-democracy parliamentary political system. The question is therefore, what lessons can we take from Weimar period Germany for contemporary politics?

\paragraph{}Chantel Mouffe\footnote{\cite{Mouffe:1999,Mouffe:2000fk}} engages with Schmitt's contingent democracy, stabilising it, and teases out the challenges for contemporary democrats, liberals and civic republicans. Her close reading of Schmitt redeploys some of his arguments against him and in support of political ideas that he would not have supported. I will briefly outline some of her challenges to democracy derived from Schmitt, particularly community homogeneity and contingent closure.

\paragraph{}Schmitt is interested in the `sameness' of a political community. In highlighting his desire for homogeneity at the price of heterogeneity, Mouffe describes two ways in which community homogeneity is problematic. Firstly when ideas of nation and national identity are attached to the state creating the nation-state and secondly when citizens are detached from the state, creating a cosmopolitan stateless global citizen. In the first case, membership of a political community can become dependent on meeting criterion defined by identity politics stemming from realms which are non-political in Schmitt's conception. Nazi ideas of a pure Aryan nation should be sufficient a warning against politics based on a non-political criterion. In the second case an example of contemporary relevance is revealed. A post-modern conception of membership of a global political community emerges, one without spatial location. Mouffe evaluates the viability of a cosmopolitan global democracy negatively: ``We should indeed be aware that without a demos to which they belong, [these] cosmopolitan citizen pilgrims would in fact have lost their democratic rights of law-making. They would be left, at best, with their liberal rights of appealing to transnational courts to defend their individual rights when they have been violated''.\footnote{\cite{Mouffe:1999cs}}

\paragraph{}As I have shown, Schmitt's conception of democracy and democratic equality is specific and could never scale to encompass the whole of humanity. Democracy, like Schmitt's concept of the political, is understood as an antithesis: the democratic principle of equality must have the necessary correlate of inequality in order for the whole concept to have meaning. Furthermore, democracy can only be understood as a political concept because of the political, the friend-enemy antithesis. Schmitt's central concept in democracy is not `humanity' but `the people'. Consequently democracy can exist only within a political community; democracy can exist only for a people.\footnote{\cite[p.41]{Mouffe:2000fk}}  For Mouffe, this is the necessary contingent closure of democracy: the paradoxical closure of democratic equality to anyone outside of the political community, outside of the people who are friends, demonstrates the inequitable nature of democratic equality.

\paragraph{}I have already outlined that Schmitt viewed the state, the embodiment of a political community, as being under attack on two fronts; on the one hand by the popular sovereignty espoused by democrats, on the other hand by that of liberal philosophy, limiting the state by constitutionalism. I will now turn to Schmitt's engagement with liberal philosophy as it relates to his conception of the political. Schmitt's interpretation of liberal thought distinguished between its ethical and economic strands:
\begin{quote}
``\ldots liberal concepts typically move between ethics (intellectuality) and economics (trade). From this polarity they attempt to annihilate the political as a domain of conquering power and repression. The concept of private law serves as a lever and the notion of private property forms the centre of the globle, whose poles - ethics and economics - are only the contrasting emissions from this central point.''\footnote{\cite[p.71]{schmitt:1966cop}}
\end{quote}

\paragraph{}Neither strand can be understood as `political' in Schmitt's terms. Ethical liberalism prizes the liberty of the individual, economic liberalism defends the autonomy of the economic sphere. Schmitt is clear that on the question of ``whether a specific political idea can be derived from the pure and consequential concept of individualistic liberalism'', this much is to be denied.\footnote{\cite[p.71]{schmitt:2007cop}}

\paragraph{}Ethical liberal thought, in Schmitt's view, weakened the ability of the state to act both in normative circumstances and, most crucially, in times of exception. This was due to two factors: constitutionalism and pluralism. The state was weakened by constitutional restraints on the scope of government. These constitutionally defined limits are public and delineate the boundaries of normal state action. Here we see a challenge to the sovereignty of a political community. The moment of exception is by definition \emph{outside of normative circumstances}, i.e. those predefined by a constitution. This means that when faced with an exceptional circumstance the liberal state is incapable of acting to defend its self.

\paragraph{}The second aspect of Schmitt's criticism of liberalism is its philosophical weakness when dealing with the empirical fact of a plurality of interest groupings in society. The existence of different social groups and civil associations, i.e. pluralism as a fact, is not a direct challenge to Schmitt's concept of the political. These groupings are outside of the political. Indeed by Schmitt's analysis these various social and civil interest groups each possess the capacity to transform themselves in to political communities and do so the moment that they can make the sovereign demand of killing and being killed of their members as a result of the decision in the exception. What is problematic in liberalism's acceptance of `pluralism as a fact' is that it tends towards conflict negation through rational consensus, i.e. democracy. Democracy is used to escape the problematic in pluralism. Democracy is accepted by liberals as a rational decision-making mechanism, manufacturing unity where none could exist. Mouffe criticises liberal governmental rationality for Foucaultian `governmentality' on this basis.

\paragraph{}While ethical liberalism may masquerade as a political philosophy, economic liberalism is certainly not a political philosophy. That notwithstanding, economic criteria, that is to say antithesis from the economic sphere such as economic/uneconomic, can escape out of the private realm and into politics, reducing the political to mere competition and diluting government down to the technocratic administration of transactions.

\paragraph{}Leo Strauss in his \textit{Notes on the Concept of the Political}\footnote{\cite[Note 3]{Strauss:2007ncp}} confidently asserts that Schmitt's thesis is ``entirely dependent upon the polemic against liberalism''. Charles Frye agrees that The Concept of the Political and much of Schmitt's other work seeks to demolish liberalism.\footnote{\cite[esp. p.824-6]{Frye1996}} Renalto Cristi, while on the one hand acknowledging Schmitt's assault on liberalism, rejects the primacy given to this aspect of his thought by Strauss and Frye. Cristi argues that Schmitt ``came to realise the genuinely apolitical nature of liberalism. He now understood that its pluralist demands need not be conjoined with the democratic ideals. Liberalism is not a \emph{political} imperative and so its pluralist demands may be restricted to the \emph{social} sphere''.\footnote{\cite[p.296, orig. emph.]{cristi1993ldc}}

\paragraph{}Given historical fact it would be easy to dismiss the erstwile Nazi Schmitt's ideas as having no relevance to contemporary politics. He was an conservative thinker writing about the specific circumstances of Weimar period Germany with a disdain for both liberal and democratic politics. His desire was to reinvent some form of absolute sovereignty, his purpose was to rescue Germany from modernity. He could be portrayed through a surface reading of his work as having little to say that is helpful for contemporary politics. Yet Schmitt needs to be understood in terms of his place in a lineage of political thought; from Hobbes' state of nature, through Kantian absolute monarchy to Hegal's aristocratic Prince. These thinkers of the enlightenment period were each treading the pathway between the twin poles of authority and liberty, a journey that leaves no pilgrim unscathed. Their ideas were brought forward by the dialectical negation of what had gone before. Greater and wider demands for individual liberty vis-a-vis political authority caused these thinkers and others to examine the nature of the liberty of the individual and the legitimacy and sovereignty of the state. It was the demise of monarchical sovereignty, its replacement with aristocracy and the laters later negation and succession by liberal parliamentary democracy which fueled Schmitt's conservative desire for the reestablishment of authoritarian markers of certainty. With this transformation in mind we can understand Schmitt's Conception of the Political and other texts not as heralding the collapse of the `non-viable' liberal democratic regime, but as a moment in the dialectical negation of liberal democratic forms. In engaging with the themes and problems of liberty and democracy as described by Schmitt, Mouffe et al\footnote{\cite{Mouffe:1999}} are continuing this dialectic.

\paragraph{}So what to take from and what to leave with Schmitt? Firstly, Schmitt's conception of the political draws our attention to the establishment and membership of a political community. Secondly, and most crucially in my view, Schmitt reminds us that the nature of our contemporary political communities, that is states which are liberal and democratic, is contestable. That `the political' in its essence is not bound by any other concepts.

\paragraph{}Schmitt's contemporary relevance is in demonstrating that we must always remain open to group (re)definition and that this definition is a political one, not racial, social, economic, moral \&tc. There can be no political community and therefore no order without first defining outsiders. There can be no politics without this basic antagonism. Contemporary political activity continues the rhetorical performance of this distinction.\footnote{George Bush's `Axis of Evil' speech clearly defined the enemy of the American state.}

\paragraph{}There is a postmodern view of Schmitt which makes use of his criticism of liberalism and democracy. This view holds that contemporary liberal democracy lacks debate over common values, its only common value being that it values all values equally and apolitically. However the continued empirical fact of incommensurable differences between peoples which creates conflict runs contrary to the liberal democratic narrative and the presuppositions in its twin claims of a) securing rational unity and b) that there can be no politics without consensus. Schmitt shows that consensus is irrational. Consensus stifles argument and negates rational unity, the very goal of liberals throughout the Enlightenment. Schmitt is also used to defend the value of argument and by demonstrating that the political is found in antagonism he reminds us that politics is as much about what we don't want as it is an arena for getting what we do want.

\paragraph{}The threat of conflict remains although in our twenty-first century multicultural global society it is no greater than any fear the nineteenth century Schmitt may have felt. Conflicts continue to break out due to the destabilisation of the friend-enemy distinction. Yet violence does not have to be the outcome of this conflict. However there will be violence because it has never not happened this way. The acceptance of the certainty of conflict is a key lesson to take from Schmitt. One need not share his anthropological pessimism with regard to the nature of human beings in order to recognise this certainty.

\paragraph{}In his book on of the relationship between liberalism and democracy Norberto Bobbio demonstrates that the old liberal Friedrich von Hayek, like the conservative Schmitt, acknowledged the distinctiveness of liberal and democratic doctrines\footnote{\cite[p.81]{Bobbio:2005vn}} There was once utility in the conjoining of liberalism and democracy and that was in the struggles against monarchical absolutism. Yet the time for this hypenation has past. Schmitt teaches contemporary politics that liberalism and democracy are two solutions to two different problems. The former with governmental functioning, the latter with who is to govern and by what procedures. This reminds us that the nature of our contemporary political communities, the consensus on liberal democratic forms, is contestable and temporal.  Strauss's \emph{Notes\ldots} reinforce Schmitt's point about the inviability of liberal democratic regimes and the desirability of replacing them with regimes that recognise conflict rather than negate the political.\footnote{\cite[Notes 5 \& 6]{Strauss:2007ncp}}

\paragraph{}Mouffe's reconception of the friend-enemy antithesis as adversaries, that is, as friendly-enemies, builds on Schmitt's conception of the political.\footnote{\cite[esp Ch.2 \& 4]{Mouffe:2000fk}} She conceives of an agonistic plurality of adversaries, acting in a shared symbolic space. Both liberal and democratic ideas are grounded in an individualistic conception of society and this is reflected in their competition for the same symbolic space. Can the agonism in the meeting of liberalism and democracy lead us towards a viable union of ethical liberalism and democracy, on to new conceptions of liberal pluralism and of democratic equality? For Mouffe this is possible and desirable. The liberty of the individual and the self-rule of a community are of equal importance and are reflected in her agonistic conception of the political.
\newpage
\singlespacing
\bibliography{/Users/robdyke/Documents/GoldsmithsCourses/bibliography/globalbib}
\medskip
\paragraph{}Words : 3858
\end{document}
