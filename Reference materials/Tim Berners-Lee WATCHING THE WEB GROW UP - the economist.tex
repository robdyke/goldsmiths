WATCHING THE WEB GROW UP

Mar 8th 2007  

Tim Berners-Lee created the web in 1991. Now people are talking about Web 2.0--but he is more excited by other things

IN 1994, when Tim Berners-Lee left CERN, the particle physics laboratory near Geneva where he created the world wide web, to move to the Massachusetts Institute of Technology (MIT), his children were toddlers--just like the fledgling information-sharing system he had released onto the internet three years earlier. Since then the web has grown up fast, expanding from around 10,000 websites in the world at the end of 1994 to over 100m today. After this rapid growth spurt the web is now, like Sir Tim's children, in its teenage years. The painfully self-conscious "Web 2.0" movement--a label which encompasses a range of technologies such as blogs, wikis and podcasts--represents the web's adolescence. It has all the hallmarks of youthful rebellion against the conventional social order, and is making many traditional media companies tremble.

Sir Tim ought to be thrilled. After all, his original vision was for the web to be a two-way medium, in which writing information was just as simple as reading it--but as the web took off in the late 1990s, publishing tools failed to keep up with web browsers in ease of use, and it is only with the rise of blogs and wikis that the balance has been redressed. Yet Sir Tim is less excited by all this than you might expect. He regards Web 2.0 as just a fancy name for some useful, if still rather basic, web-publishing tools, and was not at all surprised by the emergence of "user-generated content"--since that was what he had intended all along. "The web was designed so every user could be a contributor," he says. "That sort of participation was the whole idea and was there from the start."

Although he is somewhat sceptical of the hype around Web 2.0, Sir Tim is excited by three other areas of the web's development: its spread to millions of new users via mobile devices, the growing interest in the technology's social and political impact and the "semantic" web, in which information is labelled so that it makes sense to machines as
well as people. "If you look at the number of internet-capable mobile phones, PDAs and so on, they are rapidly outnumbering the things we think of as computers," he says. "As the price of these devices falls, large parts of the developing world will get web access. When you have a large mass of new users, you will get many new applications, written by people with other needs."

The number of internet users reached 1 billion in 2005. But although about 70percent of the population now has access to the internet in North America, the figure is just 11percent in Asia and less than 4percent in Africa. To the jaundiced observer who remembers the disappointment of WAP, the first attempt to bring the internet to mobile phones, Sir Tim's enthusiasm for mobile-internet access may sound like deja vu. But he insists that there are crucial differences. "WAP was not based on standard internet protocols, there was no competition for browsers, and operators had a stranglehold on access," he says.

GOING MOBILE
All of this stifled innovation. The trick now is to bring the web, rather than some limited, cut-down version of it, to mobile devices. "The point of the web is that it is open, and anyone can create a new resource, instantly get feedback, and rapidly have a money stream flowing," he says. Under Sir Tim's leadership, the W3C--the web-standards body at MIT that he has headed since 1994--has launched a mobile-web initiative to adapt web standards so that information can be more easily accessed via mobile devices. Such standards will, he hopes, help to make the riches of the web available to the next billion users.

That should help to extend the benefits of the web to more of the world's population. But it will also help to spread what critics regard as the web's negative aspects. "Clearly, any technology can be used for good or for bad. That's always been the case," says Sir Tim. "But in general, more communication is a good thing." It can, he notes, build
bridges between cultures, boost commerce and accelerate scientific progress. The web can be used by criminals and racists--but it can also be used to counter them. Totalitarian regimes can filter content on the web and use it to track dissidents, but human ingenuity means that attempts to block the flow of information altogether are doomed to fail.

Sir Tim is credited with creating the web's technical underpinnings--such as HTML and HTTP, the protocols used to encode web pages and transmit them across the internet--but he has always argued, with characteristic humility, that the web is as much a social creation as a technical one. In fact, its social and technical aspects are intertwined, and understanding how the networks of people and computers that make up the web interact and reinforce each other has given rise to "web science", a nascent field that blends sociology with computer science.

To encourage this line of enquiry, Sir Tim last year helped to establish the Web Science Research Initiative (WSRI), a collaboration between MIT and the University of Southampton. "Web science looks at the web as a large system which depends on the laws of behaviour between people, like copyright law, as well as the protocols that govern how computers communicate with each other," he says. Such laws and protocols are drawn up in the hope of producing a large-scale effect, such as creating the blogosphere or facilitating scientific publishing. The point of web science is "to understand how these large-scale effects depend on the underlying laws and protocols".

An analogy can be drawn with the way that a few simple chemical interactions between atoms define how they bond together to form molecules, and that in turn specifies how proteins interact and DNA can copy itself, all of which ultimately determines the properties of immensely complex systems such as the human brain. Understanding how such "emergent phenomena" can result from simple laws has been all the rage in physics circles for over a decade, though in the web's case the basic rules are man-made, rather than being laws of nature.

Why bother studying any of this? Sir Tim is optimistic that analysis of the web will be followed by synthesis of this understanding into technical proposals that can encourage and support new social trends. Ultimately, he hopes the WSRI will produce recommendations for improvements to the web that can be fed into the technical agenda of the W3C. Understanding the interaction between technical protocols and social rules ought to make it possible to use the web more effectively in society, he thinks. 

Analogies between the web and the brain have long played a profound role in Sir Tim's thinking. He is the son of two mathematicians who worked on the team that developed the world's first commercial, stored-program computer, the Manchester University Mark 1, which was commercialised by Ferranti, a British company, in the 1950s. He remembers his father reading books on the brain, looking for ways to make computers able to identify connections between things, as the brain does. 

This line of thought has stuck with Sir Tim and is at the core of one of his most enduring passions, the semantic web. Whereas the web today provides links between documents which humans read and extract meaning from, the semantic web aims to provide computers with the means to extract useful information from data accessible on the internet, be it on web pages, in calendars or inside spreadsheets. Such data--much of it stored in databases that can be queried by humans via the web--is part of what is referred to as the "deep web", and cannot be accessed by the web-crawler programs used by most search engines.

No one knows exactly how much information is in the deep web, but estimates range from hundreds to thousands of times more than in the "surface web" that search engines currently index, which is thought to contain over 10 billion pages. If semantic-web technology can help computers access even a fraction of this hidden data, and make sense of it, it could make possible new forms of searching and would even allow software to retrieve information and make deductions from it.

A MATTER OF SEMANTICS
The semantic web has been a work in progress for over a decade--indeed, Sir Tim has said that he has been working on it since he started work on the web. A few detractors have argued that it is simply not feasible in practice to expect people to apply labels to all the information sitting on the internet so that machines can make sense of it. But Sir Tim remains optimistic. He points to solid steps forward, including XML, a language which provides a basic syntax for sharing data and has been widely adopted. The key building block is the Resource Description Framework (RDF), which provides a way to refer to different objects (such as sets of data) and the relationships between them--the underlying semantics of documents and data. Then there is the Web Ontology Language (OWL), which provides a way to characterise objects. 

"There's a thriving community working on the semantic web," says Sir Tim. The basic standards are in place, a query language called SPARQL is about to be launched, and the first semantic-web browsers are being prototyped. These are far more complex than web browsers, says Sir Tim. "You have to be able to display the data, draw graphs and so on. This is still very much at the research stage," he says.

When pressed for examples of applications of the semantic web that common mortals might appreciate, Sir Tim enthuses about "friend-of-a-friend" networks, where individuals in online communities provide data in the form of links between themselves and their colleagues and friends. The semantic web could help to visualise such
complex networks and organise them to allow a deeper understanding of the community's structure. He also sees semantic-web technology playing a role in syndicating useful data, such as weather information, in the same way that RSS feeds are now used to syndicate news items and blog postings on the web. And just as the web was first embraced by particle physicists, the semantic web may well take root in the life sciences, where it could allow separate genomic databases to be linked, searched and compared in novel ways.

These examples may not sound like a revolution in the making. But doubters would do well to remember the web's own humble origins. In 1989 Sir Tim submitted a rather impenetrable document to his superiors at CERN, entitled "Information Management: A Proposal", describing what would later become the web. "Vague but exciting" was the comment his boss, the late Mike Sendall, scribbled in the margin. The semantic web may seem equally vague today, but it could prove just as exciting. 
 

See this article with graphics and related items at http://www.economist.com/search/displaystory.cfm?story_id=E1_RSGGDJP

Go to http://www.economist.com for more global news, views and analysis from the Economist Group.

- COPYRIGHT -

This e-mail message and Economist articles linked from it are copyright (c) 2007 The Economist Newspaper Group Limited. All rights reserved. http://www.economist.com/help/copy_general.cfm

Economist.com privacy policy: http://www.economist.com/about/privacy.cfm

The Economist, Economist.com and CFO Europe are trading names of: 

The Economist Newspaper Limited 
Registered in England and Wales. No.236383
VAT no: GB 340 436 876 
Registered office: 25 St James's Street, London, SW1A 1HG 



