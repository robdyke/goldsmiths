\newpage
\singlespacing
\cleardoublepage
\addcontentsline{toc}{chapter}{Bibliography}
\bibliography{/home/doitsysadmin/Documents/GoldsmithsCourses/bibliography/globalbib}
\begin{appendices}
\section{Notes on the production of this text}
\paragraph{}My dissertation was produced using only free-software and open-source tools. The text was written, typeset and printed using the LaTeX document markup language. It is printed in 12pt Helvetica. My footnotes and bibliography are formatted using the Harvard/DCU LaTeX template.
\begin{description}
\item[Fedora GNU/Linux]{A distribution of the GNU Linux operating system, packaged by the Fedora Project.\newline\url{http://www.fedoraproject.org}}
\item[Freemind]{A free-software open-source mind mapping tool. \newline\url{http://freemind.sourceforge.net/}}
\item[LaTeX]{A document preparation system, released as free software. \newline\url{http://www.latex-project.org/}}
\item[TeX Maker]{A free-software LaTeX editor, released under GNU GPL. \newline\url{http://www.xm1math.net/texmaker/index.html}}
\item[JabRef]{An open-source bibliography reference manager, creating BibTeX files for use with LaTeX. \newline\url{http://jabref.sourceforge.net/}}
\item[BibTeX]{A tool and a file format which are used to describe and process lists of references in conjunction with LaTeX documents. \newline\url{http://www.bibtex.org/}}
\item[Subversion]{A free-software open-source distributed version control system for collaboration in a networked environment. \newline\url{http://subversion.tigris.org/}}
\end{description}
\newpage
\section{Glossary of Acronyms}
\begin{description}
\item[ANT]{Actor Network Theory. A network sociology associated with Bruno Latour. See \textit{On Actor Network Theory: A few clarifications} Pt.1 (\cite{latour:1998ant1}) \& Pt.2 (\cite{latour:1998ant2})}
\item[FSD]{Free Software Definition. ``Free software is a matter of liberty, not price. To understand the concept, you should think of free as in free speech, not as in free beer. Free software is a matter of the users' freedom to run, copy, distribute, study, change and improve the software.''  \newline\url{http://www.gnu.org/philosophy/free-sw.html}}
\item[FSF]{Free Software Foundation promotes computer user freedom and defends the rights of all free software users.  \newline\url{http://www.fsf.org/}}
\item[GNU]{The GNU Project, launched in 1984, develops free software. The name “GNU” is a recursive acronym for “GNU's Not Unix”; it is pronounced g-noo, as one syllable with no vowel sound between the g and the n.  \newline\url{http://www.gnu.org/}}
\item[GPL]{General Public License. A codified expression of the Free Software Definition, maintained by the GNU Project.  \newline\url{http://www.gnu.org/licenses/licenses.html}}
\item[ISO]{International Organization for Standardization is a global network, developing and publishing technical specifications. \newline\url{http://www.iso.org}}
\item[NGO]{Non-Government Organisation, used as short-hand for independent civil society and grassroots organisations, private voluntary organisations, self-help organisations\linebreak and other non-state actors.}
\item[NSM]{New Social Movement, used as shorthand for the great diversity of groups in co-ordinated social change networks.}
\item[WIPO]{World Intellectual Property Organisation. A specialised agency of the United Nations, ``dedicated to developing a balanced and accessible international intellectual property (IP) system, which rewards creativity, stimulates innovation and contributes to economic development while safeguarding the public interest.''  \newline\url{http://www.wipo.int}}
\item[WTO]{The World Trade Organisation is ``the only global international organisation dealing with the rules of trade between nations.''\newline\url{http://www.wto.org}}
\end{description}
\newpage
\section{The Open Source Definition}
\paragraph{}\url{http://opensource.org/docs/osd}
\paragraph{}Open source doesn't just mean access to the source code. The distribution terms of open-source software must comply with the following criteria:
\begin{description}
\item[Free Redistribution]{The license shall not restrict any party from selling or giving away the software as a component of an aggregate software distribution containing programs from several different sources. The license shall not require a royalty or other fee for such sale.}
\item[Source Code]{The program must include source code, and\linebreak must allow distribution in source code as well as compiled form. Where some form of a product is not distributed with source code, there must be a well-publicized means of obtaining the source code for no more than a reasonable reproduction cost preferably, downloading via the Internet without charge. The source code must be the preferred form in which a programmer would modify the program. Deliberately obfuscated source code is not allowed. Intermediate forms such as the output of a preprocessor or translator are not allowed.}
\item[Derived Works]{The license must allow modifications and derived works, and must allow them to be distributed under the same terms as the license of the original software.}
\item[Integrity of The Author's Source Code]{The license may restrict source code from being distributed in modified form only if the license allows the distribution of "patch files" with the source code for the purpose of modifying the program at build time. The license must explicitly permit distribution of software built from modified source code. The license may require derived works to carry a different name or version number from the original software.}
\item[No Discrimination Against Persons or Groups]{The license\linebreak must not discriminate against any person or group of persons.}
\item[No Discrimination Against Fields of Endeavor]{The license\linebreak must not restrict anyone from making use of the program in a specific field of endeavor. For example, it may not restrict the program from being used in a business, or from being used for genetic research.}
\item[Distribution of License]{The rights attached to the program\linebreak must apply to all to whom the program is redistributed without the need for execution of an additional license by those parties.}
\item[License Must Not Be Specific to a Product]{The rights\linebreak attached to the program must not depend on the program's being part of a particular software distribution. If the program is extracted from that distribution and used or distributed within the terms of the program's license, all parties to whom the program is redistributed should have the same rights as those that are granted in conjunction with the original software distribution.}
\item[License Must Not Restrict Other Software]{The license must not place restrictions on other software that is distributed along with the licensed software. For example, the license must not insist that all other programs distributed on the same medium must be open-source software.}
\item[License Must Be Technology-Neutral]{No provision of the license may be predicated on any individual technology or style of interface.}
\end{description}
\newpage
\section{Creative Commons License}
\paragraph{}Attribution-Noncommercial-Share Alike 3.0 Unported\linebreak\url{http://creativecommons.org/licenses/by-nc-sa/3.0/}
\paragraph{You are free:}
\begin{description}
\item[to Share]{-- to copy, distribute and transmit the work}
\item[to Remix]{-- to adapt the work}
\end{description}
\paragraph{Under the following conditions:}
\begin{description}
\item[Attribution.]{You must attribute the work in the manner specified by the author or licensor (but not in any way that suggests that they endorse you or your use of the work).}
\item[Noncommercial.]{You may not use this work for commercial purposes.}
\item[Share Alike.]{If you alter, transform, or build upon this work, you may distribute the resulting work only under the same or similar license to this one.}
\end{description}
\paragraph{}For any reuse or distribution, you must make clear to others the license terms of this work.\linebreak The best way to do this is with a link to this web page: \url{http://creativecommons.org/licenses/by-nc-sa/3.0/} \linebreak Any of the above conditions can be waived if you get permission from the copyright holder.\linebreak Nothing in this license impairs or restricts the author's moral rights.

\end{appendices}
\end{document}
