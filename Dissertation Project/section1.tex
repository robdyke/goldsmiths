\newpage
\chapter{Who Governs in a Networked Society?}
\paragraph{}Dahl asked `Who Governs?' and found a gradual shift from oligarchy to pluralism.\footnote{\cite{Dahl:1961fj}} Barry's \emph{Political Machines}\footnote{\cite{Barry:2001ff}} was concerned with `governing a technological society'\ldots This section, then, develops both of these lines of enquiry and can be conceived as `Who Governs in a Networked Society?' I am concerned with the present and the shift from pluralism to networked society. My focus here is to consider `interconnectedness', revealing the relationships in and between complex social systems. I contend that ever increasing complexity and ever greater plurality developing in tandem in the political, economic, and cultural spheres has two parallel consequences. First, the depoliticisation of sites government and second, the politicisation of areas of the social world previously outside of the scope of government. This has two implications for political communities. First, with the depoliticisation of government comes increased disconnection of politics from its community. The links between governors and governed becomes weaker and other interests can expand into this political space. Second, the politicisation of aspects of the social world creates new political communities and new political activities, destabilising established governmental structures.

\paragraph{}The network metaphor, associated with sociologist Antony Giddens\footnote{\cite{Giddens:1990cm}} and geographer Manuel Castells\footnote{\cite{Castells:1996ns}}, is a rhetorical framework for considering the complexity of the interconnectedness of contemporary cosmopolitan cultures. This metaphor has, in Barry's assessment, become a ``critical term in contemporary political and economic life\ldots associated with a broad range of political opinion, and deployed in association with what might appear to be quite contradictory political strategies.''\footnote{\cite[p.85]{Barry:2001ff}} The strength of this metaphor is its ability to describe both the moment and the movement of the social world, the situated actors and their agency. The complex and often contradictory relationships of human social existence are not obscured by deploying this concept. Just as some aspects of the social world are hierarchical and regulated, and some are rhizomatic and anarchic, the metaphor of the network draws attention to these two aspects. Most recently the term network has become closely associated with information and communications technologies, especially with the internet - the prefix \textit{inter-} highlights again my theme of connectedness, of `being among'.  ``However'' as Parsons councils us to remember, ``against this is the weakness that the metaphor is highly diverse in its use and interpretation.''\footnote{\cite[p.185]{parsons:1995}} To clarify my use of this concept then as it relates to the question of ``Who Governs in a Network Society'', I am denoting the complex formal and informal inter-relationships of communication and power in the context of contemporary technologically mediated social exchanges.

\paragraph{}I find the network society metaphor significantly more sophisticated than the `globalisation thesis'. This is because a network is local at all points and global only by association and interconnection. The network metaphor is universal in that it is social.\footnote{The illustration used by Nustad (\citeyear[p.127]{Nustad:2003}) is that of a rail network. The physical points, rails, sleepers \&tc are local to the viewer at the point observed yet are a part of a complex system, global by association to the rest of the network and to other transportation networks. Yet the rail network can not universal, it can not pass through all points.} A further reason for not resting my argument on globalisation is that this notion is frequently grounded in the empirical reality of the boundaried nation-state. ``This epistemology which starts out with society as a given, consisting of so many closed, bounded entities\ldots contained entities in interaction with an equally contained society [was] modelled on the state, with its clear boundaries vis-\`{a}-vis other entities.''\footnote{\textit{Ibid.,} p.126} Yet, this is precisely where these models fall down - the functional and reductionist primacy of the nation-state ``can not capture the increasing complexity of reality in their apparatus; that is, \emph{complex in relation to earlier assumptions.}''\footnote{\textit{Ibid.,} p.126, emphasis added} In contrast the `networked society' epistemology does not start out with nation-states and society, rather it reflects the increased complexity of interactions in relation to earlier models of society.\footnote{Such as the centre-periphery models associated with Wallerstein (\citeyear{Wallerstein:1979cw}) (see also \cite{worsley:1990ow}) and Gunder Frank (\citeyear{Frank:1975du})} This description of contemporary world society reflects the stabilising and destabilising, territorialising and deterritorialising tendencies of global flows in political, economic, legal and social networks, transmuting individuals and communities, associations and corporations, states and state-like actors.

\paragraph{}It is this transmutation that is of critical interest to a number of writers who argue that this shifting of the traditional markers of certainty of the global social order denotes not a quantitative change, but a significant qualitative change in global interactivity. I am positing here a decline of the nation-state vis-\`{a}-vis the emergence of a cosmopolitan and networked society; yet this is not the whole story - I am concerned here not only with the decline of traditional  governance in a networked society, but also with the politicisation of areas of the social world previously outside of the scope of government.

\paragraph{}Thinking first then about the depoliticisation of government in the networked society. I perceive that the previously contained polity of a given nation-state has been cross-cut as the network modality of global society diminishes the containers of imperial sovereignty. The world has, as Paul Virillio describes, been turned inside out like a glove.\footnote{\cite[p. 10]{Virilio:2005jl}} The global is now inside, no longer `out there', elsewhere. Geospatial boundaries such as the nation-state can no longer be the centre of our contemporary political and economic order. Rather, these local structures have been pushed to the outside; sovereignty is now inside, pushing outwards, a centrifugal force.  He writes ``\ldots the pips are no longer inside the apples, nor the segments in the middle of the orange: \textit{the skin has been turned inside out}. The exterior is not simply the skin, the surface of the Earth, but all that is \textit{in situ}, all that is precisely localised, wherever it may be.''\footnote{\textit{Ibid.}, original emphasis} This is a reorientation of political geography of as great a significance as the Copernican revolution.

\paragraph{}Modernity's conceptions of zonal time and delineated borders are reshaped by the immediacy and ubiquity of transnational  networks, both electronic and physical. The physical distance between the \textit{real cities} of London and Baghdad remains the same as in the imperial age. Yet, our contemporary relationship to that same distance has altered radically. The distorting effect of contemporary high speed and large scale movements of symbols over seemingly vast expanses of space at a time-warping velocity necessarily tends to the concentration and centralisation of power.\footnote{\textit{Ibid.} p. 11}  For Virilio, the complex interplay between the power concentrated in the \textit{real city} and that in the \textit{virtual city} causes the imperial politics of the former to give way to a deterritorialised metropolitics of a universalising and totalitarian character.\footnote{\textit{Ibid.}}

\paragraph{}For Hardt and Negri this real/virtual dichotomy is a catalyst of a spatial-temporal reformulation of sovereignty leading to a total networked interdependence in \textit{Empire}.\footnote{\cite{Hardt:2001jl}} This new supranational sovereignty, \textit{Empire}, arises from a ``shift in contemporary capitalist production and global relations of power\ldots to bring together economic power and political power, to realise, in other words, a properly capitalist order.''\footnote{\textit{Ibid}., pp. 8-9} The paradigm of \textit{Empire} emerges positively from the ``definitive decline of the sovereign nation-state[s], by the deregulation of international markets, [and] by the end of antagonistic conflict among state subjects.''\footnote{\textit{Ibid}., p. 13} \textit{Empire} is a mixed paradigm of supranational sovereignty based on rights and the realisation of these rights through inter-networked complex systems. \textit{Empire} tends towards ``governance without government'' which further problematizes my initial question. This does not lead, according to Hardt \& Negri's analysis, to a power-vacuum, rather to the very depoliticisation of politics and the politicisation of the social world that I am concerned with in.

\paragraph{}These two takes on contemporary globality from Virilio and Hardt \& Negri each describe the depoliticisation of modern government and the site of government, the nation-state, and find the cause to be the diminishing sovereignty of each. In our networked society governments of nation-states are now one-actor-amongst-many within their notional territory, rather than the sovereign power that the modern International Relations `realist' paradigm of `black-box sovereignty' holds.\footnote{e.g. \cite{Waltz:1979}} In reality the sovereign decisions of governments are overturned by transnational quasi-governmental organisations such as the World Trade Organisation, local jurisprudence is rewritten because of a judgement by an international court, peoples national `commons' are enclosed as a conditionality of international financial or political support.

\paragraph{}New social movements are another example of such actors in our networked society.\footnote{See \cite{Dalton:1990zr} and \cite{Eder:1993fj}. A more recent introduction to New Social Movements, including aspects of hacker culture which relates to the following chapters, is \cite{Porta:2006kx}} These movements are diverse in support, objectives and strategies. The annual gatherings of NSMs at the World Social Forum (WSF) and the World Economic Forum (WEF) demonstrate fantastic abundance of nodes and associations in the network, from independent and voluntary sector organisations, to trades unions, from grassroots organisations, to international trade confederations, trans-national business, philanthropists and other non-state actors.\footnote{See \cite{Hardt:2002} for picture of Porto Algre WSF meetings as \textit{Today's Bandung}.} Most crucially the WSF and the WEF demonstrate the organising and mobilising power of these networked movements. The rhizomatic character of these actors contributes to the destabilisation of the hierarchical centralised structure of the state. The causes of these new social movements are the new political spaces in our networked society.

\paragraph{}Herein lies a critical paradox, demonstrative of current hegemonic interest-group in our networked society, and leads us closer to understanding `who governs'. When an aggregation of NGOs concerned with ecology or poverty mobilise to protect the extinction of an ecosystem or to prevent mass-starvation, national and transnational governmental structures appear to tense up, disrupting flows of information and restricting action. The heavily policed, proscribed protest zones and the dividing security fences for G8 / WTO meetings, or the political `long grass' of cross-party investigative commissions are two examples of this. Conversely, the capitulation of national and central banks to transnational financial business interests during the present `credit crunch' demonstrates how these same governmental structures can rapidly mobilise action to protect the interests of capital.

\paragraph{}While NSMs may not been able to hold the state hostage to their interests in the same way that private capital does, these movements can and do mobilise, again through networks, hundreds of thousands of activists for campaign objectives. Forty years on from the social change brought about by the movement of people on the streets; the peace movement, the student movement, the civil rights movement: what hopes are there for social change from New Social Movements in our global networked society?

\paragraph{}For Hardt \& Negri the discourse of New Social Movement theory has ``done a great service by insisting on the political importance of cultural movements against narrowly economic perspectives that minimise their significance.''\footnote{\cite[p. 275]{Hardt:2001jl}} However, they contend the post-materialist basis of New Social Movements has been accepted uncritically, that is to say NSM discourse ``perpetuate[s] narrow understandings of the economic and the cultural'' by failing to take into consideration the ``increasing indistinguishability of economic and cultural phenomena.''\footnote{\textit{Ibid}., p. 275} Therefore social movement discourse needs to develop new frameworks capable of addressing this tendency towards the convergence of economic and cultural phenomena in a networked society. Free/libre, open-source and commons conceptions recognise this convergence and, I argue, offer a rich  contribution to NSM discourse. This is an avenue I continue to explore in a later section.

\paragraph{}Several theorists of network culture argue that the increasing transnational mobility of both individuals and symbols in a network society should challenge us to think again about other dimensions of politics. The work of Appadurai in exploring ``the image, the imagined, the imaginary\ldots the imagination as a social practice''\footnote{\cite[p. 31, original emphasis removed]{Appadurai:1996lp}} in a networked society considers one such dimension. He argues that ``the imagination has become an organised field of social practice, a form of work (in the sense of both labour and culturally organised practice), and a form of negotiation between sites of agency (individuals) and globally defined fields of possibility.''\footnote{\textit{Ibid}.}

\paragraph{}Appadurai's transnational `mobility without moving', of identities and cultures in transmission, is made possible by the ever expansive `building out' of the infrastructure of global networks, interconnecting the social world. The interaction of global flows of images, identities and cultures in the networked society adds further complexity to the constitutive effect of images, the imagined, the imaginary. The question of `Who Governs' is further problematised, too. When the identity of each individual in any given polity is, if not constructed, then at least influenced by these global flows, that polity itself is destabilised from within. The extreme plurality of contemporary liberal societies tends towards `governance without government' as a remedy to this destabilisation.\footnote{\cite[pp. 13-14]{Hardt:2001jl}} This manifests itself in two ways. The first can be seen in the institutions of government in the establishing of bi-partisan committees or agencies at arms length of the state, and therefore from the democratic process, depoliticising governance in pursuit of third-way politics. The second aspect of `governance without government' is more a counter-weight, a response; here we see NGOs/NSMs as powerful aggregations of individual interest, operating as the moral exclamation in resistance to the first depoliticisation. The site of representation and resistance is now global and networked. Conditioned by this, these movements are their own network and the network extends the movement.\footnote{For discussion of the Environmental movement as a network see \cite{rosenblatt:2004mn}.}

\paragraph{}How then are we to understand political struggle when the site of representation and resistance is now global and networked? The most crucial characteristic of the network society, as described by Hardt, is that ``no two nodes face each other in contradiction; rather, they are always triangulated [\ldots] by an indefinite number of others''.\footnote{\cite[p. 117]{Hardt:2002}} This means that political struggle in our networked society is no longer directly antagonistic, yet neither is it passive. The resistance and activism movements organised within networks ``displace contradictions and operate instead a kind of alchemy, or rather a sea change, the flow of the movements transforming the traditional fixed positions; networks imposing their force through a kind of irresistible undertow.''\footnote{\cite[p. 117]{Hardt:2002}}

\paragraph{}These social processes of depoliticisation / politicisation are global. As Castells describes the cyclical events\ldots ``societies evolve and change by deconstructing their institutions under the pressure of new power relationships and constructing new sets of institutions that allow people to live side by side without self-destroying, in spite of their contradictory interests and values.''\footnote{\cite[p. 258]{castells:2007cp}} The shifting nature of power in a networked society is amplified and accelerated as the plural and contained polity of the imperial age gives way to the  networked society.

\paragraph{}I began this paper asking `Who Governs'. In journeying beyond the waining imperial order into the politics of the networked society we have seen the destabilisation and deterritorialisation of the site and agent of government, the nation-state, and recognised the distributed and diffuse nature of power in our networked society. I have considered how the same destabilising effects of global networks have had an affect on the identity of the individual, the subject of sovereignty. The problem of `Who Governs in a Networked Society?' has become more complex, provoking more questions. Two goals emerge clearly from this discussion, the articulation of a new `network aware' ontology and an account of politics in a network society. Both are needed in order to begin to frame and answer to the question `Who Governs'.
