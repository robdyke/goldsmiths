\paragraph{}For Anderson the dynamic co-development of human societies and technologies conditions the social order.\footnote{\cite{Anderson:1991ic}} The `nation' is an imagined community manifested in 'states'; the emergence of this political category is closely related to the spread of technologies that increased inter-personal communication in vernacular languages. The global networks that have extended and accelerated communications in our contemporary world have facilitated the multiplication of imagined communities which now scale globally. Hackers communities are the architype of a networked community. I think that the relational networked communities and the political culture that conditions the governance of the same provide useful examples for developments in our contemporary networked society.

\paragraph{}Hackers reproducing themselves as they are are also reproducing their culture.\footnote{\cite{Appadurai:1996lp}} The reproduction of the hacker identity, an identity innately sensitive to 'being among', in a networked society happens in a number of ways. Philosophically and legally through the Free Software definitions and the GNU General Public License. These are community currated constitutional documents. Members of f/los communities can and do participate in the revision of these documents.

\paragraph{}The idea of `community' is frequently evoked in f/los discourse and most f/los activity takes place within a group, with disparate and dispersed members. The role of `the community' in maintaining and distributing knowledge and in inculcating a good f/los civic citizen is central to the hacker ethos.\footnote{Note the distinction here. Contemporary capitalist societies value education, placing an emphasis on skills for employment, training for jobs. Hackers and f/los society, in contrast to the Fordist production-line approach of contemporay capital, emphasises knowledge and learning.}