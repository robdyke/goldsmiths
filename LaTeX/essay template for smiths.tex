\documentclass[12pt,a4paper,titlepage]{article}
\usepackage{geometry}
\usepackage{helvet}
\usepackage{harvard}
\usepackage{setspace}
\renewcommand{\familydefault}{\sfdefault}
\geometry{a4paper}
\title{Your title here\ldots}
\author{Student Registration No. 22164733}
\date{The submission date\ldots}

\begin{document}
\harvardparenthesis{none}
\citationstyle{dcu}
\bibliographystyle{agsm}
\maketitle
\doublespacing
\begin{quote}
\textit{The friend-enemy distinction identifies ``exactly who is to be affected, combated, refuted, or negated'' in political language and in political actions.\footnote{\cite[p.31]{schmitt:2007cop}}}
\end{quote}

\paragraph{}Schmitt's definition of `the political' is forceful due to its simplicity. His concept of `the political' is the antithesis of friend and enemy. Schmitt's friend/enemy dualism is a ``definition in the sense of a criterion and not as an exhaustive definition or indicative of substantial content.''\footnote{\cite[p.26]{schmitt:2007cop}} Crucially this antithesis is independent of other antithesis, such as moral, religious, rational, legal or economic antithesis. The use of the word friend is in its down to earth, literal sense. The meaning and therefore the identity of friend in this antithesis is easily understood. So while non-friends may appear immoral, heretical, irrational, illegal, or uneconomic these criterion would not be enough to distinguish these non-friends as `enemy'; indeed these are as likely to be characteristics of our `friends', in Schmitt's antithesis, as of our `enemies'. An enemy is separated from the union of friends; friends associate with one another and are dissociated from their enemies. The enemy is stranger, other, different, alien. Schmitt is not concerned here with private confrontations and competitions. Both friend and enemy are public. It is the public definition of friend and of enemy that makes this antagonism \emph{political}. Schmitt's Concept of the Political then is this antagonism, the ``most intense and extreme antagonism \ldots [becoming] that much more political the closer it approaches the \emph{most extreme point}, that of the friend-enemy grouping.''\footnote{\cite[p.29, emphasis added]{schmitt:2007cop}}
\newpage
\singlespacing
\bibliography{/Users/robdyke/Documents/GoldsmithsCourses/bibliography/globalbib}
\medskip
\paragraph{}Words : xxx
\end{document}