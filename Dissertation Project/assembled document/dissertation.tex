\documentclass[11pt,titlepage]{book}
\usepackage{graphicx}
\usepackage[paperheight=9in,paperwidth=6in]{geometry}
\usepackage{helvet}
\usepackage{harvard}
\usepackage{setspace}
\usepackage[toc,page]{appendix}
\usepackage[none]{hyphenat}
\renewcommand{\familydefault}{\sfdefault}
\renewcommand{\chaptermark}[1]{%
\markboth{#1}{}}
\title{Hacking the Networked Society.}
\author{Rob Dyke}
\date{\today}
\begin{document}
\harvardparenthesis{none}
\citationstyle{dcu}
\bibliographystyle{dcu}
\sloppy
\maketitle
\newpage
\phantomsection{}
\paragraph{Creative Commons License.}\newline This work is licensed under the Creative Commons Attribution-Noncommercial-Share Alike 3.0 Unported License. To view a copy of this license, visit \url{http://creativecommons.org/licenses/by-nc-sa/3.0/} or send a letter to Creative Commons, 171 Second Street, Suite 300, San Francisco, California, 94105, USA.
\newline\newline
\includegraphics{by-nc-sa.pdf}
\newline\newline
\paragraph{Contact Information.}Please email me at this address: \email{emailme@robdyke.com} or visit my website \www{http://www.robdyke.com}
\tableofcontents
\newpage
\doublespacing
\chapter{Introduction.}
\paragraph{}The dynamic between free-software and open-source is often misunderstood by social and political theorists. As a consequence it is also under-theorised within socio-political theory. In this paper, I show how philosophies of free/libre, open-source and commons regimes have engendered new forms of sociopolitical consumption and new political economies of meaning. My emphasis on the interplay between the local and the global/structure and agency, shows new ways of `thinking' the cosmopolitan, sedimented in the interconnected networks of the technical age. My thesis is concerned with our present moment of opportunity. I believe that positive possibilities for politics and political economy are presented in the philosophies of free/libre, open-source and commons regimes. In this paper I will demonstrate the contribution of these new sociopolitical categories and the new politics that is being made public because of free/libre hacking.

\paragraph{}My thesis explores two themes: networks and movement. I recognise in the literature around social movements, global politics and government a similar interest in networks that develops in tandem with advances in physical and technical networks, such as transportation, computing or utilities. I, too, am interested in networks in both social and technical senses of the concept. According to Parsons the network concept was attractive to social movement and public policy theorists of the 1950 and 1960s because this model was flexible enough to describe the fluid and complex interplay between both the formal and informal political and social relationships that condition political culture.\footnote{\cite{parsons:1995}} During the 1980s the network concept was developed by a number of French sociologists. I have been particularly interested in the work of Bruno Latour in this area. He is associated with the Actor-Network-Theory (ANT), a sociology which claims that ``modern societies cannot be described without recognizing them as having a fibrous, thread-like, wiry, stringy, ropey, capillary character that is never captured by the notions of levels, layers, territories, spheres, categories, structure, systems. [ANT] aims at explaining the effects accounted for by those traditional words without having to buy the ontology, topology and politics that goes with them.''\footnote{\cite{latour:1998ant1}}

\paragraph{}In terms of global society, the network metaphor has been widely deployed in the field of international relations, economics, anthropology and socio-political geography. Most prominent is the geographer Manuel Castells whose seminal work \textit{The Rise of the Network Society}\footnote{\cite{Castells:1996ns}} considers the social and political dynamics created by the flows of people, goods and services, and capital around the world in terms of networks. More recent examples of the use of the network concept can be found in the ethnographies of \citename{Appadurai:1996lp},\footnote{\cite{Appadurai:1996lp}} the political/cultural research of \citename{Terranova:2004ly},\footnote{\cite{Terranova:2004ly}} and the socio-political geography of \citename{Barry:2001ff}.\footnote{\cite{Barry:2001ff}} The network concept has helped these authors to move beyond the `ontologies, topologies and politics' of world systems theorists, hyper-globalist free-marketeers and models of dependency and development.

\paragraph{}Most crucial in the ANT approach is that this sociology includes the material; that is to say it recognises ``the facts manufactured by natural and social sciences and the artefacts designed by engineers.''\footnote{\cite{latour:1998ant1}} Some may be critical of this approach, fearing a technological determinism or a cyborg future. However, I agree with Benkler who, in arguing that strict technological determinism\footnote{Understood as ``if you have technology `t,' you should expect social structure or relation 's' to emerge.'' \cite[pp. 11-12]{benkler:2006wn}} is false, states that ``different patterns of adoption and use [of a certain technology] can result in very different social relations.''\footnote{\textit{Ibid}., pp. 11-12} Benkler finds wealth in the network and not in nations, as Adam Smith did. His argument in \textit{The Wealth Of Networks} considers both the technological and social aspects of relational (re)production in networked society. 

\paragraph{}The closely related ideas of free/libre, open-source and commons regimes have developed in a similar time-frame to the network concept. These ideas have developed symbiotically with the rise of networks in the technological sense too, most crucially the global information-computation network, the internet. Computer hackers and political activists Richard Stallman and Eric S. Raymond along with lawyer Laurence Lessig are the people most strongly associated with free/libre and open-source philosophies and commons regimes, respectively.\footnote{\cite{Stallman:2002xy}, \cite{raymond:1999catb} \& \cite{Lessig:2004pi}} These individuals are recognised as pioneers in their areas and their work is widely cited. Stallman's work is focused on articulating and developing the free/libre philosophy and curating several `constitutional' documents. For Raymond, amateur anthropologist and `accidental revolutionary', it is the socio-economic manifestation of free/libre philosophy as open-source in a networked society that is of importance. In his book he explores what he perceives as a shift from organisational hierarchy, which he describes as a Cathedral, to the contrasting horizontal networks of a Bazaar. Lessig's work focuses on the problematic relationship between culture and property in advanced capitalist societies, concentrating on commons regimes.

\paragraph{}A second aspect of my dissertation is concerned with movement and movements, again considering social and technical aspects. I am interested in social movements, for example the socio-political organisations considered by New Social Movement (NSM)\footnote{Two excellent surveys of New Social Movement discourse are \cite{daltonetal1990} and \cite{Porta:2006kx}} theorists, and in technical movement, considering the flows of signs and symbols in transmission. Both movement and movements are considered through the rubric of global networked society. The groups and communities around free/libre and open-source philosophies and commons production regimes form what has been described by many as a New Social Movement. I am interested in the movement of concepts and practices from free/libre and open-source groups to other social movements and political organisations in our networked society.

\paragraph{}My approach in this thesis is to pay attention to both the interconnectedness of nodes in networks, be they biological or techno-sociological, and the movement of signs and symbols in those same networks. I have taken a hyper-networked attitude in my thesis. There is nothing but networks or, to use Latour's words, ``there is no \textit{aether} in which the networks should be immersed.''\footnote{\cite{latour:1998ant1}} In this sense the network concept is both reductive and relativist. Yet, when considered in parallel with the political philosophy of free/libre, open source, and commons ideals, the network concept takes us towards an ontology that is relational and not reductive. I acknowledge the atomic, i.e. indivisible, nature of each individual being. Yet, my ontology of this being also recognises the multi-dimensional inter-connections, that is to say the networks that each individual is a node in.\footnote{Latour describes the network concept as  ``a change of metaphors to describe essences: instead of surfaces one gets filaments (or rhizomes in Deleuze's parlance).'' \cite{latour:1998ant1}, citing Gilles Deleuze et F\'{e}lix Guattari, \textit{Mille plateaux. Capitalisme et schizophr\'{e}nie}, Minuit, Paris, (1980).}

\paragraph{}My dissertation has four parts. Firstly, I pose the question `Who Governs in a Networked Society?', considering the effect of global dynamics on modern conceptions of subject, governance and sovereignty. The post-modern history of our contemporary geospatial and political present presented by Hardt and Negri in \textit{Empire}\footnote{\cite{Hardt:2001jl}} has been a critical influence on my thoughts on networks and movement in this regard.

\paragraph{}Freedom and openness are two central concepts in hackerism and the contemporary social movements it has influenced. In the second section of this paper I explore in depth the political philosophy of free/libre and open source ideas that have arisen from hacker culture. In examining the epistemology of these emerging redefinitions of freedom and openness I demonstrate the challenges and opportunities they present to the corresponding dominant liberal notions of the same terms.

\paragraph{}The third section of this paper engages with the question of production and reproduction under the conditions of `communicative capitalism'\footnote{After \cite{dean:2005cc}, \textit{The Networked Empire: Communicative Capitalism and the Hope for Politics}} and the `positive possibility'\footnote{After \cite{Hardie:2005px}, \textit{Change of the Century: Free Software and the Positive Possibility}} presented by free/libre and open-source conceptions of the same. 

\paragraph{}The ``contemporary proliferation of political spaces and the multiplicity of democratic demands''\footnote{\cite[p.17]{Mouffe:2000fk}} needs a new politics, a politics that works at the global and the local level, the universal and the particular. New relationships of interconnectedness between people, communities, identities, artifacts, locations and events need to be reflected in the political culture of our institutions. What are the dynamics of these relationships? How are these new political movements creating spaces in our networked society? Furthermore, with ``advocates of freedom in the new digital society [\ldots] decried as pirates, anarchists, communists''\footnote{\cite{moglen:2003dcm}} by the those who dominated the imperial/industrial economy - those who have the most to loose - the stakes are clearly high. We are witnessing, as Moglen announces in his work \textit{The dotCommunist Manifesto}, ``the arrival of a new social structure, born of the transformation of bourgeois industrial society by the digital technology of its own invention.''\footnote{\textit{Ibid}.} This networked remixing and retransmission of free/libre philosophy, directed back at the forces of capital in liberal democracies that (re)produced and conditioned its emergence presents, perhaps, the challenges and the new opportunites for radical political communities and political economy. The effects of free/libre, open source and commons in transmission will conclude my dissertation.
\newpage
\chapter{Who Governs in a Networked Society?}
\paragraph{}Dahl asked `Who Governs?' and found a gradual shift from oligarchy to pluralism.\footnote{\cite{Dahl:1961fj}} Barry's \emph{Political Machines}\footnote{\cite{Barry:2001ff}} was concerned with `governing a technological society'\ldots This section, then, develops both of these lines of enquiry and can be conceived as `Who Governs in a Networked Society?' I am concerned with the present and the shift from pluralism to networked society. My focus here is to consider `interconnectedness', revealing the relationships in and between complex social systems. I contend that ever increasing complexity and ever greater plurality developing in tandem in the political, economic, and cultural spheres has two parallel consequences. First, the depoliticisation of sites government and second, the politicisation of areas of the social world previously outside of the scope of government. This has two implications for political communities. First, with the depoliticisation of government comes increased disconnection of politics from its community. The links between governors and governed becomes weaker and other interests can expand into this political space. Second, the politicisation of aspects of the social world creates new political communities and new political activities, destabilising established governmental structures.

\paragraph{}The network metaphor, associated with sociologist Antony Giddens\footnote{\cite{Giddens:1990cm}} and geographer Manuel Castells\footnote{\cite{Castells:1996ns}}, is a rhetorical framework for considering the complexity of the interconnectedness of contemporary cosmopolitan cultures. This metaphor has, in Barry's assessment, become a ``critical term in contemporary political and economic life\ldots associated with a broad range of political opinion, and deployed in association with what might appear to be quite contradictory political strategies.''\footnote{\cite[p.85]{Barry:2001ff}} The strength of this metaphor is its ability to describe both the moment and the movement of the social world, the situated actors and their agency. The complex and often contradictory relationships of human social existence are not obscured by deploying this concept. Just as some aspects of the social world are hierarchical and regulated, and some are rhizomatic and anarchic, the metaphor of the network draws attention to these two aspects. Most recently the term network has become closely associated with information and communications technologies, especially with the internet - the prefix \textit{inter-} highlights again my theme of connectedness, of `being among'.  ``However'' as Parsons councils us to remember, ``against this is the weakness that the metaphor is highly diverse in its use and interpretation.''\footnote{\cite[p.185]{parsons:1995}} To clarify my use of this concept then as it relates to the question of ``Who Governs in a Network Society'', I am denoting the complex formal and informal inter-relationships of communication and power in the context of contemporary technologically mediated social exchanges.

\paragraph{}I find the network society metaphor significantly more sophisticated than the `globalisation thesis'. This is because a network is local at all points and global only by association and interconnection. The network metaphor is universal in that it is social.\footnote{The illustration used by Nustad (\citeyear[p.127]{Nustad:2003}) is that of a rail network. The physical points, rails, sleepers \&tc are local to the viewer at the point observed yet are a part of a complex system, global by association to the rest of the network and to other transportation networks. Yet the rail network can not universal, it can not pass through all points.} A further reason for not resting my argument on globalisation is that this notion is frequently grounded in the empirical reality of the boundaried nation-state. ``This epistemology which starts out with society as a given, consisting of so many closed, bounded entities\ldots contained entities in interaction with an equally contained society [was] modelled on the state, with its clear boundaries vis-\`{a}-vis other entities.''\footnote{\textit{Ibid.,} p.126} Yet, this is precisely where these models fall down - the functional and reductionist primacy of the nation-state ``can not capture the increasing complexity of reality in their apparatus; that is, \emph{complex in relation to earlier assumptions.}''\footnote{\textit{Ibid.,} p.126, emphasis added} In contrast the `networked society' epistemology does not start out with nation-states and society, rather it reflects the increased complexity of interactions in relation to earlier models of society.\footnote{Such as the centre-periphery models associated with Wallerstein (\citeyear{Wallerstein:1979cw}) (see also \cite{worsley:1990ow}) and Gunder Frank (\citeyear{Frank:1975du})} This description of contemporary world society reflects the stabilising and destabilising, territorialising and deterritorialising tendencies of global flows in political, economic, legal and social networks, transmuting individuals and communities, associations and corporations, states and state-like actors.

\paragraph{}It is this transmutation that is of critical interest to a number of writers who argue that this shifting of the traditional markers of certainty of the global social order denotes not a quantitative change, but a significant qualitative change in global interactivity. I am positing here a decline of the nation-state vis-\`{a}-vis the emergence of a cosmopolitan and networked society; yet this is not the whole story - I am concerned here not only with the decline of traditional  governance in a networked society, but also with the politicisation of areas of the social world previously outside of the scope of government.

\paragraph{}Thinking first then about the depoliticisation of government in the networked society. I perceive that the previously contained polity of a given nation-state has been cross-cut as the network modality of global society diminishes the containers of imperial sovereignty. The world has, as Paul Virillio describes, been turned inside out like a glove.\footnote{\cite[p. 10]{Virilio:2005jl}} The global is now inside, no longer `out there', elsewhere. Geospatial boundaries such as the nation-state can no longer be the centre of our contemporary political and economic order. Rather, these local structures have been pushed to the outside; sovereignty is now inside, pushing outwards, a centrifugal force.  He writes ``\ldots the pips are no longer inside the apples, nor the segments in the middle of the orange: \textit{the skin has been turned inside out}. The exterior is not simply the skin, the surface of the Earth, but all that is \textit{in situ}, all that is precisely localised, wherever it may be.''\footnote{\textit{Ibid.}, original emphasis} This is a reorientation of political geography of as great a significance as the Copernican revolution.

\paragraph{}Modernity's conceptions of zonal time and delineated borders are reshaped by the immediacy and ubiquity of transnational  networks, both electronic and physical. The physical distance between the \textit{real cities} of London and Baghdad remains the same as in the imperial age. Yet, our contemporary relationship to that same distance has altered radically. The distorting effect of contemporary high speed and large scale movements of symbols over seemingly vast expanses of space at a time-warping velocity necessarily tends to the concentration and centralisation of power.\footnote{\textit{Ibid.} p. 11}  For Virilio, the complex interplay between the power concentrated in the \textit{real city} and that in the \textit{virtual city} causes the imperial politics of the former to give way to a deterritorialised metropolitics of a universalising and totalitarian character.\footnote{\textit{Ibid.}}

\paragraph{}For Hardt and Negri this real/virtual dichotomy is a catalyst of a spatial-temporal reformulation of sovereignty leading to a total networked interdependence in \textit{Empire}.\footnote{\cite{Hardt:2001jl}} This new supranational sovereignty, \textit{Empire}, arises from a ``shift in contemporary capitalist production and global relations of power\ldots to bring together economic power and political power, to realise, in other words, a properly capitalist order.''\footnote{\textit{Ibid}., pp. 8-9} The paradigm of \textit{Empire} emerges positively from the ``definitive decline of the sovereign nation-state[s], by the deregulation of international markets, [and] by the end of antagonistic conflict among state subjects.''\footnote{\textit{Ibid}., p. 13} \textit{Empire} is a mixed paradigm of supranational sovereignty based on rights and the realisation of these rights through inter-networked complex systems. \textit{Empire} tends towards ``governance without government'' which further problematizes my initial question. This does not lead, according to Hardt \& Negri's analysis, to a power-vacuum, rather to the very depoliticisation of politics and the politicisation of the social world that I am concerned with in.

\paragraph{}These two takes on contemporary globality from Virilio and Hardt \& Negri each describe the depoliticisation of modern government and the site of government, the nation-state, and find the cause to be the diminishing sovereignty of each. In our networked society governments of nation-states are now one-actor-amongst-many within their notional territory, rather than the sovereign power that the modern International Relations `realist' paradigm of `black-box sovereignty' holds.\footnote{e.g. \cite{Waltz:1979}} In reality the sovereign decisions of governments are overturned by transnational quasi-governmental organisations such as the World Trade Organisation, local jurisprudence is rewritten because of a judgement by an international court, peoples national `commons' are enclosed as a conditionality of international financial or political support.

\paragraph{}New social movements are another example of such actors in our networked society.\footnote{See \cite{Dalton:1990zr} and \cite{Eder:1993fj}. A more recent introduction to New Social Movements, including aspects of hacker culture which relates to the following chapters, is \cite{Porta:2006kx}} These movements are diverse in support, objectives and strategies. The annual gatherings of NSMs at the World Social Forum (WSF) and the World Economic Forum (WEF) demonstrate fantastic abundance of nodes and associations in the network, from independent and voluntary sector organisations, to trades unions, from grassroots organisations, to international trade confederations, trans-national business, philanthropists and other non-state actors.\footnote{See \cite{Hardt:2002} for picture of Porto Algre WSF meetings as \textit{Today's Bandung}.} Most crucially the WSF and the WEF demonstrate the organising and mobilising power of these networked movements. The rhizomatic character of these actors contributes to the destabilisation of the hierarchical centralised structure of the state. The causes of these new social movements are the new political spaces in our networked society.

\paragraph{}Herein lies a critical paradox, demonstrative of current hegemonic interest-group in our networked society, and leads us closer to understanding `who governs'. When an aggregation of NGOs concerned with ecology or poverty mobilise to protect the extinction of an ecosystem or to prevent mass-starvation, national and transnational governmental structures appear to tense up, disrupting flows of information and restricting action. The heavily policed, proscribed protest zones and the dividing security fences for G8 / WTO meetings, or the political `long grass' of cross-party investigative commissions are two examples of this. Conversely, the capitulation of national and central banks to transnational financial business interests during the present `credit crunch' demonstrates how these same governmental structures can rapidly mobilise action to protect the interests of capital.

\paragraph{}While NSMs may not been able to hold the state hostage to their interests in the same way that private capital does, these movements can and do mobilise, again through networks, hundreds of thousands of activists for campaign objectives. Forty years on from the social change brought about by the movement of people on the streets; the peace movement, the student movement, the civil rights movement: what hopes are there for social change from New Social Movements in our global networked society?

\paragraph{}For Hardt \& Negri the discourse of New Social Movement theory has ``done a great service by insisting on the political importance of cultural movements against narrowly economic perspectives that minimise their significance.''\footnote{\cite[p. 275]{Hardt:2001jl}} However, they contend the post-materialist basis of New Social Movements has been accepted uncritically, that is to say NSM discourse ``perpetuate[s] narrow understandings of the economic and the cultural'' by failing to take into consideration the ``increasing indistinguishability of economic and cultural phenomena.''\footnote{\textit{Ibid}., p. 275} Therefore social movement discourse needs to develop new frameworks capable of addressing this tendency towards the convergence of economic and cultural phenomena in a networked society. Free/libre, open-source and commons conceptions recognise this convergence and, I argue, offer a rich  contribution to NSM discourse. This is an avenue I continue to explore in a later section.

\paragraph{}Several theorists of network culture argue that the increasing transnational mobility of both individuals and symbols in a network society should challenge us to think again about other dimensions of politics. The work of Appadurai in exploring ``the image, the imagined, the imaginary\ldots the imagination as a social practice''\footnote{\cite[p. 31, original emphasis removed]{Appadurai:1996lp}} in a networked society considers one such dimension. He argues that ``the imagination has become an organised field of social practice, a form of work (in the sense of both labour and culturally organised practice), and a form of negotiation between sites of agency (individuals) and globally defined fields of possibility.''\footnote{\textit{Ibid}.}

\paragraph{}Appadurai's transnational `mobility without moving', of identities and cultures in transmission, is made possible by the ever expansive `building out' of the infrastructure of global networks, interconnecting the social world. The interaction of global flows of images, identities and cultures in the networked society adds further complexity to the constitutive effect of images, the imagined, the imaginary. The question of `Who Governs' is further problematised, too. When the identity of each individual in any given polity is, if not constructed, then at least influenced by these global flows, that polity itself is destabilised from within. The extreme plurality of contemporary liberal societies tends towards `governance without government' as a remedy to this destabilisation.\footnote{\cite[pp. 13-14]{Hardt:2001jl}} This manifests itself in two ways. The first can be seen in the institutions of government in the establishing of bi-partisan committees or agencies at arms length of the state, and therefore from the democratic process, depoliticising governance in pursuit of third-way politics. The second aspect of `governance without government' is more a counter-weight, a response; here we see NGOs/NSMs as powerful aggregations of individual interest, operating as the moral exclamation in resistance to the first depoliticisation. The site of representation and resistance is now global and networked. Conditioned by this, these movements are their own network and the network extends the movement.\footnote{For discussion of the Environmental movement as a network see \cite{rosenblatt:2004mn}.}

\paragraph{}How then are we to understand political struggle when the site of representation and resistance is now global and networked? The most crucial characteristic of the network society, as described by Hardt, is that ``no two nodes face each other in contradiction; rather, they are always triangulated [\ldots] by an indefinite number of others''.\footnote{\cite[p. 117]{Hardt:2002}} This means that political struggle in our networked society is no longer directly antagonistic, yet neither is it passive. The resistance and activism movements organised within networks ``displace contradictions and operate instead a kind of alchemy, or rather a sea change, the flow of the movements transforming the traditional fixed positions; networks imposing their force through a kind of irresistible undertow.''\footnote{\cite[p. 117]{Hardt:2002}}

\paragraph{}These social processes of depoliticisation / politicisation are global. As Castells describes the cyclical events\ldots ``societies evolve and change by deconstructing their institutions under the pressure of new power relationships and constructing new sets of institutions that allow people to live side by side without self-destroying, in spite of their contradictory interests and values.''\footnote{\cite[p. 258]{castells:2007cp}} The shifting nature of power in a networked society is amplified and accelerated as the plural and contained polity of the imperial age gives way to the  networked society.

\paragraph{}I began this paper asking `Who Governs'. In journeying beyond the waining imperial order into the politics of the networked society we have seen the destabilisation and deterritorialisation of the site and agent of government, the nation-state, and recognised the distributed and diffuse nature of power in our networked society. I have considered how the same destabilising effects of global networks have had an affect on the identity of the individual, the subject of sovereignty. The problem of `Who Governs in a Networked Society?' has become more complex, provoking more questions. Two goals emerge clearly from this discussion, the articulation of a new `network aware' ontology and an account of politics in a network society. Both are needed in order to begin to frame and answer to the question `Who Governs'.
\newpage
\chapter{The Free/Open Community.}
\paragraph{}The networked meshing of hackers, artists and activists has made public free/libre, open-source and creative-commons concepts.\footnote{I will use f/los when referring to both free/libre and open-source together in this paper. A more common acronym, FLOSS, demonstrates the genealogy of the term and its origins in computing.} This section will consider the hacker identity and hacker social and cultural relations. I posit that increasing contact with f/los \& creative-commons artifacts, technologies and economics and the wider dissemination of f/los \& creative-commons memes is raising popular consciousness of several emerging redefinitions for concepts such as freedom, property, production and ownership. Freedom and openness are two central concepts in hackerism and the contemporary social movements it has influenced. This section examines critically the political philosophies of hackerism in order to reveal the meaning of these emerging redefinitions of freedom and openness and the challenges they present to the corresponding dominant liberal notions of the same terms.

\paragraph{}To hack is to play, to learn, to survive.\footnote{\cite[pp. 48-53 and the prologue `Linus's Law']{himanen:2001he}} The hacker supports his/her own being through the application of his/her individual characteristics in relation to his/her social and material conditions and gains pleasurable satisfaction in doing so. As MacKenzie Wark puts it in \textit{A Hacker Manifesto}, the hacker ``produces the possibility of production, the possibility of making something of and with the world - and of living off the surplus produced by the application of abstraction to nature - to any nature.''\footnote{\cite[077]{wark:2004hm} There are no page numbers in Wark's \textit{Manifesto}, simply numbered points/demands. These appear in the place of page numbers in my references.}. The hacker disposition is demonstrated in the real, lived lives of humans and the development of all human civilisations. What I am suggesting by ascribing these characteristics to all humans qua human is that the hacker is the innate modality of humanity. Hacking, according to Wark, discovers ``the nature of nature, its productive - and destructive - powers. This applies as much in physics as in sexuality, in biology as in politics, in computing as in art or philosophy. The nature of any and every domain may be hacked. It is the nature of hacking to discover freely, to invent freely, to create and produce freely.''\footnote{\cite[076]{wark:2004hm}. In \textit{Empire} Hardt and Negri argue that capital has a similarly expansive character. Just as the hacker sees no boundary to what may be hacked, capital similarly has no boundaries, expanding into culture, science, sex.}

\paragraph{}To hack is also to subvert and resist. The desire to hack the information society emerges negatively, in opposition to the challenge of increasingly rapid enclosure of the immaterial and cultural commons and against the dislocation from history and society that virtual and networked property relations effect. Indeed, as both Wark and Himanen have argued, it is the hacker who is capable of answering the property question in capitalist societies by proposing an ``alternative spirit for the network society''\footnote{\cite[p. 12]{himanen:2001he}} creating ``new kinds of relation, with unforeseen properties, which \textit{question the property form itself}''\footnote{\cite[036, emphasis added.]{wark:2004hm}}. This presents a challenge to the dominant spirit of contemporary advanced capitalism. The hacker proposes free production and free access in contrast to the private property of capital. I will return to these new kinds of relationships later.

\paragraph{}Modern freedom, the liberty of contemporary liberals, emerged negatively, as a revolution against the dominant absolutism and authority of the pre-modern period. The freedom of free/libre also emerges negatively, a revolution against the widespread practice of enclosure of technologies and the encroachment of enclosure in the academy. In \textit{Revolutions OS}\footnote{\cite{moore:2002rv}} Richard Stallman, founder of the Free Software Foundation and computer hacker, describes his hostility the to enclosure of the `tools of his trade', in this case, software, in the guise of `intellectual property.'\footnote{For Stallman, ``The term “intellectual property” carries a hidden assumption - that the way to think about all these disparate issues [copyright, patents, trademarks] is based on an analogy with physical objects, and our ideas of physical property.'' \url{http://www.gnu.org/philosophy/words-to-avoid.html}} He felt his was being forced to ``sign a promise you won't share [software] with anybody else.'' He continues\ldots

\begin{quote}``And to me that was essentially a promise to be a bad person, to betray or cut myself off from society, from the cooperating community. So, I just wasn't going to do that. I felt this is wrong. I am not going to live this way.''\footnote{\cite[9m55sec]{moore:2002rv}}
\end{quote}

\paragraph{}This led Stallman and others to develop and share free and open tools. This work began in the 1980s under the banner of The GNU Project\footnote{GNU is recursive acronym that stands for ``GNU's Not Unix'', it is, says Stallman ``a hack'' \cite[11m35sec]{moore:2002rv}} and led to the foundation of the Free Software Foundation (FSF).  Stallman's political and legal work is part of the foundational philosophy of the free/libre and, later, open source movements.

\paragraph{}The freedom Stallman and the FSF espouse in the Free Software Definition (FSD)\footnote{See \url{http://www.gnu.org/philosophy/free-sw.html}} is free-as-in-freedom, that is to say positive freedom: `free-as-in-speech' not `free-as-in-beer', i.e. \textit{gratis}, no charge. The four freedoms of the FSD are:

\begin{itemize}
\item Freedom 0 - The freedom to run the program, for any purpose
\item Freedom 1 - The freedom to study how the program works, and adapt it to your needs. Access to the source code is a precondition for this.
\item Freedom 2 - The freedom to redistribute copies so you can help your neighbour.
\item Freedom 3 - The freedom to improve the program, and release your improvements to the public, so that the whole community benefits. Access to the source code is a precondition for this.\footnote{See \url{http://www.gnu.org/philosophy/free-sw.html}}
\end{itemize}

\paragraph{}Stallman's reasoning behind this radical stance is evident in the title of his collected essays \textit{Free Software, Free Society}.\footnote{\cite{Stallman:2002xy}. See \cite{Williams:2007uq} for a critical engagement with FSF philosophy and \cite{Stallabrass:2002kx} for a review of Williams's book.} He askes ``what kind of rules make possible a good society that is good for the people in it?''\footnote{\cite[Ch. 8]{Williams:2007uq}} The FSD goes some way to answering that question. The FSD describes ``the freedoms that enable people to form a community.''\footnote{\cite[15m30sec]{moore:2002rv}} The definition seeks to protect an individuals' ``right to cooperate with other people and form a community.''\footnote{\textit{Ibid.,}17m32sec} Stallman's point is clear: ``if you don't have all these freedoms [in the FSD], you're being divided and dominated by somebody''\footnote{\textit{Ibid.,}15m30sec} Some hackers, such as Hill and Coleman, contend that the ``four freedoms'' of the FSD are ``based in and representative of an extreme form of anti-discrimination resistant to categorisation into the typical ``left, centre and right'' tripartite political schema.''\footnote{\cite{coleman:2004fo}} Yet free/libre is not completely resistant to categorisation. I contend that the `freedom' of free/libre is based on an appeal to rights, an exclamation of a right of self-determination and self-selection of a community and of property. Therefore, in many respects, the definition of the ``freedom'' ideal-type for free/libre is markedly liberal and individualistic in ontology and political economy.

\paragraph{}The FSF is, however, only one voice among the multitude of f/los hackers, artists and activists; the radicalism of the FSD is only one expression of hackerism. The free/libre philosophy in transmission, as one set of symbols among many in a networked society, can be found in different practical articulations. That is to say that by ``recalibrating the broad meaning of freedom outlined in the FSD to align with their own philosophies and politics\ldots groups perceive [f/los] as a model of openness and collaboration particularly well suited to meet their own goals.''\footnote{\textit{Ibid.}} 

\paragraph{}Here the second part of the f/los acronym emerges, \textit{open-source}.\footnote{See Appendix 3, The Open Source Definition and \url{http://opensource.org/}}Open-source can be read as simply a technology development methodology that co-opts the advantages of free-software development - principally freedoms 1 and 3, open access to the source-code -  while playing down its freedom principle. Open-source can also be described as a social movement. The `cause' of this social movement is the adoption of open-source 
practices; its strategy is the neutralisation of free-software, `recalibrating the broad meaning of freedom' as put forward by Stallman in the FSD, in order to become more business-friendly. There is a third reading of open-source which shows its political implications. Opening the source code requires an attitude of openness towards modification, duplication and redistribution. Open-source, therefore, becomes a metaphor for transparency, for accountability, for democracy; a point to which I will return. Openness necessitates and configures new social relationships and creates new political spaces.

\paragraph{}Building on the central motifs of the networked society, the information economy and contemporary organisational theory, Raymond muses on the phenomenology of open-source in a number of essays, collectively published as \textit{The Cathedral and the Bazaar}.\footnote{\cite{raymond:1999catb}} He notes a shift from hierarchical models of organisational, that he refers to as `the Cathedral', to a more ad-hoc horizontal organisation model that he calls `the Bazaar'. The Cathedral model symbolises the paradigm of industrial production, the competing industrial giants of the 20th century, and the Bazaar model symbolises the paradigm of open-source production in a networked society. There are other parallels that can be drawn from his observation. The centralised and hierarchical Cathedral model can be viewed as analogous to the waining structures of modern government that are being destabilised and deterritorialised by the rise of an open-source networked society paradigm, Raymonds' rhizomatic Bazaar model. 

\paragraph{}Openness conditions new social relationships and 
practices with new technologies in a new political space, the networked society of the Bazaar. For Anderson the dynamic co-development of human societies and technologies conditions the social order.\footnote{\cite{Anderson:1991ic}} The `nation' is an imagined community manifested in `states'; the emergence of this political category in the 19th century is closely related to the spread of technologies that increased inter-personal communication in vernacular languages. The global networks that have extended and accelerated communications in our contemporary world have facilitated the multiplication of imagined communities which now scale globally. The idea of `community' is frequently evoked in f/los discourse and most f/los activity takes place within a group, with disparate and dispersed members. Hacker communities are the archetype of a networked community. The role of `the community' in maintaining and distributing knowledge and in inculcating a good f/los civic values is central to the hacker ethos. 

\paragraph{}I contend that the networked communities and the political cultures they condition provide useful examples for positive changes to our social order and 
practices. Many different New Social Movements have co-opted the symbols and 
practices of free-software as well as deriving benefits from the free software produced by hackers. An example of f/los in transmission to social-political activism, which is frequently presented, is IndyMedia.org. This is `` a collective of independent media organisations and hundreds of journalists offering grassroots, non-corporate coverage. IndyMedia is a democratic media outlet for the creation of radical, accurate, and passionate tellings of truth.''\footnote{Quoted from IndyMedia.org homepage, accessed 24th April 2008, \url{http://www.indymedia.org/en/index.shtml}. As mentioned by \cite{Prug:2007fs}, \cite{Boomen:2005uq}, \cite{coleman:2004fo}} The openness of open-source fits with the radical transparency of this network of movements and the free/libre software produced by hackers sits easily with the broadly anti-capital stance of IndyMedia.

\paragraph{}There are other more mainstream examples of the implementation of open-source methods at the organisational level, the use of free/libre software tools at a functional level and the general adoption of the principles of collectivism, active and open participation, and consensus decision-making and sharing through the organisation. The emergence of `social source' may be defined narrowly as the ``marrying [of] open source software development with social service and social change applications''\footnote{\cite{rosenblatt:2005mn}}, or, more, broadly as the marrying of open-source principles and free/libre production and distribution with the express goal of social change. The `civic hacking' of the mySociety organisation embodies in every way the positive socio-political potential of f/los, building technological tools to open up governmental activity and reinvigorate democracy.\footnote{See mySociety.org, \textit{What's it all about then, eh? - mySociety Frequently Asked Questions.} \url{http://www.mysociety.org/faq}, accessed 24th April 2008.}

\paragraph{}So, in summary, I have sketched the transition from hierarchy to rhizome, from the monopolistic monolithic structures of the industrial age to the distributed, diverse peer-nodes of networked society. There are further questions to ask: of a future f/los society of individuals participating in communities, producing and socially sharing with each other? Rhizomatic democratic networks overcoming the tyranny of both structure and structurelessness? What of the tendency towards transnational cosmopolitan `governance without government'? Before returning to the challenges for the contemporary political order, I will discuss the hack of the dominant economic system presented by free/libre philosophy.\newpage
\chapter{Hacking Communicative Capitalism.}
\paragraph{}In the previous sections I considered the networked society paradigm, the philosophy of free/libre and the practices of the open-source movements. I now turn to explore how a free/libre political economy of social production and reproduction might challenge the dominant liberal `free-market' political economy of late-stage capitalist market economies.

\paragraph{}The dynamic between free-software and open-source is often misunderstood by social and political theorists. As a consequence it is also under-theorised within socio-political theory. For example Terranova's \textit{Network Culture: Politics for the Information Age}\footnote{\cite{Terranova:2004ly}} has but a few notes on  `Software: freeware \& open-source' but none on free-software. This is a disappointing omission because it is an opposition to, indeed, the political act of resistance to, enclosure that created the free-software philosophy in our information age. As Prug observes in his examination of the `political act' of free/libre activity, Hardt and Negri manage two references to the open-source movement in \textit{Multitude}.\footnote{\cite[`Hackers and the Protestant ethics']{Prug:2007fs} and \cite[p. 301 \& pp. 339-40]{Hardt:2005zt}} Terranova and Hardt \& Negri, perhaps willingly, omit the origins of open-source in free-software and as a result miss the challenge to capitalist property rights that free-software presents. To address these failings, in this section I consider questions of production and reproduction under the conditions of `communicative capitalism'\footnote{After \cite{dean:2005cc}, \textit{The Networked Empire: Communicative Capitalism and the Hope for Politics}} in search of `positive possibilities'\footnote{After \cite{Hardie:2005px}, \textit{Change of the Century: Free Software and the Positive Possibility}} presented by free/libre and commons conceptions of the same.

\paragraph{}In \textit{Empire} Hardt and Negri, like Marx before them, demonstrate that capital has an expansive character. Just as the hacker sees no boundary to what may be hacked, capital similarly has no boundaries, expanding into culture, science, and sex. The expansion and increasing sophistication of production possibilities, created from nature by hacking, necessarily increases the sophistication of the capitalist system. The contemporary economic paradigm has been variously described as a digital economy, an information economy and a weightless economy. Yet I think that Dean's concept `communicative capitalism' most accurately describes the complexity of this conjuncture, by properly retaining a reference to `captialism'!

\paragraph{}Dean's term highlights the economic reconfiguration of communication, referring to the networked movement of information and the production of affect, in a technologically mediated society.\footnote{\cite[pp. 272-273]{dean:2005cc}} In conceiving of capitalism as communicative in this way, Dean brings to the foreground the social and technical aspects of informational capitalism in networked society. By social I am referring to the affective dimension of production and reproduction, taking a biological turn similar to Hardt and Negri in their use of Foucault's concept of biopower.\footnote{\cite[p. 28]{Hardt:2001jl}} The `biological turn' can thought of as a ``techno-scientific reconceptualisation of life''\footnote{\cite[p. 101]{Terranova:2004ly}} away from the hierarchical structures and the centralisation of power of modernity's oligopolistic Cathedral model of organisation and production, towards more ad-hoc, rhizomatic network conceptions, such as the Bazaar.

\paragraph{}By technology, like Appadurai, I am referring to technology high and low, both physical and informational technologies. Similarly, Barry demonstrates that global regimes, such as ISO certifications and international `intellectual property' regulation are equally technological as more conventional understandings of technology, such as equipment and scientific knowledge.\footnote{\cite[p. 34]{Appadurai:1996lp}. See also \cite[Chapter 2, \textit{Technological Zones}]{Barry:2001ff} for a discussion of the `content' of Appadurai's technoscapes} Dean's term `communicative capitalism' reflects this understanding of technology. It reveals the multiple sites of contestation and diverse technologies of mediation in ``rhizomatic communications networks that are themselves biopolitical, generative, productive, of capital, of subjectivity, of life itself.''\footnote{\cite[p. 285]{dean:2005cc}}

\paragraph{}Crucial to my argument here are the concepts of abstract labour and of immaterial production. These post-workerist terms move beyond the over-determined political subject of the industrial proletariat and describe the biopolitical nature of production and re-production under communicative capitalism.\footnote{See \cite[Chapter 3 \textit{Free Labour}]{Terranova:2004ly} and also \cite{Wright:2005wb}} Using these categories, we can see that all collective labour is channelled and structured within the logic of capital.

\paragraph{}The informatisation of cultural labour and production, and of that information then becoming property under the conditions of communicative capitalism, tends towards the increasing indistinguishability of economic and cultural phenomena. The publishing of creative works as commodities and the digital distribution of these works blurs the markers of the industrial production paradigm. From abstract labour and immaterial production similarly abstract property forms emerge, such as `intellectual property' and patents, accompanied by copyright regimes to protect these new property forms. With the modern site of rights, and of rights protection, diminished in our networked society the defence of the rights of the holders of these new property forms becomes increasingly problematic. The battle for protection of these abstract property forms becomes global, with communicative capitalism accelerating the expansion and normalisation of transnational structures of `governance without governing', such as the World Trade Organisation (WTO), the World Intellectual Property Organisation (WIPO) and the Organisation Internationale de Normalisation (ISO).\footnote{The International Organization for Standardization is non-governmental transnational organisation that works to globally normalise technical standards.}

\paragraph{}The WIPO understand `intellectual property' as referring to ``creations of the mind: inventions, literary and artistic works, and symbols, names, images, and designs used in commerce.''\footnote{WIPO website, \textit{What Is Intellectual Property}, \url{http://www.wipo.int/about-ip/en/}, Accessed 26th April 2008.} For Stallman and the FSF, ``the term ``intellectual property'' carries a hidden assumption - that the way to think about all these disparate issues [copyright, patents, trademarks] is based on an analogy with physical objects, and our ideas of physical property.''\footnote{\cite{Stallman:2008wa}} Johnson is critical of the FSF position, insisting that the ``word `intellectual' is used as a prefix to `intellectual property' precisely to distinguish it from material or real property.''\footnote{\cite{Johnson:2004wa}} This definitional argument doesn't necessarily require resolution, as it is serves me here just to highlight the problematic relationship between culture and property in advanced capitalist societies. If culture is understood as art, science, code, the radical diversity and interdependence of all human knowledge, then the enclosure of knowledge, its commodification as exchangeable property, deprives humans of access to it.

\paragraph{}The hegemony of liberal political economy models in the realms of technology and culture is demonstrated by the spread of international patent protection and `intellectual property' law. The radicalism of free/libre concepts in transmission to non-software realms, particularly to virtual and abstract labour, but also to cultural production in general, have created the revolutionary abstractions of copyleft, a hack of contemporary capitalism's notion of copyright. It has also created creative-commons, a hack which institutionalises a opening dynamic of sharing against the closure of patents and licensing.\footnote{\cite[17m32sec]{moore:2002rv}} Copyleft and the creative-commons subvert the regimes of copyright and `intellectual property', reusing the main symbols of the latter pair of concepts to make public a new logic.\footnote{See The GNU Project, \textit{What is Copyleft?}, \url{http://www.gnu.org/copyleft/copyleft.html} and Creative Commons, \textit{Frequently Asked Questions}, \url{http://wiki.creativecommons.org/FAQ}. Both accessed 27th April 2008}

\paragraph{}The free/libre hacks of copy-left and creative-commons are based on an appeal to the rule of law, demonstrated by the constitutional nature of the Free Software Definition and the codification of expressions of the FSD in various licenses.\footnote{See \cite[Chapter 1, \textit{Open Source Licensing, Contract, and Copyright Law}]{Laurent:2004}} The appeal to the rule of law of much of free/libre discourse demonstrates the neo-liberal character of free/libre. How does this mean we should understand free/libre's first principle, freedom? For Hardie, the freedom espoused by the FSF in teh FSD is a freedom ``bound intimately with the logic of open democracy and with free and open markets.''\footnote{\cite[p. 2]{Hardie:2005px}} On this reading, the political economy of free/libre, is subversive only because it appears counter-intuitive. That is to say it appears paradoxical within the axioms of the capitalist market to produce only to then give away what is produced.

\paragraph{}Prug's work proposes a broader reading of free/libre philosophy as radical hack of the regime of rights in general. His approach demonstrates a true positive possibility in the free/libre hack of communicative capitalism. By hacking rights in favour of egalitarianism over meritocracy, the property rights central to liberal capitalism are changed from the substantive `right to' and the normative `right of' property to a `right in' property, which is both substantive and normative. As Prug argues, ``Stallman re-conceptualised the idea of rights to encourage [a] volunteer, co-operative and decommodified society with the notion of shared wealth.''\footnote{\cite{Prug:2007fs}}

\paragraph{}Influenced by the openness of free/libre philosophy the idea of Open Capital\footnote{See \url{http://www.opencapital.net/}} re-conceptualises other logics central to communicative capitalism in a radical hack of the terms `profit' and `loss'. In a departure from the ``competitive economy based upon shareholder value and unsustainable growth results from a transfer of risks outwards, and the transfer of reward inwards,''\footnote{\cite[p. 17]{cook:2004cp}} Open Capital neutralises the conflict created by that centrifugal/centripetal strategy, by hacking `profit' and `loss' to `reward' and `risk', as defined by all participants in the network, to result in a mutually satisfactory exchange of value from the productive activity.

\paragraph{}In conclusion, free/libre presents opportunities for reconfiguring the logic(s) of communicative capitalism. Now I return to the political question of organisation and resistance, of the wider translation of the free/libre hacks of political categories and concepts.
\newpage
\chapter{Hacking a Free/Open Politics.}
\paragraph{}The ``contemporary proliferation of political spaces and the multiplicity of democratic demands''\footnote{\cite[p.17]{Mouffe:2000fk}} in our networked society needs a new politics, a politics that works at the level of the global and the local, the universal and the particular. New relationships of interconnectedness between people, communities, identities, artifacts, locations and events need to be reflected in the political culture of our institutions. What positive benefits can free/libre and open-source in transmission to our wider society bring to politics? How are we to move beyond the individualist ontology of liberalism and a politics dependant on capital? What hope is there for a new ontology of interdependence, a way of being among?

\paragraph{}I have outlined changes in political culture that the adoption of open-source principles of collectivism, participation and consensus decision-making by organisations and institutions have produced. I have also considered the reconfiguration of economic activity through the liberating free/libre hack on property rights. In this section I will bring these threads together, imagining a free/libre open-source society which shares, and shares widely, the wealth of networks, grows open communities and strengthens democracy. No longer can free/libre and open-source be dismissed as ‘merely metaphorical’ or utopian. As van~den Boomen \& Sch{\"a}fer point out, ``there is more at stake than just a vague metaphor for a transparent, democratic and non-private constitution. Of course, notions of ‘freedom’ and ‘openness’ appeal strongly to the social imagination, and this can easily result in utopian daydreaming.''\footnote{\cite[p. 7]{Boomen:2005uq}} Let us recall Appadurai here, reminding us that the imagination is a necessary part of any innovation, that `imagination is a social practice', both a form of work, and a form of negotiation between sites of agency and globally defined fields of possibility.\footnote{\cite[p. 31]{Appadurai:1996lp}}

\paragraph{}I argue that the relational and constructive ontology of the hacker and free/libre and open source ideas are beginning to displace the dominant liberal \textit{homo economicus}, the rational utility-maximising consumer of communicative capitalism and the protestant work ethic associated with Weber.\footnote{See \cite[Ch. 1 \& 2]{himanen:2001he} for a full discussion on the hacker ethic as contra the Protestant Work Ethic associated with \cite{weber:1958}.} This reemerging \textit{homo socialis} recognises his/her historical and social situatedness, both of  which \textit{homo economicus} is emphatically encouraged not to reflect upon by the propagandistic and manipulative efforts of communicative capitalism's individualistic and ahistorical modes of thought.

\paragraph{}The network society deconstructs modernity's institutions and constructs new institutions that reflect the dynamics of new power relationships. Free/libre is a radical hack of the institution of rights, disrupting and reconfiguring the power relationships of property and demanding new politics relevant to the networked society. I'm not going to quibble with the liberal basis of much of the hacker ontology and the free/libre political philosophy. I'm all for free, autonomous individuals solving their own problems in relationship to their material conditions, adding to the sum of human knowledge. It is the logic of capital that reconfigures this knowledge as property and wields property as power. A politics for the networked society need not reject the symbols of liberal economics, and certainly must not reject the values of democracy. Yet it must move beyond the structuralist epistemology of the nation-state as the container of politics.

\paragraph{}In \textit{Multitude}, Hardt and Nergi outline four interpretations of either a challenge or contribution to democracy from the globalisation of capital.\footnote{\cite[pp. 232-236]{Hardt:2005zt}} On the `left' of the political spectrum, the social democrat sees global capital as an obstacle to democracy and in response strengthens the democracy of a people through the nation-state and uses this site to regulate capital. The liberal cosmopolitan perspective recognises a contribution to democracy from globalisation and in response strengthens the democracy of the people through global citizenship. On the `right', the neo-liberal response favours the globalisation of capital, viewing this as inherently democratic. While the `traditional values conservative' response, similar to that of the social democrat, seeks to retreat behind national borders. I agree with Hardt \& Negri that ``none of these arguments [\ldots] seem sufficient for confronting the question of democracy and globalisation.''\footnote{\textit{Ibid.,} p. 236} They place their political project, the global democracy of the multitude, outside this categorisation.

\paragraph{}Capitalism's crucial contingency is the principle of liberty, which is commonly understood as the protection of private property through regimes of `property rights'. The anti-globalisation stances of the social democrat and the `traditional values conservative' each uphold the regime of property rights against global democracy by strengthening the state / capital relationship. While the pro-globalisation stances of our cosmopolitan liberal and neo-liberal again uphold the regime of property rights, this time using global democracy to form a supra-state / capital relationship.

\paragraph{}Open-source is firmly within Hardt \& Negri's categories as markedly liberal in terms of its pro-globalisation and pro-democracy stance. Indeed open-source in and of itself is not anti-capital. Free/libre on the other hand is outside these categorisations: it is radical. It is anti-capital because it hacks (subverts) the dominant property rights regime of capitalism. It is pro-globalisation and pro-democracy in as much as it is radically free, open and egalitarian.

\paragraph{}It is certainly true that much of free/libre and open-source discourse, like NSM discourse, ``perpetuate[s] narrow understandings of the economic and the cultural''\footnote{\cite[p. 275]{Hardt:2001jl}} by failing to take adequately  consider the increased mixing of economic and cultural phenomena. To a certain degree, free/libre, open-source and creative commons concepts are all ``various forms of social contestation and experimentation, all centred on a refusal to value the kind of fixed program of material production.'' On the other hand, recalling my earlier argument for a reading of open-source as a social movement to further the adoption of open-source; free/libre, open-source and creative commons have all `sold out' by voluntarily channelling and structuring themselves within the logic of capitalist exchange.\footnote{See \cite[p. 80]{Terranova:2004ly}}

\paragraph{}The challenge for free/libre in transmission in a networked society is two fold. Firstly to recognise that these ideas are more than ``\textit{`merely cultural' experimentation}'' in isolation and that ideas have a ``\textit{very profound political and economic effect}.''\footnote{\cite[p. 274, original emphasis]{Hardt:2001jl}} The second challenge for these ideas is to recognise and guard against the counterrevolutionary responses to the effects that they have. Against the free-software revolution came the counterrevolution of open-source. So to some extent open-source can be considered a powerful pacific weapon, neutralising the anti-capitalism of free/libre. Open-source networked production and the openness that that entails has the appearance of transforming the economic relationships of late capitalism. However in reality no such transformation takes place as the fundamental logic(s) of communicative capitalism remain intact.

\paragraph{}The undecidable nature of the relationships between the hacker, who produces abstractions, and global capital, which extracts value from these abstractions is revealed in the cyclical adaption and cooption, of a pattern of deconstruction and construction, between free/libre and open-source and communicative capitalism. The positive possibility of free/libre needs defending from the unremitting attempts at neutralisation by neo-liberal and liberal cosmopolitan aristocrats. What the `liberal aristocrats', or as \v{Z}i\v{z}ek would call them, the `liberal communists'\footnote{\cite{Zizek:2006vile}} who perverted free/libre into open-source ``do not understand is that in the era of biopolitical production, liberalism and liberty based on the virtue of the few or even the many is becoming impossible.''\footnote{\cite[p. 236]{Hardt:2005zt}} Hardt and Negri continue, arguing that ``the logic of private property is being threatened by the social nature of biopolitical production.''\footnote{\textit{Ibid.,} p. 236} The transformative hack (subversion) of private property rights by free/libre philosophy recognises the social nature of production, a point sadly missed by Hardt and Negri.

\paragraph{}The liberal communists have vehemently adhered to the Weberian spirit of capitalism piously accumulating wealth by the sleight of hand of open-sourced enclosure. True to the Weberian logic, the liberal communists demonstrate their virtue through a directed distribution of this wealth. This `open philanthropy' is not a hack like `open capital', rather it is `philanthrocapitalism', an appropriation of the symbols of open-source to extract private profit first and to `do good' second. So, while entering into `tactical alliances' with the liberal communists may be necessary in order to win local, specific, `subjective' gains, as \v{Z}i\v{z}ek reminds us, they are the ``agents of the structural violence'' of communicative capitalism: ``liberal communists are \textit{the} enemy of every true progressive struggle today.''\footnote{\cite{Zizek:2006vile}, original emphasis}

\paragraph{}Our re-emerging \textit{homo socialis} is able to overcome both the tyranny of structure and of structurelessness by adopting the radical politics of openness in social practices and institutions.\footnote{\cite{Freeman:1970}} The monolithic structures of corporate bodies such as business, state agencies, and social movements are restructured by openness. The radical transparency of openness reveals the critical interdependency of our social relationships. The adoption of openness for an ``open-source society'' makes possible a deep public scrutiny and democratic accountability, but, most crucially it encourages collaboration in new, better social programmes.\footnote{\cite[pp. 339-40]{Hardt:2005zt}}
\skip
\paragraph{}The new politics configured by free/libre philosophy and the practice of openness are an integral part of the ecology of the pro-democracy, pro-globalisation and anti-capital networked social justice movements, described here by David Graber:
\begin{quote}``This is a movement about reinventing democracy. It is not opposed to organisation. It is about creating new forms of organisation. It is not lacking in ideology. Those new forms of organisation are its ideology. It is about creating and enacting horizontal networks instead of top-down structures like states, parties or corporations; networks based on principles of decentralised, non-hierarchical consensus democracy. Ultimately, it aspires to be much more than that, because ultimately it aspires to reinvent daily life as whole. But unlike many other forms of radicalism, it has first organised itself in the political sphere - mainly because this was a territory that the powers that be (who have shifted all their heavy artillery into the economic) have largely abandoned.''\footnote{\cite{Graeber:2002}}
\end{quote}

\paragraph{}The free/libre hack of neo-liberal property rights presents the possibility to re-populate the abandoned political sphere with a politics of self-determinisation and of being in common, `reinventing daily life as a whole'. The radical principles of openness present the possibility of a lightly structured networked society, of self-governing individuals, self-organising the movement of movements.\newpage
\singlespacing
\cleardoublepage
\addcontentsline{toc}{chapter}{Bibliography}
\bibliography{/home/robd/projects/robdyke/goldsmiths/bibliography/globalbib}
\begin{appendices}
\section{Notes on the production of this text}
\paragraph{}My dissertation was produced using only free-software and open-source tools. The text was written, typeset and printed using the LaTeX document markup language. It is printed in 12pt Helvetica. My footnotes and bibliography are formatted using the Harvard/DCU LaTeX template.
\begin{description}
\item[Fedora GNU/Linux]{A distribution of the GNU Linux operating system, packaged by the Fedora Project.\newline\url{http://www.fedoraproject.org}}
\item[Freemind]{A free-software open-source mind mapping tool. \newline\url{http://freemind.sourceforge.net/}}
\item[LaTeX]{A document preparation system, released as free software. \newline\url{http://www.latex-project.org/}}
\item[TeX Maker]{A free-software LaTeX editor, released under GNU GPL. \newline\url{http://www.xm1math.net/texmaker/index.html}}
\item[JabRef]{An open-source bibliography reference manager, creating BibTeX files for use with LaTeX. \newline\url{http://jabref.sourceforge.net/}}
\item[BibTeX]{A tool and a file format which are used to describe and process lists of references in conjunction with LaTeX documents. \newline\url{http://www.bibtex.org/}}
\item[Subversion]{A free-software open-source distributed version control system for collaboration in a networked environment. \newline\url{http://subversion.tigris.org/}}
\end{description}
\newpage
\section{Glossary of Acronyms}
\begin{description}
\item[ANT]{Actor Network Theory. A network sociology associated with Bruno Latour. See \textit{On Actor Network Theory: A few clarifications} Pt.1 (\cite{latour:1998ant1}) \& Pt.2 (\cite{latour:1998ant2})}
\item[FSD]{Free Software Definition. ``Free software is a matter of liberty, not price. To understand the concept, you should think of free as in free speech, not as in free beer. Free software is a matter of the users' freedom to run, copy, distribute, study, change and improve the software.''  \newline\url{http://www.gnu.org/philosophy/free-sw.html}}
\item[FSF]{Free Software Foundation promotes computer user freedom and defends the rights of all free software users.  \newline\url{http://www.fsf.org/}}
\item[GNU]{The GNU Project, launched in 1984, develops free software. The name “GNU” is a recursive acronym for “GNU's Not Unix”; it is pronounced g-noo, as one syllable with no vowel sound between the g and the n.  \newline\url{http://www.gnu.org/}}
\item[GPL]{General Public License. A codified expression of the Free Software Definition, maintained by the GNU Project.  \newline\url{http://www.gnu.org/licenses/licenses.html}}
\item[ISO]{International Organization for Standardization is a global network, developing and publishing technical specifications. \newline\url{http://www.iso.org}}
\item[NGO]{Non-Government Organisation, used as short-hand for independent civil society and grassroots organisations, private voluntary organisations, self-help organisations\linebreak and other non-state actors.}
\item[NSM]{New Social Movement, used as shorthand for the great diversity of groups in co-ordinated social change networks.}
\item[WIPO]{World Intellectual Property Organisation. A specialised agency of the United Nations, ``dedicated to developing a balanced and accessible international intellectual property (IP) system, which rewards creativity, stimulates innovation and contributes to economic development while safeguarding the public interest.''  \newline\url{http://www.wipo.int}}
\item[WTO]{The World Trade Organisation is ``the only global international organisation dealing with the rules of trade between nations.''\newline\url{http://www.wto.org}}
\end{description}
\newpage
\section{The Open Source Definition}
\paragraph{}\url{http://opensource.org/docs/osd}
\paragraph{}Open source doesn't just mean access to the source code. The distribution terms of open-source software must comply with the following criteria:
\begin{description}
\item[Free Redistribution]{The license shall not restrict any party from selling or giving away the software as a component of an aggregate software distribution containing programs from several different sources. The license shall not require a royalty or other fee for such sale.}
\item[Source Code]{The program must include source code, and\linebreak must allow distribution in source code as well as compiled form. Where some form of a product is not distributed with source code, there must be a well-publicized means of obtaining the source code for no more than a reasonable reproduction cost preferably, downloading via the Internet without charge. The source code must be the preferred form in which a programmer would modify the program. Deliberately obfuscated source code is not allowed. Intermediate forms such as the output of a preprocessor or translator are not allowed.}
\item[Derived Works]{The license must allow modifications and derived works, and must allow them to be distributed under the same terms as the license of the original software.}
\item[Integrity of The Author's Source Code]{The license may restrict source code from being distributed in modified form only if the license allows the distribution of "patch files" with the source code for the purpose of modifying the program at build time. The license must explicitly permit distribution of software built from modified source code. The license may require derived works to carry a different name or version number from the original software.}
\item[No Discrimination Against Persons or Groups]{The license\linebreak must not discriminate against any person or group of persons.}
\item[No Discrimination Against Fields of Endeavor]{The license\linebreak must not restrict anyone from making use of the program in a specific field of endeavor. For example, it may not restrict the program from being used in a business, or from being used for genetic research.}
\item[Distribution of License]{The rights attached to the program\linebreak must apply to all to whom the program is redistributed without the need for execution of an additional license by those parties.}
\item[License Must Not Be Specific to a Product]{The rights\linebreak attached to the program must not depend on the program's being part of a particular software distribution. If the program is extracted from that distribution and used or distributed within the terms of the program's license, all parties to whom the program is redistributed should have the same rights as those that are granted in conjunction with the original software distribution.}
\item[License Must Not Restrict Other Software]{The license must not place restrictions on other software that is distributed along with the licensed software. For example, the license must not insist that all other programs distributed on the same medium must be open-source software.}
\item[License Must Be Technology-Neutral]{No provision of the license may be predicated on any individual technology or style of interface.}
\end{description}
\newpage
\section{Creative Commons License}
\paragraph{}Attribution-Noncommercial-Share Alike 3.0 Unported\linebreak\url{http://creativecommons.org/licenses/by-nc-sa/3.0/}
\paragraph{You are free:}
\begin{description}
\item[to Share]{-- to copy, distribute and transmit the work}
\item[to Remix]{-- to adapt the work}
\end{description}
\paragraph{Under the following conditions:}
\begin{description}
\item[Attribution.]{You must attribute the work in the manner specified by the author or licensor (but not in any way that suggests that they endorse you or your use of the work).}
\item[Noncommercial.]{You may not use this work for commercial purposes.}
\item[Share Alike.]{If you alter, transform, or build upon this work, you may distribute the resulting work only under the same or similar license to this one.}
\end{description}
\paragraph{}For any reuse or distribution, you must make clear to others the license terms of this work.\linebreak The best way to do this is with a link to this web page: \url{http://creativecommons.org/licenses/by-nc-sa/3.0/} \linebreak Any of the above conditions can be waived if you get permission from the copyright holder.\linebreak Nothing in this license impairs or restricts the author's moral rights.

\end{appendices}
\end{document}
