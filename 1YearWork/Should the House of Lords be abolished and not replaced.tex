Should the House of Lords be abolished and not replaced?


�Should the House of Lords be abolish and not replaced?� I feel as though the question posed could incite me to attempt a polemic tract, such is the strength of my feeling on this area of British constitutional reform. However, I rather think that dialogue is what is called for given the complexity of the issue and the variety of opinion of the subject. To add to the extensive discussion on this topic I shall offer a brief history of the House of Lords and outline its constitutional function in order to give an understanding of the current situation. I will also give a review of arguments and points of view on its future direction.

I will begin with a short introduction to parliament. The parliament of the UK as we know it today finds its genesis in the 1688 �Glorious Revolution� which established the constitutional authorities and legitimacies of the Crown, the Lords Temporal and Lords Spiritual, and the Commons in relation to the governance of the people. The historic power of the House of Lords has withered with the ascendance of the House of Commons. Parliamentary reform since the eighteenth century has shifted the balance of power away from a Monarch, supported by landed aristocracy, towards �The People�. 

The House of Lords is the �Upper House� of the bicameral parliament that governs the United Kingdom, the House of Commons the �Lower House�. The House of Lords is almost alone among chambers of western parliaments in that its members are not democratically selected, a legacy of its heritage. The traditional majority of its members, peers, were there by accident of birth having inherited a peerage, with the remaining peers sitting by patronage of government or the Crown. 

The current Labour Government has modified the traditional composition of the Lords since coming to power in 1997. Recent research reports published by the House of Lords give interesting insight to its current membership. Membership of the Lords is not fixed to a constituency like the Commons, and numbers around 700. Of this, 25 are Bishops, representatives of the Church of England (the Lords Spiritual). 27 are Law Lords; a further 92 are hereditary peers. The remaining 537 or so members are appointed �Life Peers� (HofLwork:20). Appointed peers have usually received patronage from the government in recognition of expertise or achievement in a specific area. 

What is the purpose of the House of Lords and how do its functions fit into the process of government?  Common in the literature on this topic are the following specific functions of the House of Lords: legislative, scrutiny, legitimating and judicial (see for example Norton, 2004 and Roberts et al, 1999). An understanding as to the scope and importance of each of these functions will needed before a pronouncement on the question can be given. I will first consider the legislative function of the Lords. The two chambers of Parliament work together as part of the legislative process. Both Houses can initiate legislation. However it is the House of Commons, the elected representatives of the people that initiates the majority of legislation. Legislation passed by the Commons goes to the Lords, who are able to make amendments and ultimately delay bills. The ability of the Lords to delay bills is limited by the Parliament Acts of 1911 and 1949. 

The House of Lords, in common with other second chambers of parliaments, provides scrutiny for legislation before it reaches the statute book. Norton (2004:437-8) gives three reasons for the suitability of the House of Lords to this task. Firstly, as peers are not elected, they cannot reject a bill because of its principle substance. Second, the professional experience of some peers gives the benefit of �perspective [from] practitioners in the field� (Ibid). Thirdly, the Lords have more time to devote to detail than the Commons, ensuring internal coherence of legislation.

It would be difficult for citizens of nation-states governed by an elected representative bicameral legislature to understand how the House of Lords can proffer legitimacy on Acts of Parliament; the �pre-eminence� (HofLBpHistory:1) of the House of Commons, its members selected by popular elections, having been established during the seventeenth century. However, for legislation to pass into statute, deliberation and revision by the House of Lords is the constitutional tradition, lending latent legitimacy. The House of Lords no longer draws on its historic position of power, its manifest legitimacy is now perceived to be in the experience and expertise of peers. (Norton, 2004:436).

The judicial work of the House of Lords is carried out by the Lords of Appeal Ordinary sitting as the Appellate Committee. This makes the House the highest court of appeal for civil and criminal cases in the United Kingdom, bar criminal cases from Scotland. The Lord Chancellor, who is responsible for the appointment of Crown judges, is a member of the Appellate Committee. 

As I have illustrated, the House of Lords is integral to the constitutional fabric of the United Kingdom. Its legislative role and duty of scrutiny exist as a constitutional check on the government of the day. Whether to retain, reform or abolish the House of Lords is a difficult question. The only aspect politicians appear able to agree upon is that the Lords cannot continue in its traditional form, i.e. wholly unelected. Modern western democracy finds its virtue in being representative; reconciling this with hereditary and appointed peers is problematic. Taking either a reformist or abolitionist view results in the arrival at the same question: where do the existing functions of the Lords go?

The need for the functions performed by the House of Lords will not disappear. Scrutiny of legislation is a vital function and the delay caused by deliberation in a second chamber allows for additional points to be heard and considered before a bill becomes law. A reformed Lords would be able to continue to perform this valuable work. But the abolition of the second chamber would mean that this workload would fall on the Commons. For me, this raises questions raised about the compound of legislative power in a single chamber. Currently the House of Lords provides a constitutional safe-guard against an elected dictatorship establishing itself, as the Lords retains a veto over any legislation prolonging the life of a parliament. To whom would this responsibility fall if the Lords were abolished?

Separation of power between legislature and judiciary should result from either reform or abolition of the House of Lords. The appointed Law Lords are currently part of the legislature because they sit as peers. The current Labour Government has proposed the establishing of a supreme court and the transfer of the judicial functions of the Lords to this body. But as Heffer (2004:457) observes, the proposed constitution of the European Union will �formally subjugate British law to European law�. Should the present Labour government ratify the constitutional change at the EU, the function of the Lords as the highest court of appeal will be superseded by, for example, the European Courts of Justice and of Human Rights.

I shall chronologically outline some reform measures since 1900 before moving on to consider the reforms of the current Labour government. The erosion of the constitutional position of House of Lords began with the Parliament Act of 1911 that first restrained the legislative role of the House. This Act made provision for Money Bills, which were to pass unhindered through the Lords, as well as legislation repeatedly passed by Commons vote. The act was seen by some as a �temporary measure� (HofLBpReform:2), the beginning of further reform, but the following 38 years were of little consequence.

Further piecemeal reform attempts were made during the late 1940s, which became law under the Parliament Act 1949. Focused on reducing the powers of the Lords, the act decreased the period the House could delay legislation from the Commons. This act, originating during a period of Labour government, also sought �that there should not be a permanent majority assured for any one political party� (Ibid:4). Women were able to become Lords of Parliament under this legislation.

During the 1950s, the Conservative government made no radical attempts at Lords reform. Life Peerages were created in 1958, giving the government scope for patronage as never before. Peers created by government patronage and hereditary peers gave the Conservative government a massive majority in the Lords. Radical reform returned to the agenda during the late 1960s under a Labour government. A bicameral legislature was still favored, although its hereditary membership would be abolished. These attempts at reform did not pass a Commons vote and were abandoned. 

The mid-1970s saw a resurgence of political will within the Labour Party to tackle constitutional reform. A part hereditary, part appointed second chamber was viewed as an outdated part of government that had no place in a representative democracy. However, a wholly appointed second chamber represented what was viewed in some quarters as a deplorable use of patronage and yet an elected second chamber would constraint the pre-eminence of the House of Commons. �Abolition was therefore preferable to reform� (Ibid:7). But with a change in government in 1979, the momentum was lost once again. Lords reform was off the political agenda for the Conservatives, their majority in both Houses of Parliament secure.

From the chronology outlined, reform of the House of Lords appears to be party centric. Radical reform of the House of Lords has long been a socialist concern. The Labour Party, when in power and when in opposition, has maintained its opposition to the �Upper Chamber�. For Bell (1981:1) the reason for this was obvious: the House of Lords was dominated by the �massed forces of Tory peers, unelected, unrepresentative�, frustrating the social programs of the Party.

Upon its return to government in 1997, the Labour Party was swift to act on its manifesto program of reform. The party had shed its abolitionist stance and now favored reform, firstly of Lords membership and secondly of the Houses powers. Its manifesto pledge to remove the hereditary rights of peers to sit and vote in the Lords was not as complete as envisaged when finally implemented; the Parliament Act 1999 allowed 92 hereditary peers to retain their seats � the elected peers.

The beat of the reforming drum continued the modernising march. Those who favor retaining the House of Lords will find turning back the eviction of hereditary peers an insurmountable challenge. Also, abolitionist sentiment has waned, at least within government. The development of a different type of bicameral legislature appears to be the goal. Current debate continues as to the specifics of the second stage of Lords reform.

With the removal of hereditary peers from the Lords almost complete, the debate on the future of life peerages continues. This is part of the unanswered question of what composition the second chamber should have. Life peers bring with them professional expertise. If all members of the House of Lords were to be elected then the professional expertise within the chamber could be lost. While political patronage is a potential source of power for the nepotistic, life peerage appointments have introduced a depth of experience to the second chamber. The potential for abuse of power and the loss of expertise are both addressed by the solution proposed by Charter 88 and The Institute of Public Policy Research; a majority elected, minority appointed second chamber or Senate.

The �Secondary Mandate� proposed by Billy Bragg and published by the Fabian Society is a strong solution to the current stalemate. Under this proposal the second chambers� membership would be constituted from party lists with seats allocated in proportion to votes cast at a general election. This solution would create a �genuine expression of the Will of the People� (Bragg, 2004) in the second chamber. With careful consideration given to the preservation of the pre-eminence of the House of Commons, the House of Lords reformed in this way could continue to provide the functions of scrutiny and of legitimacy is does today.

The future of the House of Lords needs to be understood as part of a wider program of constitutional reform for the United Kingdom. While the UK continues to be a unitary state, devolution has eroded power of central government from below. Furthermore, European Union membership compacts powers of central government from above. Additionally, supporters of any constitutional reform in the UK need to be mindful that there is no single codified constitution for our nation. Absolute clarification of the roles, remits and responsibilities of the institutions of governance should be an outcome of any constitutional reform.

The present Labour government has pushed ahead on its manifesto pledge of devolved power for Wales and Scotland and has begun work on establishing regional assemblies in England. Devolved power weakens the unitary structure of British government. In considering House of Lords reform along with devolved government, there is a strong argument for the continuation of a bicameral legislature. If the House of Commons were to become a parliament solely for the English with parliaments in Scotland, Wales and Northern Ireland, then the House of Lords may make a transition to a chamber representative of a �Federal United Kingdom�. The impacts of devolution and the European dimension on House of Lords reform were considered in a House of Commons White Paper Modernising Parliament: Reforming the House of Lords published in 1999.

Having reviewed the second-stage proposals of the Labour government published in the White Paper The House of Lords: Completing the Reform (DCA Nov 2001) my view is that they do not go far enough in achieving the twin goals of a representative and democratic chamber. The proposals in the White Paper support a majority appointed second chamber. The disestablishment of the Anglican Church is not included in the government�s proposals, leaving 16 Bishops in the membership of the future House of Lords. A further complaint with the second stage reforms is that provision is made for 12 Law Lords to remain part of the legislature.

Returning to the question posed: �Should the House of Lords be abolished and not replaced?� No. I firmly believe that British government needs to retain its bicameral formation as the size and complexity of our nation requires the due-diligence that a unicameral legislature may struggle to provide. Yet I do not wish to see the pre-eminence of the directly elected House of Commons negatively impacted by the creation of a wholly elected second chamber. I am in favor of radical constitutional reform and view hold that the greater need is that of an explicit codified constitution. With a codified constitution the roles and responsibilities of both the House of Commons and a reformed House of Lords will be clarified for all.
Bibliography

Bell, Stuart, (1981) How to Abolish the Lords Fabian Society, Blackrose Press

Heffer, S., in Jones, B., Kavanagh, D., Moran, M., Norton, P., eds (2004) Politics UK, Harlow, Longman

Norton, P., in Jones, B., Kavanagh, D., Moran, M., Norton, P., eds (2004) Politics UK, Harlow, Longman

Roberts, D. editor (1999) British Politics In Focus, Ormskirk, Causeway Press


Online Resources

�A Genuine Expression of the Will of the People by Billy Bragg�, The Fabian Society, reviewed November 2004.
http://www.fabian-society.org.uk/documents/searchdocument.asp?DocID=72
 
Charter 88 & Institute of Public Policy Research
http://www.charter88.org.uk/pubs/facts/enq_hol.html


Resources from Parliament

HofLwork � �The Work of the House of Lords�, January 2004, Downloaded November 2004.
http://www.parliament.uk/documents/upload/HoLwork.pdf 

HofLBpHistory � �History of the House of Lords�, March 2004, Downloaded November 2004.
http://www.parliament.uk/documents/upload/HofLBpHistory.pdf

HofLBpReform � �Reform and Proposal for Refrom since 1900�, February 2003, Downloaded November 2003.
http://www.parliament.uk/documents/upload/HofLBpReform.pdf

The House of Lords: Completing the Reform, November 2001
http://www.dca.gov.uk/constitution/holref/index.htm

