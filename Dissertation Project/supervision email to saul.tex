I am writing to ask if you will supervise my dissertation project in my third year. I have a sustained commitment to Free/Opensource projects and I am active in political hactivism towards a better understanding of these ideas and to further the adoption of these approaches. Because of this
interest for my dissertation project I propose to explore the politics of openness.

My project would examine the history and diffusion of Free/Opensource ideas with a discussion on the scientization of politics. I will consider the politics of technology looking at the concepts of 'free' and 'open',
showing their interrelationships and the hybrid models that are dependent on these models.

New definitions of the concepts of property and of ownership arise from Free/Opensource ideas. The critique of contemporary practice posed by these new definitions will be outlined and practical opportunities for the
individual, enterprise, third and state sectors will be demonstrated.

Deftly avoiding the obvious trap of techno-utopianism... I seek a Free/Opensource politics and wonder what it would look and feel like: ideas of radical transparency, open organisations, citizen participation
and citizen accountability...