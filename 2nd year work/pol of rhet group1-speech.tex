\documentclass[14pt]{article}
\usepackage{helvet}
\usepackage{setspace}
\renewcommand{\familydefault}{\sfdefault}
\usepackage{geometry}
\geometry{a4paper}

\title{Politics and Rhetoric Group Speech Task}
\author{Course PO52011A : Group 1}

\begin{document}

\maketitle
\date{}

\begin{onehalfspace}

\paragraph{}\emph{Question}: I ask the Prime Minister if he will list his official engagements for Friday 16th March.

\paragraph{}\emph{The Prime Minister}: Honorable Members of the House, this morning I had meetings with ministerial colleagues and others in pursuit of my duties to this House. Before I list my further engagements I seek leave from the Speaker to address the house.

\paragraph{}More than a decade ago my party gave me their trust and elected me as leader. To lead them in developing and delivering new policy for the new challenges of the age.

\paragraph{}A decade ago the British people, gave me their trust, to lead this nation when they elected me Prime Minister. They endorsed me and gave my party a mandate to govern. And in government, for them and with them, the British people; together, we, have led a truly modern Britain into the 21st century.

\paragraph{}Government, it is said, is about delivering what is possible, not what is best, or right, or good, just what is practically possible. That's not my view of government, filmsy and weak. Government policy is not what is possible, it is about what is made possible. Our shared principles of social democratic justice have made the policy of this government possible. Unity of values, strength in purpose. a government that makes things possible; an enabling state.

\paragraph{}Government policy is not developed from gazing in a crystal ball, nor a societies problems solved with a simple wave of a magic wand. Doing the right thing is tough. I make no apology for policy mandate endorsed by the great British people at general elections. I make no apology for meeting the challenges that we have faced with policy based on principles, endorsed by public and parliamentary majority.

\newpage
\paragraph{}I once said, perhaps foolishly, that history would be my judge. Now let me remind you what is my history.... what has been made possible in a decade of democracy, in the decade of my leadership.

\begin{itemize}
\item fairer
\item more equitable
\item just society
\item better public services
\item improvements in education and in health care
\item safer streets and stronger communities
\end{itemize}


\paragraph{}Devolved governments, strengthened local democracy and increased public accountability, greater transparency in the decision making processes. A strengthening the union between the people and their government.

\paragraph{}Support for democracy has been a crucial part in the my governments foreign policy too. My efforts supporting the Peace process in Northern Ireland and the tireless commitment shown in support of the desire of the people of Northern Ireland for the stability and security to live their lives free from fear.

\paragraph{}Our strongly supported humanitarian interventions have saved lives and have supported the establishment of democratic government in many places of the world, in Sierra Leoine, Afghanistan. Balkans. As a European community we are now welcoming young, stable, democratic governments from that previously tortured region into our political community.

\paragraph{}I've shown Britain supports democracy. We support for democracy as democracy delivers. Democracy cares. Democracy encourges peace, delivers justice and fosters harmony. Today in the world we confront dangerous forces which threaten our stability and security, forces that threaten democracy, our shared value of democracy. I feel a strong duty to promote and protect our values in the world. I have led by example.

\paragraph{}There have been some deeply divisive issues debated over the ten years I have spent as leader. The judgments I have come to, the decisions I have made, I made with the best intentions, with the full belief that the decision taken was the right thing to do.

\paragraph{}Our principles have to be defended, our principle value of democracy, our principle belief in social justice, when human rights and individual liberty are threatened, where the good needs to be protected, and when repressive forces threaten to be overcome - Britain will be strong and defend these values.

\paragraph{}My values have not changed. I do not see my party's values as having changed. While there is a cruel wind blowing, weakening the resolve of some in my own party in staying the course - I do not lead my party into a storm. My values, our values, our shared values. We should continue to dedicate ourselves to the issues that these values lead us to engage with - growing our local communities, tackling and reducing poverty, improving our domestic and global security.

\paragraph{}I've said many things already about democracy this afternoon and there is other business for this House to attend to. So I make myself clear now. The democratic processes that I believed in, that I won support for in this party and from the British public have spoken and I rightfully and respectfully, as a committed social democrat, do what is called for in the circumstance.

\paragraph{}So with a heavy heart. After a decade of delivery, there is no deluge of disapprobation - Debate and criticism and changes of direction, where they are resovled through democratic means, only compliment and further cultivate the modernisation I have led. Modernisation is not an event; it is a process and we must continue to renew and revive. It is now time to look to the future, at home and abroad.

\paragraph{}To that end... Recently during this historic fourth term it has become evident to me that I no longer have a strong enough political base in my cabinet to continue as Prime Minister. With the disappearance of that base, the message is clear. I believe that my constitutional purpose has been served. It is for this reason I shall resign as Prime Minister this afternoon.

\end{onehalfspace}
\end{document}