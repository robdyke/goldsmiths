\documentclass{article}
\usepackage{geometry}
\usepackage{harvard}
\usepackage{helvet}
\usepackage{setspace}
\renewcommand{\familydefault}{\sfdefault}
\geometry{a4paper}
\title{Dissertation Research Proposal, SP53004A}
\author{Student Registration No. 22164733}
\date{\today}

\begin{document}
\begin{doublespace}
\citationstyle{dcu}
\bibliographystyle{agsm}
\maketitle
\paragraph{}This research proposal addresses arguments within Political Theory and Political Economy about the nature of our contemporary political settlement. My dissertation will explore the politics and economics of contemporary western societies. I will consider the contents of and the relationships between the `master concepts' of liberalism and democracy and consider what opportunity there is for their renewal. I will put forward an ethnography of Free/Libre and Open Source and examine the political philosophy and political economy of this archetypical `networked society'. With an increasing prevalence and an apparent counter-system agenda, an analysis of Free/Libre and Open-Source in relationship with the liberal democratic and capitalist system that sustains it is important and of contemporary relevance.

\paragraph{}The meshing of the politics of the Hacker, the Artist and the Activist has developed a manifesto of Free/Libre and Open-Source ideas, which I will refer to as \emph{f/los}.\footnote{A more common acronym, FLOSS, demonstrates the genealogy of the term and its origins in computing. I will not refer to free/libre and open-source ideas with an additional \emph{s} as I am not directly concerned here with the software aspect of the definition which the letter denotes.} Increasing contact with f/los artefacts, technologies and economies and the wider dissemination of f/los concepts is raising popular consciousness of emerging redefinitions for master concepts such as freedom, property and ownership. Freedom and openness are two ideas central to the hacker culture and the movements it has influenced. My research will lead me to examine critically the political philosophy of f/los to reveal the content of these emerging definitions, how they differ from other conceptions, and what new understandings could contemporary politics and economic inflect from these ideas.

\paragraph{}I describe myself as a hacker\footnote{``To understand the full meaning of the word `hacker', it helps to examine the word's etymology over the years. [...] Once a vague item of obscure student jargon, the word `hacker' has become a linguistic billiard ball, subject to political spin and ethical nuances. Perhaps this is why so many hackers and journalists enjoy using it.'' \cite[Appendix B.]{Williams:2007uq}}. I participate in open-source projects and `gift' a significant proportion of my creative endeavors to `the community' and to the benefit of humanity as a whole. While I question the likelihood of someone benefiting from my individual openness and free gifting, the coordinated collaboration of individuals in various f/los communities has bestowed significant benefits on us all. The idea of `community' is frequently evoked in f/los discourse and most f/los activity takes place within a group, with disparate and dispersed members. I will examine the construction and maintenance of identity \cite{Appadurai:1996lp} and the definition of a community by f/los because I think that the social relationships that form these communities and the political culture that conditions the governance of them may provide useful examples for developments in contemporary society.

\paragraph{}To hackers closedness is an obstacle to development. These obstacles \emph{must} be dismantled, the process documented and the knowledge gained by the exercise shared with others. The imperative is critical to the philosophy: there is an obligation to fellow humans to share. Closedness is a lack of recognition of relationships. Closed devices will not inter-operate. Closed systems do not share information. Closed people relate to each other as means and not as ends.

\paragraph{}Closure characterizes the liberal-democratic political culture. The balance between these hyphenated concepts is delicate, leaving little but breathing space. Political liberalism's freedoms for the individual are curtailed by democratic mechanisms on behalf of the same free individuals, sovereignly constituted. Yet democracy's 'people', liberated from being cogs, now appear as little more than figures watching the machine they have been freed from continue its work without them. What was perhaps once ideal now seems like a bad idea. Worryingly it appears that these weakly linked concepts, liberal-democracy, are the only ideas left. I will critically examine the nature of liberalism and democracy, but most crucially I will examine the practical meaning of the hyphen that links them in our contemporary context. The discourse analytical approach of Laclau \cite{Laclau:2004dn} and other post-structuralists will help me rigorously demonstrate my assertion of `closure', a `closure' to anything prior to the acceptance of the liberal-democratic consensus \cite{Mouffe:2000fk}. To examine the implications of this perceived closure in liberal-democractic political systems I will draw on the work of \citeasnoun{Crick:2005il}, whose \emph{In Defense of Politics} is a powerful criticism of closure.

\paragraph{}Underpinning or central to - depending on your preference for ordering these things - liberal-democratic politics and the institutions they configure are economic activities and social relationships. In some societies these are still quite distinct from each other, in others the distinction is blurring. There is a  tendency towards the subsuming of the social in the economic. With regard to the status of economic activities in contemporary society, it is clear that capitalism now towers above `the market', now large-scale, diversified and abstracted, and it moves on seeking new arenas where transactions take place to capitalize. What are the power relations that appear to enable capitalism to `Extend, embrace, extinguish'?\footnote{This term seems apt to describe the growth imperative, the \emph{modus operandi} of capitalism. It was used to describe the business practices of Microsoft, one of the worlds largest businesses. Microsoft are dependent on the revenue they can generate from proprietary products which are protected by intellectual property rights in order to make a profit.} In what ways does f/los philosophy and political economy represent a challenge these power relationships? To what extent is f/los part of the same political economy as neo-liberal doctrine that f/los appears, in many aspects, to be the antithesis of.

\paragraph{}As I intend to demonstrate of liberal-democracy, so I seek also to demonstrate the closure of capitalism as an economic system. Capitalism considers anything external to its self as raw materials, as means towards the ends of primitive accumulation. Here of course Marx and Luxemberg have written plenty, although my own understanding of capitalism as `closed' has been developed by a contemporary analysis of and introduction to `fictitious capital' by \citeasnoun{Goldner:2007pb}. 

\paragraph{}The ``contemporary proliferation of political spaces and the multiplicity of democratic demands'' \cite[p.17]{Mouffe:2000fk} needs a new politics, a politics that works at the global and the local, the universal and the particular. New relationships of interconnectedness between people, communities, identities, artifacts, locations and events need to be reflected in the political culture of our institutions. What are the dynamics of these relationships? Where are these new political spaces emerging in such a tightly closed political culture?

\paragraph{}Anarchic? Communitarian? Autonomous? A future f/los society of free individuals freely participating in communities socially sharing with others without the motivation of profit? My first chapter will be an ethnography of the politics and economics of f/los, revealing its radical ideas of community. I will draw upon the writing of alpha-hackers such as Eric \citeasnoun{Raymond:2000lr} and Richard \citeasnoun{Stallman:2002xy} as well as interviews, hacker-folklore and personal experience. My approach to an evaluation and criticism of f/los will be discourse analytical. I have been reading \citeasnoun{Terranova:2004ly} and \citeasnoun{Appadurai:1996lp} and I will draw on their arguments in considering the invention of hacker identity in a networked society. The role of `the community' in educating and inculcating a good f/los civic citizen will also be examined.

\paragraph{}From f/los I move on to the contemporary western political settlement for my second chapter. Here I will examine the nature of modern political liberalism and democracy, considering the constituency of `the people' with regards to the principle of popular sovereignty and the diminishing of demos through universalized rights. I will examine the tension in the hyphen between  liberal-democracy using the discourse analytical method of \citeasnoun{Mouffe:2000fk}. Drawing on her work in the \emph{The Democratic Paradox} with that of \citeasnoun{Bobbio:2005vn} and also \possessivecite{Keenan:2003tw} \emph{Democracy in Question} I will look for opportunities for radical `democratic openness in a time of political closure'. \citeasnoun{Dahl:1961fj} asked `Who Governs?' and found a gradual shift from oligarchy to pluralism. \possessivecite{Barry:2001ff} \emph{Political Machines} was concerned with `governing in a technological society'\ldots This chapter, then, develops both those lines of enquiry and can be conceived as `Who Governs in a Networked Society'. The concept of \emph{multitude} \cite{Hardt:2005zt} may be useful to me here in my attempt at re-conceptualizing the popular sovereignty of the people.

\paragraph{}I continue with my focus on contemporary politics in my third chapter with a consideration of the political economy of capitalism. What are the typologies of power conditioned by neo-liberal democratic capitalism? What challenges, if any, does f/los political economy present to the dominant system of exchange? What happens when the empty signifiers of `free' and `open' are co-opted and adapted by commercial enterprise; when `free as in speech' becomes `free as in beer', as demonstrated in `web 2.0's' social sharing spaces, do questions of access now replace questions of ownership of means of (re-)production? \citeaffixed{Hartzog:2007qm,Wright:2005wb}{e.g.}

\paragraph{}The political economy of f/los is based on an appeal to the rule of law, constitutionalized in licenses and institutionalized in the methods for debate and revision of the licenses. Here f/los perhaps demonstrates a neo-liberal - understood as the conflation of political liberalism and its economic variant, or perhaps more accurately, the subsuming of political liberalism by economic liberalism - character. How does this character square with f/los's first principle, freedom? This potential disjuncture between intention and effect caused by licensing will be explored using the works of \citeasnoun{Hardie:2005px} as contra to \citeasnoun{Lessig:2004pi}. I will also reflect on the apparent universalized American legal-constitutionalism in f/los and its role in the project of Empire \cite{Hardt:2001jl}. Is f/los political economy essentially a neo-imperial non-tarrif barrier to trade and development? How does f/los respond to the charge of being neo-liberalism's handmaiden? Is `the mirror going to steal the soul' of f/los? \cite{Prug:2006xr}

\paragraph{}I will draw together the ethnography of f/los and my analysis of contemporary politics in my final chapter which will contain my search for a new Politics of Openness. Here I will examine f/los philosophy in transmission, reflecting on what economic, social and cultural scenarios inspired by f/los can be envisaged. What hope is there for a radical democracy and a radical political economy? Will we see the emergence of a f/los society and state? \footnote{I would certainly not be the first to consider such ideas as an `open source state' \citeaffixed{Claude:2005jx,Rushkoff:2003wu}{e.g.}; nor will I simplistically imaginer such a society from the clumsy, techno-utopian application of a software development and distribution methodology to the enormous complexity of the real world.} I am in search of a strategy for a revival of the left \cite{Laclau:2001rt,Laclau:1990rc} against the hegemony of neo-liberal political and economic order, so in this chapter I will discuss what aspects of f/los are compatible with a new radical politics and economy while at the same time, no doubt, posting questions for further exploration.

\paragraph{}Word count: 1795

\end{doublespace}
\newpage
\bibliography{/Users/robdyke/Documents/GoldsmithsCourses/bibliography/globalbib}
\end{document}