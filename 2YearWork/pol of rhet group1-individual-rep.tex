\documentclass{article}
\usepackage{geometry}
\usepackage{harvard}
\usepackage{helvet}
\usepackage{setspace}
\renewcommand{\familydefault}{\sfdefault}
\geometry{a4paper}
\title{Politics and Rhetoric Group Speech Task}
\author{Individual Report : Student Registration No. 22164733}
\date{\today}

\begin{document}
\begin{doublespace}
\bibliographystyle{kluwer}
\maketitle

\paragraph{What was your task in the writing of the speech and how did you contribute to the whole?} I took a initial lead in organising the group, ensuring that we had contact details, communicating our meeting arrangements outside of workshop times \&tc. I established the ways of working, encouraging the use of an online collaboration tool, a wiki, in order to facilitate our group work. The use of the wiki tool was most beneficial as it encouraged each person to contribute directly in the writing and redrafting of our speech. The tool publicly audited the contributions from each of our group, clearly demonstrating whether we had fulfilled our commitments to each other in our independent work. Perhaps this public history guarded against the ``freeloader problem''\ldots

\paragraph{}I facilitated our group working. When we were working as a group, mediating between opinions and creating the space for everyone to speak was important in order that the content of the speech reflected contributions from each group member. We captured a lot of ideas for the various sections of the speech and each of us took turns in revising and reordering the speech online. My most crucial contribution to the group was the final draft of the speech and of our group report. Our group report is representative of the contribution of each member of our the group. My task here, in completing the report, was, on the whole one of synthesis and editing of our combined contributions.

\paragraph{Explain in your view what was the most important or effective part of the speech?}The most effective part of our speech is the suspense that it builds and the shocking bombshell of resignation exploding in the final sentence. Given the background to our speech, the ``wave of hostility'', the scene is already set; the audience are aware of the Prime Ministers position and are expecting an announcement of some kind. We play with the captive anticipation of our audience - while we have their attention we will vindicate our Prime Minister. While we have their attention the audience is really listening. We set out to achieve this but our first drafts had our PM delivering his resignation too early in the speech. My contribution was the final revision of structure of our speech to hold our audiences attention long enough for our PM to secure in his valediction his legacy.



\end{doublespace}
\newpage
\bibliography{globalbib}
\end{document}