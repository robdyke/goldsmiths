You'll be pleased to know that my Master's dissertation was pretty much on radical democracy - though specifically with Judith Butler and William Connolly in mind - rather than Laclua & Mouffe. So certainly two authors influenced by Derrida. You are welcome to have a read if you like. I have it online so let me know if you want it. Basically this is the area of political thought that really got me interested and I ended up getting lost for a long time, though mainly due from the philosophical side. Which you'll hardly be able to avoid.

I've not read much of Mouffe's stuff that she does on her own, partly because its all just a bit 'liberal' and fluffy, for my liking. The stuff she has done with Ernesto Laclau is far more interesting, intellectually challenging and generally politically imaginative and useful. I strongly recommend reading "Hegemony & Socialist Strategy: Towards A Radical Democratic Politics," both for its imaginative deconstruction of certain forms of socialist analysis and also for its development of certain key radical democratic themes. I have a copy of it at home, so can happily bring that along if you like. Though I'd best refresh my memory of it first. Its been a while.

For a slightly more philosophical/Derridean view on radical democracy and the centrality of the insufficieny of theory/abudance of 'world' over concepts I would also recommend the chapter �Postmarxism without Apologies', in Laclau's, "New Reflections on the Revolution of our Time." I don't have a copy here but I could probably find one when am back at my parents at the weekend.

A good general overview of some of the debates and divisions within radical democratic thought is Tonder and Thomassen's "Radical Democracy: Politics between abundance and lack." Most of the book won't be of much use for your purposes, but the introduction will certainly help. I won't put in a bid for my former tutors chapter...

Then Zizek would probably pipe up with something about how all this theorising was just an attempt to escape the indivisible dark stain at the centre of existence, whilst still basically being an old fashioned Leninist and apologist for mass terror. So there's definitely lots of areas to explore...though I will admit I've not read anywhere near enough Zizek to being to do the man any sort of (in)justice other than spout cliches that I've encountered elsewhere.