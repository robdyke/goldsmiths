\documentclass[13pt]{article}
\usepackage{geometry}
\usepackage{harvard}
\usepackage{helvet}
\usepackage{setspace}
\renewcommand{\familydefault}{\sfdefault}
\geometry{a4paper}
\title{What is `New' about New Social Movements?}
\author{Student Registration No. 22164733}
\date{\today}

\begin{document}
\begin{onehalfspace}
\bibliographystyle{kluwer}
\maketitle

\paragraph{}A significant body of research into New Social Movements exists. Indeed, movements in society have long been the subject of analysis. My analysis of ``New Social Movements'', or NSM, will show the history of social movements in European politics, from 19th Century emancipation struggles to the resurgence of identity politics in the post-war baby-boomer generation. I will outline the relationships between the acclaimed `new' social movements and the emancipation movements of the past in order to show how the latter conditioned the former. As to what is `new' about ``New Social Movements'', I will argue that NSMs and NSM scholarship are an example of neologism and that perhaps a move fitting prefix would be ``Contemporary''. 

\paragraph{}The history of emancipation can be understood as the struggle for parity between each unit of a binary distinction: women/men, black/white, straight/gay. Emancipation movements rest on the politics of identity. Across Europe these identities manifested themselves in social cleavages that affected the political sphere. The struggle for political parity between women and men, the struggle for racial equality, the struggle for equality of sexuality have manifested themselves in both parliamentary and extra-parliamentary action since the Suffragette movement of the 19th century. Further political cleavages that arise from the self-perceived identity of an individual can be seen along religious, ethnic, linguistic, regional and urban/rural lines. These cleavages are usually understood in terms of class, that is to say, by classification into `group member' and `non-group member'. This analysis extends the Marxist economic analysis of class, that of a binary antagonism between labour and capital. Emancipation movements then are about achieving the liberty of one class from the oppression of another. Armed with this definition of class politics, we can now consider the term ``social movements''. If we understand the `social' aspect of the term as the groups within a society, the `movement' aspect can be understood as the translation of the growing consciousness of the circumstances of a particular group into action or actions to improve these circumstances. Emancipation struggles are clearly demonstrated as social movements.

\paragraph{}One of the reasons for seeing NSM as a contemporary manifestation of the traditional forms of social movement is the slow institutionalized demise of class-based politics over the course of the 20th century and the rise of a new politics. First I will give an account of the end of class conflict through institutional means. I will then examine the development and characteristics of this new politics which commonly associated with the rise of NSM.

\paragraph{}The political agendas of the movements concerned with the emancipation of women and of ethnic groups have been, over time, co-opted into the core political agenda of European democracies. This is demonstrated in the promulgation of equality and anti-discrimination legislation by governments of European nations at different times over the last 60 years. Indeed, the extension of the enfranchisement to women and the realization of parity between electorates has a much longer history. Yet the institutionalization of the aims of the emancipation movements of the 19th Century does not indicate the complete realization of these goals through parliamentary methods. Across Europe examples of class-based discrimination are clear and there is evidently significant action remaining in completing parity and extinguishing prejudices in our wider society. The illustration of institutionalization only demonstrates the \emph{political} successes of an emancipation movement, diminishing prejudices in society is much more gradual.

\paragraph{}This institutionalization can be seen across Europe in the politics of individual countries. My argument is further supported by the normalizing of the parity between classes into legislation that binds all member-states of the European Union. Articles IV, IX and X of the European Constitution on Human Rights\footnote{\cite{Europe:1950uq}} are examples of the normalization of emancipation into the political process. 

\paragraph{}Political cleavages arising from language, religion and regional identity have long been integrated in the parties and political systems of Europe's nations. Italy's political system for much of the latter half of the 20th century institutionalized the divisions between capital and labour and between the sacred and the secular. The dominance of Italian politics by the Christian Democrat party and the exclusion of the Communist party demonstrats both cases. German's bi-cameral legislature balanced regional and later, post reunification, `national' cleavages. These two examples further demonstrate the assimilation of class conflict into the political system. For a fuller elaboration of this institutionalization argument see Chapter IV of \citename{Eder:1993fj}.

\paragraph{}Class conflict in the Marxist sense has also declined as living standards have improved and as economies have developed from industrial to post-industrial / service economies. For some nations of Europe this shift in the mode of the economy is evident in the decline of manufacturing and the growth of public sector and service provide unions (e.g. Britain). In other nations this transition is evidently not cemented because manufacturing still holds a crucial place in the economy (e.g. Germany and Italy). \possessivecite{Porta:2006kx} chapter ``Social Movements and Structural Changes'' talks further on the conditions of the decline of class based politics. This analysis demonstrates the importance of post-material values as a determinant on social change.

\paragraph{}The decline of class conflict and an increasingly post-materialist political culture created the breathing space for new themes of politics to emerge. Common examples found in the literature on this subject are the currents of feminism \cite{Gelb:1990ys}, pacifism \cite{Rochon:1990vn} and ecologism. These currents have two conditions in common. First, each current is based on ``broadly humanist critique'' of modernity demonstrating a concern which goes beyond the individual and considers the total of human existence. Second, they have ``little, if any, inclination to escape into some spiritual refuge'' \cite[pp. 280]{Dalton:1990zr}, showing a continuation of the Enlightenment rejection of religious absolutism as a political order.

\paragraph{}These three specimen currents obviously concern themselves with different issues but there are striking similarities between the movements that arose around them. Before I elaborate on these similarities, I need to return to my definition of emancipation movements in order to extrapolate a definition of NSM. I contend that NSM are a contemporary continuation of emancipation movements. Broadly speaking, NSM represent a continuation of emancipation movements in that as a society we may all be liberated from sources of oppression (our three themes seek emancipation from misogyny, violence and damage to our biosphere, respectively) in order that we might all live a whole human existence. For the members of these these particular NSM we are all oppressed by a social order which is conditioned by values that are incongruent with a positive whole human existence.

\paragraph{}Emancipation movements were concerned with changing the material conditions of a specific group. In post-materialist societies the individual is materially secure. The `social' aspect of the term `social movements' then reflects this development: if our material well-being is secure, concerns beyond the material have the opportunity to develop. The `movement' aspect of the term has also shifted from the emancipation movement description. Emancipation movements were identified through binary membership criteria. In contrast, the themes that concern us all can-not be sufficiently well understood in terms of this dichotomy and therefore the `movement' aspect of NSM should be understood as the translation of this growing holistic interest into the establishing and interaction of groups to improve these circumstances. New Social Movements as a term, then, identifies aggregate manifestations of diffuse concerns in an atomized society.

\paragraph{}A difference between emancipation movements and NSM, (and perhaps something that is `new') is the variety of actors who participate in a given movement. A consideration of actors in the ecology movement and their relationships will demonstrate this idea. Ecologism considers the importance of the biosphere to \emph{all} of nature and is therefore the preeminent example of a New Social Movement. Identification as a supporter or participant in the ecology movement can be demonstrated, at the micro-level, in the adoption of green lifestyles by individuals, and at the macro-level, by the existence of organisations devoted to environmental issues, such as Greenpeace. Participation and membership are fluid, with no mandatory relationship between individual and groups in NSM. The relationship between the human capital broadly engaged with a social movement and the mobilization of this capital by organisations is related to the resource mobilisation theory of NSM. This approach emphasizes organisational actors and the relationship between these bodies and the political establishment. See \possessivecite{Brand:1990ly} investigation of NSM using this approach.

\paragraph{}Continuing my consideration of the ecology movement I will now examine the activities of actors that engage with the political process. Green politics are manifested in different forms and with different degrees of political success across Europe. I have already shown the requirement of post-materialist values for the new politics of a NSM to establish itself. The political success, or otherwise, of NSM actors is conditioned by the institutional arrangements in the particule European nation. In Germany, \emph{Die Gr�nen},\footnote{The website for the Greens is http://www.gruene.de} the Greens, have had grown from a loose network of local environmental groups to the minority party of government. The success of this political party was conditioned by the proportional representation mechanism used in the elections to government. In Italy, where coalition government is the usual outcome of the PR electoral system, the \emph{Federazione dei Verdi}, or simply \emph{Verdi}\footnote{The website for the Green Federation is http://www.verdi.it} have also had political success, with party leaders holding ministerial positions in recent centre-left coalition governments. Federality and proportional representation appear to be conditions for political success. The contrast to Britain demonstrates my conclusion here: the unity state and the majoritarian electoral system has discriminated against Green political parties at the national level. However, where proportional representation has been used as the mechanism for election the Green party has had much greater success. The UK Greens\footnote{The website for the Uk Green Party is http://www.greenparty.org.uk} hold two seats on the Greater London Assembly\footnote{Source: GLA website} and have two MEPs.

\paragraph{}\emph{Verdi}, \emph{Die Gr�nen} and other European Green parties demonstrate the fluid composition of NSM and the relationship between co-ordinated political manifestations of a social movement and the supporters of a social movement. \emph{Verdi} is a federation of activist ecology movements. Intra-party, the British Greens, like \emph{Verdi}, are a federation of local ecological political actors. The inter-party and extra-party relationships between other actors in the ecology movement needs to be considered. The national Green parties of Europe sit in the same European Parliamentary Group and mobilize political capital in concert at the inter-governmental level. But what can be found that is common to the relationships between the actors of the ecology movements in general? 

\paragraph{}My question leads me back to my suggestion that NSM study is a neologism. The nexus where the constituent actors of a new social theme interact, the origins and the practices of these actors, is perhaps already well covered by other analytical perspectives. Indeed the arguments in \possessivecite{Dalton:1990zr} volume and the Editors conclusions demonstrate that there is no consensus as to whether NSMs can be considered `new' in any theoretical or analytically meaningful way.\footnote{\cite[esp. Ch 14]{Dalton:1990zr}} However, the definition of the term `New Social Movements' and the `new' synthesis of theoretical perspectives that arises from the study of such a multifaceted phenomena enriches our discourse. New Social Movements are a product of our post-materialist society. The issues they highlight need to treated with the same seriousness as the emancipation movements from which they have grown.

\end{onehalfspace}
\bibliography{globalbib}
\end{document}
