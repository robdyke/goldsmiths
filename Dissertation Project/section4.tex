\newpage
\chapter{Hacking a Free/Open Politics.}
\paragraph{}The ``contemporary proliferation of political spaces and the multiplicity of democratic demands''\footnote{\cite[p.17]{Mouffe:2000fk}} in our networked society needs a new politics, a politics that works at the level of the global and the local, the universal and the particular. New relationships of interconnectedness between people, communities, identities, artifacts, locations and events need to be reflected in the political culture of our institutions. What positive benefits can free/libre and open-source in transmission to our wider society bring to politics? How are we to move beyond the individualist ontology of liberalism and a politics dependant on capital? What hope is there for a new ontology of interdependence, a way of being among?

\paragraph{}I have outlined changes in political culture that the adoption of open-source principles of collectivism, participation and consensus decision-making by organisations and institutions have produced. I have also considered the reconfiguration of economic activity through the liberating free/libre hack on property rights. In this section I will bring these threads together, imagining a free/libre open-source society which shares, and shares widely, the wealth of networks, grows open communities and strengthens democracy. No longer can free/libre and open-source be dismissed as ‘merely metaphorical’ or utopian. As van~den Boomen \& Sch{\"a}fer point out, ``there is more at stake than just a vague metaphor for a transparent, democratic and non-private constitution. Of course, notions of ‘freedom’ and ‘openness’ appeal strongly to the social imagination, and this can easily result in utopian daydreaming.''\footnote{\cite[p. 7]{Boomen:2005uq}} Let us recall Appadurai here, reminding us that the imagination is a necessary part of any innovation, that `imagination is a social practice', both a form of work, and a form of negotiation between sites of agency and globally defined fields of possibility.\footnote{\cite[p. 31]{Appadurai:1996lp}}

\paragraph{}I argue that the relational and constructive ontology of the hacker and free/libre and open source ideas are beginning to displace the dominant liberal \textit{homo economicus}, the rational utility-maximising consumer of communicative capitalism and the protestant work ethic associated with Weber.\footnote{See \cite[Ch. 1 \& 2]{himanen:2001he} for a full discussion on the hacker ethic as contra the Protestant Work Ethic associated with \cite{weber:1958}.} This reemerging \textit{homo socialis} recognises his/her historical and social situatedness, both of  which \textit{homo economicus} is emphatically encouraged not to reflect upon by the propagandistic and manipulative efforts of communicative capitalism's individualistic and ahistorical modes of thought.

\paragraph{}The network society deconstructs modernity's institutions and constructs new institutions that reflect the dynamics of new power relationships. Free/libre is a radical hack of the institution of rights, disrupting and reconfiguring the power relationships of property and demanding new politics relevant to the networked society. I'm not going to quibble with the liberal basis of much of the hacker ontology and the free/libre political philosophy. I'm all for free, autonomous individuals solving their own problems in relationship to their material conditions, adding to the sum of human knowledge. It is the logic of capital that reconfigures this knowledge as property and wields property as power. A politics for the networked society need not reject the symbols of liberal economics, and certainly must not reject the values of democracy. Yet it must move beyond the structuralist epistemology of the nation-state as the container of politics.

\paragraph{}In \textit{Multitude}, Hardt and Nergi outline four interpretations of either a challenge or contribution to democracy from the globalisation of capital.\footnote{\cite[pp. 232-236]{Hardt:2005zt}} On the `left' of the political spectrum, the social democrat sees global capital as an obstacle to democracy and in response strengthens the democracy of a people through the nation-state and uses this site to regulate capital. The liberal cosmopolitan perspective recognises a contribution to democracy from globalisation and in response strengthens the democracy of the people through global citizenship. On the `right', the neo-liberal response favours the globalisation of capital, viewing this as inherently democratic. While the `traditional values conservative' response, similar to that of the social democrat, seeks to retreat behind national borders. I agree with Hardt \& Negri that ``none of these arguments [\ldots] seem sufficient for confronting the question of democracy and globalisation.''\footnote{\textit{Ibid.,} p. 236} They place their political project, the global democracy of the multitude, outside this categorisation.

\paragraph{}Capitalism's crucial contingency is the principle of liberty, which is commonly understood as the protection of private property through regimes of `property rights'. The anti-globalisation stances of the social democrat and the `traditional values conservative' each uphold the regime of property rights against global democracy by strengthening the state / capital relationship. While the pro-globalisation stances of our cosmopolitan liberal and neo-liberal again uphold the regime of property rights, this time using global democracy to form a supra-state / capital relationship.

\paragraph{}Open-source is firmly within Hardt \& Negri's categories as markedly liberal in terms of its pro-globalisation and pro-democracy stance. Indeed open-source in and of itself is not anti-capital. Free/libre on the other hand is outside these categorisations: it is radical. It is anti-capital because it hacks (subverts) the dominant property rights regime of capitalism. It is pro-globalisation and pro-democracy in as much as it is radically free, open and egalitarian.

\paragraph{}It is certainly true that much of free/libre and open-source discourse, like NSM discourse, ``perpetuate[s] narrow understandings of the economic and the cultural''\footnote{\cite[p. 275]{Hardt:2001jl}} by failing to take adequately  consider the increased mixing of economic and cultural phenomena. To a certain degree, free/libre, open-source and creative commons concepts are all ``various forms of social contestation and experimentation, all centred on a refusal to value the kind of fixed program of material production.'' On the other hand, recalling my earlier argument for a reading of open-source as a social movement to further the adoption of open-source; free/libre, open-source and creative commons have all `sold out' by voluntarily channelling and structuring themselves within the logic of capitalist exchange.\footnote{See \cite[p. 80]{Terranova:2004ly}}

\paragraph{}The challenge for free/libre in transmission in a networked society is two fold. Firstly to recognise that these ideas are more than ``\textit{`merely cultural' experimentation}'' in isolation and that ideas have a ``\textit{very profound political and economic effect}.''\footnote{\cite[p. 274, original emphasis]{Hardt:2001jl}} The second challenge for these ideas is to recognise and guard against the counterrevolutionary responses to the effects that they have. Against the free-software revolution came the counterrevolution of open-source. So to some extent open-source can be considered a powerful pacific weapon, neutralising the anti-capitalism of free/libre. Open-source networked production and the openness that that entails has the appearance of transforming the economic relationships of late capitalism. However in reality no such transformation takes place as the fundamental logic(s) of communicative capitalism remain intact.

\paragraph{}The undecidable nature of the relationships between the hacker, who produces abstractions, and global capital, which extracts value from these abstractions is revealed in the cyclical adaption and cooption, of a pattern of deconstruction and construction, between free/libre and open-source and communicative capitalism. The positive possibility of free/libre needs defending from the unremitting attempts at neutralisation by neo-liberal and liberal cosmopolitan aristocrats. What the `liberal aristocrats', or as \v{Z}i\v{z}ek would call them, the `liberal communists'\footnote{\cite{Zizek:2006vile}} who perverted free/libre into open-source ``do not understand is that in the era of biopolitical production, liberalism and liberty based on the virtue of the few or even the many is becoming impossible.''\footnote{\cite[p. 236]{Hardt:2005zt}} Hardt and Negri continue, arguing that ``the logic of private property is being threatened by the social nature of biopolitical production.''\footnote{\textit{Ibid.,} p. 236} The transformative hack (subversion) of private property rights by free/libre philosophy recognises the social nature of production, a point sadly missed by Hardt and Negri.

\paragraph{}The liberal communists have vehemently adhered to the Weberian spirit of capitalism piously accumulating wealth by the sleight of hand of open-sourced enclosure. True to the Weberian logic, the liberal communists demonstrate their virtue through a directed distribution of this wealth. This `open philanthropy' is not a hack like `open capital', rather it is `philanthrocapitalism', an appropriation of the symbols of open-source to extract private profit first and to `do good' second. So, while entering into `tactical alliances' with the liberal communists may be necessary in order to win local, specific, `subjective' gains, as \v{Z}i\v{z}ek reminds us, they are the ``agents of the structural violence'' of communicative capitalism: ``liberal communists are \textit{the} enemy of every true progressive struggle today.''\footnote{\cite{Zizek:2006vile}, original emphasis}

\paragraph{}Our re-emerging \textit{homo socialis} is able to overcome both the tyranny of structure and of structurelessness by adopting the radical politics of openness in social practices and institutions.\footnote{\cite{Freeman:1970}} The monolithic structures of corporate bodies such as business, state agencies, and social movements are restructured by openness. The radical transparency of openness reveals the critical interdependency of our social relationships. The adoption of openness for an ``open-source society'' makes possible a deep public scrutiny and democratic accountability, but, most crucially it encourages collaboration in new, better social programmes.\footnote{\cite[pp. 339-40]{Hardt:2005zt}}
\skip
\paragraph{}The new politics configured by free/libre philosophy and the practice of openness are an integral part of the ecology of the pro-democracy, pro-globalisation and anti-capital networked social justice movements, described here by David Graber:
\begin{quote}``This is a movement about reinventing democracy. It is not opposed to organisation. It is about creating new forms of organisation. It is not lacking in ideology. Those new forms of organisation are its ideology. It is about creating and enacting horizontal networks instead of top-down structures like states, parties or corporations; networks based on principles of decentralised, non-hierarchical consensus democracy. Ultimately, it aspires to be much more than that, because ultimately it aspires to reinvent daily life as whole. But unlike many other forms of radicalism, it has first organised itself in the political sphere - mainly because this was a territory that the powers that be (who have shifted all their heavy artillery into the economic) have largely abandoned.''\footnote{\cite{Graeber:2002}}
\end{quote}

\paragraph{}The free/libre hack of neo-liberal property rights presents the possibility to re-populate the abandoned political sphere with a politics of self-determinisation and of being in common, `reinventing daily life as a whole'. The radical principles of openness present the possibility of a lightly structured networked society, of self-governing individuals, self-organising the movement of movements.