\documentclass{article}
\usepackage{geometry}
\usepackage{helvet}
\usepackage{harvard}
\usepackage{setspace}
\renewcommand{\familydefault}{\sfdefault}
\geometry{a4paper}
\title{Critically discuss Appadurai's theory of globalisation with reference to his idea of scapes}
\author{Student Registration No. 22164733}
\date{\today}

\begin{document}
\harvardparenthesis{none}
\citationstyle{dcu}
\bibliographystyle{dcu}
\maketitle
\doublespacing
\paragraph{}Appadurai's theory of globalisation stands out from the myriad of theories in a number of ways. First and foremost his theory of globalisation emphasises identity and perspective over institutions and nation-states, the spatial and temporal over the economic and political; in this sense Appardurai's theory of globalisation is post-modern. Secondly, his approach is post-structural: Appadurai is challenging other models of social theory that are based on models of structure. Appadurai's theory of globalisation goes beyond the usual dualisms of global/local relations with an account of the effect of deterritorialisation and global cultural flows on the construction of identity by individuals, `communities' and that of the nation-state. My discussion of Appadurai's theory of globalisation will consider these elements and give an account of his metaphor of \emph{scapes}.

\paragraph{}Before we begin, however, we must draw a distinction between theories of globalisation and globalisation theory. Appadurai's thesis is of the former and represents a ``spatio-temporal reformulation of social theory''\footnote{\cite[p.4]{Rosenberg:2000fg}} of the sort that the latter might ground themselves on. Where accounts such as centre-periphery models associated with Wallerstein\footnote{\cite{Wallerstein:1979cw,worsley:1990ow}} and Gunder Frank\footnote{\cite{Frank:1975du}}, migration theory models and the network models of  Giddens\footnote{\cite{Giddens:1990cm}} and Castells\footnote{\cite{Castells:1996ns}} do consider time and space, they neither emphasise these dimensions to the degree that Appadurai does, or consider only \emph{effective} dimensions to the exclusion of \emph{affective} dimensions.

\paragraph{}Theories of globalisation/globalisation theory can be roughly divided into system approaches and cultural approaches. Within these groups three further categories emerge, illustrating different perceptions of pervasiveness: sceptics, transformationalists and hyperglobalists\footnote{\cite[p.120-121]{Hoogvelt:2001gp}}. Theories of the former see culture as a surface phenomenon,  which demonstrates the human tendency to organise and categorise perceptions and experience, with universal underlying ordering principles that are the same. These principles are usually descibed in terms of binary oppositions or dualisms: centre/periphery, micro/macro. Culturalist approaches are also prone to binary thinking, with global/local dichotomy used frequently. These dualisms each delinate boundaries, objectifying a closed entity for study. And whilst deliniating political and cultural approaches to the modern world I too have perpetuated a dualism, the ``hegemonic separation of the political and the cultural, of the building of the state and the construction of meanings.''\footnote{\cite[p.81]{Krohn:2003}} Where globalisation from a cultural perspective discusses cultural homogenisation (a loosely disguised fear of Americanisation) and cultural diffusion, globalisation from a system perspective discusses economic vis political themes: these are ultimately perspectives on the same coin. Appardurai's thesis is a challenge to this binary separation and a call to recognise perspective.

\paragraph{}Appadurai's \emph{Modernity At Large}\footnote{\cite{Appadurai:1996lp}} is not explicitly a work of political economy or political theory. His book is a challenge to the academic discipline of anthropology to ``...come in from the cold and face the challenge of making a contribution to cultural studies without the benefit of its previous principal source of leverage - sighting of the savage''\footnote{\cite[p.65]{Appadurai:1996lp}}. He believes that anthropology must radically rediscover its self in order to discharge its history as `the handmaiden of Imperialism' (passim). Appadurai's perspective as an anthropologist and his approach to the study of human social interaction shines a new light on the topics of `modernity' and `globalisation' for neighboring disciplines. His contribution to cultural studies is his idea of \emph{scapes}. This typology of \emph{technoscapes, mediascapes, financescapes, ethnoscapes} and \emph{ideoscapes}, describes global flows of technology, media images, finance, people and ideas, respectively.

\paragraph{}Appadurai sees our modern world as interactive in a ``strikingly new sense''\footnote{\cite[p.27]{Appadurai:1996lp}} and his framework of \emph{scapes} is an attempt to delineate and describe the relationships of it. By \emph{technoscapes} the sum total of technology transfers is meant; both high and low, physical and informational technologies. Barry's \emph{Political Machines}\footnote{\cite[Ch.2]{Barry:2001ff}} explores the contents of technoscapes. Barry demonstrates that international standardisation, such as ISO certifications and international intellectual property regulation are a part of Appadurai's technoscape along side more conventional examples of technology; equipment and knowledge. \emph{Mediascapes} refer to the modern milieu of imagery and text, distributed by new-media. A facet of Appadurai's anthropological approach to a theory of globalisation is identity politics, mediascapes are deliniated distinct from other scapes due to the cultural role of the media in the invention, or to use Appadurai's term, the `imagination' of identity. Related to to the identity function of mediascapes, are ``the dilemmas of perspective and representation'' brought about by ``changing social, territorial and cultural reproductions of group identity''\footnote{\cite[p.48]{Appadurai:1996lp}} that are the themes encompassed by \emph{ethnoscapes}. A recent example of international money flows, Appadurai's \emph{financescapes}, can be seen in the `run' on British bank Northern Rock. The reselling of packaged debt from mortgages taken up by Americans' considered sub-prime borrowers lead to a crisis of confidence in international money markets leaving a British bank unable to meet its daily cash flow requirements. This caused panic amoung depositors of the Northern Rock and required government intervention to rescue the institution. The themes of his final scape, \emph{ideoscapes}, are the ``elements of the Enlightenment world view''\footnote{\cite[p.36]{Appadurai:1996lp}}; specifically the chain of ideas, terms, images: freedom, welfare, rights, sovereignty, representation; that condition trans-national relations.

\paragraph{}A criticism of this metaphor is that Appadurai does not provide an objective account of the shape or topology of the spaces of these scapes in contrast to other models of interactions in our modern world, models which are frequently grounded on the empirical reality of the nation-state. This approach - the making of what is studied into a neat component before studying the relationships from and to it - is the \emph{modus operandi} of many globalisation theories. ``This epistomology which starts out with society as a given, consisting of so many closed, bounded entities\ldots contained entities in interaction with an equally contained society [was] modelled on the state, with its clear boundaries vis-\`{a}-vis other entities.''\footnote{\cite[p.126]{Nustad:2003}}. Yet this is precisly where these models fall down - their functional and reductionist primacy of the nation-state `can not capture the increasing complexity of reality in their apparatus; that is, \emph{complex in relation to ealier assumptions}.'\footnote{\cite[p. 126, emphasis added]{Nustad:2003}}

\paragraph{}Appadurai does not take the nation-state as a bounded entity for the (re)modeling of other entities, nor is he compressing the nation-state from above with trans-national actors as systems models do, nor from below with global actor agency networks, often modeled in New Social Movement studies.\footnote{\citeaffixed{Porta:2006kx,Dalton:1990zr}{e.g.}} Yet the nation-state is not negated by Appadurai. He sees modernity as challenging the traditional ideas of nation states and of national identies. Nation states and national identies need to understand themselves in a new way, as a `node' in a ``complex transnational construction of imaginary landscapes.''\footnote{\cite[p.31]{Appadurai:1996lp}}

\paragraph{}Appadurai's account of modernity is perspectival and highlights how circumstance means that any perspective is both refracted and partial. Nustad's chapter in Eriksen's \emph{Globalisation}\footnote{\cite[p.122-137]{Nustad:2003}. \cite{Eriksen:2003gl}. I have found this title, \emph{Globalisation}, from the Studies in Anthropology series, to be invaluable in engaging with Appadurai's work} elaborates three perspectives on perspectival accounts of modernity. Firstly the distorting effects of speed and scale, a theme that Paul Virilio has examined at length\footnote{For Virilio, the speed, scale and pervasiveness of modern media and technology detrimentally alter our perception, reducing our depth of vision, dissolving horizon. \cite{Virilio:2005jl,Virilio:2006sp}}, can confuse association and causation. This is evident when considering the same event through micro and macro lenses, or from `the centre' or `the periphery': the distinction is really two perspectives on the same point. Secondly Nustad demonstrates that a network is local at all points, global by association and yet not universal. Nustad's third point is that perspective can only ever be partial, creating uncertainty.

\paragraph{}The identity politics that figure so prominantly in Appadurai's work exhibit a Hegalian conception of human development, of self-consciousness becoming self-conscious, and a strong role of the psychodynamic in the fore-grounding of the imagination and of the imaginary refraction and inflection of images mechanically produced and trans-nationally distributed, of shared imagined communities, of diaspora and other imaginary constructs, of landscapes of mediated aspiration. ``[T]he imagination has become'' for Appadurai, an ``organised field of social practices, a form of work, and a form of negotiation between sites of agency (individuals) and globally defined fields of possibility.''\footnote{\cite[p.31]{Appadurai:1996lp}}

\paragraph{}Time and space for Appadurai are affective on the imagination and identity. The imagination is not a-historical, nor bound by the present; both the future and past can be imagined and recalled. The realisation of this imaginary dimension to time and space plays havoc with the standard western linear chronology and significantly challenges all paradigms based on this conception: evolution, social hierarchy, colonisation, \&tc. Identity remains related to `primordial'\footnote{\cite[Ch .7]{Appadurai:1996lp}} anthropological concerns such as kinship and land. Yet the post-modern identity described by Appadurai is also affected by imagined trans-national flows. Identity and imagination then are crucial to the perspectives of individuals and groups. This will inevitably lead to \emph{difference} in perceptions and account for \emph{disjuncture}; the agency of co-operation and of conflict. `\emph{Disjuncture}' and `\emph{difference}', then, mark out the contours of  \emph{scapes} and the points at which they meet with other \emph{scapes}, in a given circumstance. 

\paragraph{}One approach to appraising the utility of a brace of new social terminology such as Appadurai's \emph{scapes, disjuncture} and \emph{difference}, is to ask does it explain what it attempts to explain and, most crucially, to what extent are existing social theories displaced by a successful response to their long-standing problematisations. The question for the ideas in `Modernity at Large' is, then, has Appadurai provided us with a compelling meta-narrative with his \emph{scapes}? Does he displace the established world-systems theories that have been useful to us in analysing the world? Is Appadurai's metaphor really another macro-level world system theory and therefore open to the same criticisms as Wallerstein and Gunder Frank as being telelogical and deterministic?

\paragraph{}Appadurai's \emph{scapes} challenge to the social sciences is to leave behind the universalising tendencies of the dualistic conceptions outlined above to find new ways of representing the links between the imagination and social life, especially as the imagination is ``part of the repertoire of every society, in some culturally organised way.''\footnote{\cite[p. 53]{Appadurai:1996lp}} As individual identity and imagination is inspired by memes in the macro as well as the micro, Appadurai's \emph{scapes} are a way of stressing the importance of ``large-scale realities in concrete life-worlds''. The imagination is offered many visions of the world and of what is possible in it through \emph{mediascapes}. These include positive, empowering and emancipatory ideas as well as the most ``brutal and dehumanising of circumstances, of harsh inequalities''. From this milleu of realities and ideas that \emph{come from elsewhere}, identity is imagined, ``[t]hus, standard cultural reproduction (like standard English) is now an endangered activity that succeeds only by \emph{conscious design and political will}, where it succeeds at all.''\footnote{\cite[p.53-55, emphasis added]{Appadurai:1996lp}}

\paragraph{}`Conscious design and political will' clearly refers to what Marxist accounts of globalisation might describe as the power relationships that structure circumstances and the actions of individuals and groups in them. Power relationships are equally part of neo-liberal political thought. Appadurai's theory of globalisation, while describing the multiple dimensions of identity, does not account for these power relationships and how they might condition and dominate the identity of individuals and groups.

\paragraph{}Appadurai is drawing our attention to the complexity in fully fractal, polythetically overlapping world and stressing that established macro-metaphors and micro analysis of social theory are woefully unprepared to examine this chaos. ``\ldots [T]he great traditional questions of causality, contingency, and prediction in the human sciences'' would now be framed by Appadurai ``in a way that relies on images of flow and uncertainty, hence \emph{chaos}, rather than on older images of order, stability, and systematicness.''\footnote{\cite[p.46-47, original emphasis]{Appadurai:1996lp}} While he has not fully displaced other social theory, his work is an ``economical technical vocabulary and a rudimentary model'' \footnote{\cite[p.47]{Appadurai:1996lp}} which invites further development from the social sciences.

\newpage
\singlespacing
\bibliography{/home/robd/projects/robdyke/goldsmiths/bibliography/globalbib}
\end{document}