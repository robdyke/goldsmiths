\paragraph{}The Wilson government of 1964-66 and 66-70 marked a period of liberalism in the Labour Party. Although not manifesto commitments, Private Members Bills were tabled proposing these pieces of legislation, decriminalising homosexuality and abortion in 1967 and abolishing capital punishment in 1965. The success of these PMBs is indicative of the strength of the Parliamentary Labour Party and also the strength of the party in parliament at this time. A clear majority of 110 seats following the 1966 election gave the Labour enormous political capital to implement policy.  The Cabinet of the 1966-70 were ostensibly  of the right. Govt economic problems, pressures on sterling, debates over positition and participation in european union project.

\paragraph{}Labour was removed from government in 1970 due to the response of the electorate to economic pressures arising both nationally and internationally. As the 'post-war settlement' was based around a 'Kenyesian consensus' economic pressures, Labour government policies based on a Crosslandite approach were no longer financially sustainable.

\paragraph{}The rise of the right of the party while in opposition in the1950s can be understood in terms of a reaction to the party 'being too socialist'. Revisionism in the 1970s was, initially, about 'not being socialist enough'. \possessivecite{Miliband:1961lr} book \underline{Parliamentary Socialism} is as significant a publication to the Left of the Party as Crossland's \underline{Future of Socialism} is to the Right. His thesis was that Labour's policy in Government had not met the challenges of the world with socialism. His book was influential on the Left of the party during the 1970s.

\paragraph{}The leadership contest that followed the departure of Harold Wilson in 1976 clearly demonstrates the tensions between the various factions in the party. Callaghan won the support of the majority of the Parliamentary Labour Party and in his cabinet reshuffle preserved the strength of the party's right wing in cabinet. Michael Foot was elected deputy leader, a representative of the parties left-wing. \footnote{See \citename[pp.167-8 and pp.179]{Pelling:1993qy}}


\paragraph{}To alleviate tension in the unions-party link, the formers role in selection and in policy formulation was reduced. Kinnock's reforms firstly solved the political problems arising from the relationship between party funding and the unions and secondly democratised the perceivably undemocratic block votes of unions. Not to overstate the power of the unions, nor by using the monolithic term `the unions' to ``disaggregate'' discriminate representative bodies, \footnote{\cite{Ludlam:2003qy}}


\paragraph{}The challenge of the Conservatives was that Labour was an undemocratic party. As late as 1990, and right to the end of her term of office as Prime Minister and Leader of the Conservative party, Mrs Thatcher maintained her criticism of the mechanisms by which Kinnock came to be `elected' Leader of the Labour Party\ldots \begin{quote}``The Opposition's real reason [for bringing a vote of no confidence before the House of Commons] is the leadership election for the Conservative party, which is a democratic election according to rules which have been public knowledge for many years�one member, one vote. That is a far cry from the way in which the Labour party does these things. Two in every five votes for its leader are cast by the trade union block votes, which have a bigger say than Labour Members in that decision: precious little democracy there.''\footnote{\cite{Thatcher:1990fk}}\end{quote}



\paragraph{}What has been the New Labour project and how significant a departure with the past of the party does this represent? Has New Labour given the party a neo-modus operandi? what does this mean for the party? 

\paragraph{}Populist rhetoric, appealing direct to the British public. Redefinition of policy, re-prioritizing of values. 

\paragraph{}Dennis MacShane, Labour MP for Rotherham, writing in defense of Blair and New Labour, highlights the contrast of the positive social change his stereotypical Northern, working-class constituency has seen under a Labour Government:
\begin{quote}A decade ago, my South Yorkshire constituency posted job adverts offering wages of \pounds1.20 an hour. Workers had no statutory paid holidays. Pensioners had to choose between heating and eating. Social housing and schools had seen no refurbishment in two decades. Gay people could not enter into legal partnerships. There were no Muslims in parliament.\end{quote}

\paragraph{}Yet there are substantive breaks with the past of the party. My examination has found that New Labour represents a significant departure from the party's past in the following ways. Firstly democracy is now for its own sake, as a means rather than an end. Second, class and ownership are no longer central to policy.

