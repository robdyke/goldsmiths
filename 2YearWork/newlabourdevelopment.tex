\documentclass[13pt]{article}
\usepackage{geometry}
\usepackage{harvard}
\usepackage{helvet}
\usepackage{setspace}
\renewcommand{\familydefault}{\sfdefault}
\geometry{a4paper}
\title{To what extent did the development of New Labour between 1994 and 1997 represent a decisive break with the party's past?}
\author{Reg No 22164733}
\date{\today}

\begin{document}
\begin{onehalfspace}
\bibliographystyle{kluwer}
\maketitle

\paragraph{}This essay will examine the history of the Labour Party in order to question the significance of the New Labour project. ``New Labour'' was a brand developed by the Labour Party during the early 1990s to sell its self to the electorate. Corporations brand and redesign products to increase sales, yet a Mars bar remains a Mars bar. For a political party the goal is not unit sales but votes.\footnote{``Elected Every Day'', Chapter 1 in \cite{Cohen:2003fj}} The development of New Labour changed both the ideology and policy of the Labour Party, the substance not simply the packaging, to increase the electoral appeal of the party. That said, the development of New Labour needs to be understood in the context of a longer narrative than that framed in the question; revisionism has long been crucial in the history of the Labour Party. My first task is to demonstrate the tradition of revisionism within the Labour Party. Secondly I will look at two areas where revision and redefinition has taken place; in the values and program of the party and the party organisation. I will then demonstrate aspects of the development of New Labour and the extent of revisionism as part of this project. My argument is, firstly, that the revisionism of New Labour does represent a significant departure from the past for the party. Secondly, that this departure needs to be recognised as part of a tradition of Labour revisionism.

\paragraph{}In the foundation of the Labour Party are disaffected liberals, Fabian socialists and communists who, against a backdrop of class stratification and rapid industrialisation, joined with unionising workers and others seeking greater political participation and representation, to form the Independent Labour Party (I.L.P.) in 1906. \footnote{\cite{Pelling:1993qy}} From the inception of the I.L.P., socialist and liberal ideas were drawn upon for party policy. The socialist current within the I.L.P., and later the Labour Party, concerned its self with redistributive policy; policy entirely consistent with the heritage of this political strand. For the liberals in the I.L.P. participation and representation, and much later exclusion, were dominant themes. Each current, liberal and socialist, was broadly concerned with the welfare of the now more clearly identifiable `working class' in Britain. 

\paragraph{}It is this vibrant mix of political visions that has led to the description of the Labour Party as a ``party of values''\footnote{\cite[pp. 1]{nlsoul2004}} and a ``non-ideological party''\footnote{\cite[pp. 261]{Mitchell:2004fr}}. The different themes of this broad leftist libertarian political organisation lead to a tension between the `left' and the `right' of Labour, reflecting the principle political currents that form it: socialism and liberalism. Unionism and the special relationship that organised labour has enjoyed with the party has been both a strength and a weakness for both sides. These tensions have been the catalysts for revisionism and the redefinition, or reprioritisation of values at many times in the party's past.  As to the extent of any `break with the past' that the development of New Labour could represent\ldots

\begin{quotation}
``\ldots Part of the continuity or otherwise of a political party, if it is to be seen in terms of the goals and values of that party, will depend crucially on whether there is also continuity about how those values have been ordered and prioritised over time \ldots We not only have to get some analysis of the conceptions of values employed by a party but also how these values are ordered during the life of a political party or movement'' \ldots \footnote{\cite[pp. 116]{Plant:2004fj}}
\end{quotation}

\paragraph{}I will now consider two periods of post-war life of the Labour Party in order to demonstrate three things. Firstly that revisionism has long been a feature of the Labour Party and secondly, that revisionism always represents a departure from the immediate past of the party. Finally my examination aims to plot the course of reform of the Labour party and the utility of these reforms to New Labour. Both periods show Labour in opposition as the party's left-right ``battles'' are ``bigger and noisier in opposition than in power and [are] dominated by reactions to its record [in Government]''.\footnote{\cite{Mitchell:2004fr}}

\paragraph{}Labour found themselves loosing seats at the 1950 election, a decline that was not stemmed: a 1.1 percent swing in the electorate gave the Conservatives a majority of 17 seats and put Labour out of government in 1951.\footnote{\cite{Pelling:1993qy}} Post-war revisionism within the Labour party began principally as a response to loosing the 1951 General Election and demonstrates the tension between politics and policy for a political party. There were those who thought the Labour party had gone too far with social and industrial reforms (i.e. was too socialist) and were being rejected by the electorate, and there were those who felt the party had been too timid with reform (i.e. not socialist enough), again being rejected by the electorate.\footnote{\cite[pp. 9] {Arblaster:2004mz}}

\paragraph{}Hugh Gaitskell was the Chancellor in the 1951 Government who had introduced charges to the `free at the point of need' National Health Service. In opposition Hugh Gaitskell had further reforms in mind. He became leader of the party in 1955, beating the leading figure of the left of the party, Bevan, in the contest. While leader of the Labour Party Gaitskell wanted to revise Clause IV of Labour Party constitution.  Clause IV of the Labour constitution is totemic to the Left because it signifies the socialism in the party in its proclamation of the party's purpose:

\begin{quotation}
``To secure for the workers by hand or by brain the full fruits of their industry and the most equitable distribution thereof that may be possible upon the basis of the common ownership of the means of production, distribution, and exchange, and the best obtainable system of popular administration and control of each industry or service.''
\end{quotation}

\paragraph{}This ignited fierce debate in the party, with Bevanites\footnote{Supporters of Bevan} and others from the `Keep Left' group fighting to defend this article, and the Party, from what they saw as further redefinition by Labour's Right. The inter-party conflict caused by the distance between the leadership and the membership in the early sixties over Gaitskell's plans for the party's constitution and the perceived shift to the right that revision of this article would represent was too great to gain majority support from the different power bases in the party. Clause IV remained as cited and continued to be a totem of Left/Right debate over the next 35 years.

\paragraph{}\possessivecite{crossland1956} \underline{The Future of Socialism} was influential on the thought of the social democratic Right of the Labour Party in the late 1950's. And his redefinition of socialism is also salient to discussions on the development of New Labour in the early 1990s. Crossland's politics are based on a acceptance of capitalism as the surest way of generating wealth in a society, wealth which can then be redistributed to achieve greater social justice. Post-war capitalism was no longer understandable in homogenous class terms for Crossland: the need for socialism in the Marxist class-based sense had diminished. Nationalisation had diminished private ownership and dispersed economic power. The growth of unionism and of democratic institutions had diminished the power of private capital and ensured the interests of the workforce were taken into account. The rise of Keynesian economic policy, that is, the management of capitalism by government intervention in the demand side of the economy, and its success in sustaining the post-war social settlement\footnote{What some have referred to as the Post War Consensus. See \cite[Chapter 2.]{Kavanagh:1997rt}} assured social democrats that capitalism could be managed in the interest of greater social justice.\footnote{See \citename{Plant:2004fj}, especially pp. 110-111 for fuller elaboration of Crossland's argument}

\paragraph{}For Crossland and other revisionists on Labour's Right of the 1950s and 1960s, democratic socialism was to use democracy to achieve social justice within the capitalist economy, not seek alternatives to the economic system. The Labour Party had traditionally looked at democracy as a goal to be realised for the unrepresented and excluded. Crossland observed that universal suffrage and a stable majoritarian democratic system\footnote{In which the Labour Party had even formed a government as the majority party within 50 years of the party's birth!} as providing the basis for a transition from `democracy' as an `end' for social democrats: `democracy' now provided the `means' for achieving other policy `ends'.

\paragraph{}These two examples of the redefinition and reprioritisation of the values of the Labour Party demonstrate the long history of Labour revisionism. The waxing of the right of the party and the waining of the left while in opposition in the 1950s demonstrates the tensions between left and right in the party, the development of Labour's political thought from socialist to social democrat and a break from the parties past. However, as the unsuccessful revision of Clause IV shows, this break was not as significant as it might have been. Yet Labour's goals changed and the values of the Party were reprioritised during the 1950s, with a more `social democrat' than `socialist' identity emerging.

\paragraph{}To further demonstrate Labour revisionism and to bring us closer to the development of New Labour, I move now to the period following the 1979 General Election defeat.

\paragraph{}Kinnock wanted to make Labour electable again after the 1984 election defeat. Although this was not realised while he was leader of the party, the organisational changes that he implemented during his tenure as party leader were crucial in returning Labour to power in 1997. He led reforms of the both the organisation and the democratic structures for accountability within the Labour Party. Kinnock's reforms had two goals: to reduce the role of activists and unions, firstly in the re-selection of sitting MPs and secondly in the formulation of policy. This would loosen the grip of activists and other blocks within the party on policy and strengthen the voice of the ordinary membership of the party. Through new internal democratic processes such as `One Member One Vote' (OMOV) and a re-weighting of votes between constituent parts of the party's electoral college, a clear balance was made towards the party's leadership for policy formulation. A remodeled conference debated and endorsed (or otherwise) policy produced by extra-conference groups replacing the ill-informed and ineffectual last-minute conference battles of the 1970s.

\paragraph{}`One Member One Vote' is an example of the move to popular enfranchisement within the Labour Party. For some, e.g. \citename{Shaw:1994gf} and \citename{mairnlr2}, this move demonstrates a shift to populist from pluralist politics and contributes to an oligarchical concentration of power in the party leadership. The structures of representative democracy and accountability were effectively dismantled by Kinnock. Direct democracy became normative intra-party vis-a-vis policy.\footnote{See chapter 5 in \cite{Shaw:1994gf}}

\paragraph{}For some commentators, Kinnock's legacy was of a profound change to the party in the wake of significant organisational modernisation. From the ``highly pluralistic, deeply polarised Party characterised by the institutionalised disperal of powers and weak central authority'' he inherited, a ``powerful central authority exercising tight control over all aspects of organisational life'' dominated the party by the time of his departure in 1992.\footnote{\cite[pp. 120]{Shaw:1994gf}}

\paragraph{}Kinnock's organizational changes resulted in New Labour politics that represent a significant departure from the Labour Party of the 1970s. Then, the voice of activists, local associations, individual and blocs of unions were heard, often dissented from the Parliamentary Labour Party and the Leadership. Although it was ``Kinnock and the Modernisers'' of the 1980s who had done the hard work, rebalancing the various groups within the party, forging new relationships and moving the party to the centre ground of British politics. The New Labour project used these new organisation structures during their hegemonic revisionism of the party. Mair argues strongly that the organizational changes in the party to increase the democratic participation of the ordinary and increasingly differential members has \emph{depoliticized} the party by marginalizing the traditional sources of political thought. Internal party dissent has been eliminated by \begin{quote}``\ldots marginalizing representative procedures inside the party, introducing plebiscitarian techniques, going over the heads of the party conference and the activist layer in favour of widespread membership ballots.'' \dots ``New Labour's drive to centralize the party and ensure that it speaks with one voice at every level - on the ground, as well as in the various representative assemblies - may be intended not to strengthen the party but to marginalize it.''\footnote{\cite[pp. 22 \& 26]{mairnlr2}}\end{quote}

\paragraph{}Tony Blair succeeded where Kinnock had failed. The New Labour project completed what Gaitskell had started. In the first case, Blair ensured that OMOV was installed as the ratification mechanism for party policy and used this mechanism to gain popular mandate for the 1997 General Election manifesto. Secondly, and as a redirect result of OMOV, New Labour broke from the party's socialist past in one decisive way: the membership endorsed the rewriting of Clause IV of the party's constitution, conclusively redefining and re-prioritising the values of the party. The re-written Clause IV emphasizes a commitment to the social, but not to social\emph{ism}. In place of collectivism (a previous policy ``end'') is a faith in democratic means ``a community in which power, wealth and opportunity are in the hands of the many, not the few.'' \footnote{\cite[website]{Labour:2007yq}}

\paragraph{}So how is New Labour to be understood? To what extend does it make a decisive break with the past for the party?
\begin{quote}
``In power New Labour have addressed themes conventionally associated with socialism: social exclusion, child poverty, urban regeneration, the life chances of the disadvantaged, the educational attainment of children from poorer homes. And have not addressed such themes as income distribution, marketplace inequalities. Has it none the less left a mark on the social order comparable with its post-Atlee predecessors?'' \footnote{\cite[pp. 271]{Toynbee:2004fk}}
\end{quote}

\paragraph{}This point of view demonstrates that New Labour has continued to tackle social inequality and discrimination problems with the instrument of democratic government, indicating perhaps a New Labour, New Left analysis. However, as \citename{Toynbee:2004fk} point out, there are other readings of New Labour; as the New Right? The failure of New Labour to address income distribution and marketplace inequalities demonstrates an echo of Gaitskell and Crossland. Furthermore, the fiscal policy of the iron chancellor Brown is ostensibly neo-Kenyesian. While initially pledging ``stick to planned public spending allocations for the first two years of office'', \footnote{\cite[Election Manifesto]{Labour:kx}} illustrating a degree of political expediency in diminishing the ``tax and spend'' history of the party and appealing to the middle-classes, Labour also planned significant real increases in public spending during the development of New Labour and committed to them in the 1997 manifesto. New Labour's emphasis on the autonomy of individuals, on direct democracy and on economic prudence \emph{within} the capitalist system reinforces the New Labour as `New Right' point of view.

\paragraph{}The New Labour as New Left argument would demonstrate a linkage to the recent leftism of Kinnock and Foot. While New Labour certainly developed new practices of politics, updated for an internet connected consumerist age, in terms of \emph{policy} the development of New Labour would not represent a decisive break with the parties immediate past, nor its socialist roots. Understanding New Labour as the New Right of the party, does illustrate a contrast with the immediate (1980s) past for the party, yet at the the same time this analysis returns us to the rightist revisionism of the 1950s, Gaitskell and Crossland, indicating congruence with the more distant past. Understanding New Labour's revisionism in this way clearly demonstrates my view of a long tradition of revisionism within the party. My search for a decisive break with the Party's past in terms of values and the prioritisation of these values has found a perpetual motion in the political thought of the Labour party.

\end{onehalfspace}
\newpage
\bibliography{globalbib}
\end{document}