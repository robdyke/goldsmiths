\newpage
\chapter{Hacking Communicative Capitalism.}
\paragraph{}In the previous sections I considered the networked society paradigm, the philosophy of free/libre and the practices of the open-source movements. I now turn to explore how a free/libre political economy of social production and reproduction might challenge the dominant liberal `free-market' political economy of late-stage capitalist market economies.

\paragraph{}The dynamic between free-software and open-source is often misunderstood by social and political theorists. As a consequence it is also under-theorised within socio-political theory. For example Terranova's \textit{Network Culture: Politics for the Information Age}\footnote{\cite{Terranova:2004ly}} has but a few notes on  `Software: freeware \& open-source' but none on free-software. This is a disappointing omission because it is an opposition to, indeed, the political act of resistance to, enclosure that created the free-software philosophy in our information age. As Prug observes in his examination of the `political act' of free/libre activity, Hardt and Negri manage two references to the open-source movement in \textit{Multitude}.\footnote{\cite[`Hackers and the Protestant ethics']{Prug:2007fs} and \cite[p. 301 \& pp. 339-40]{Hardt:2005zt}} Terranova and Hardt \& Negri, perhaps willingly, omit the origins of open-source in free-software and as a result miss the challenge to capitalist property rights that free-software presents. To address these failings, in this section I consider questions of production and reproduction under the conditions of `communicative capitalism'\footnote{After \cite{dean:2005cc}, \textit{The Networked Empire: Communicative Capitalism and the Hope for Politics}} in search of `positive possibilities'\footnote{After \cite{Hardie:2005px}, \textit{Change of the Century: Free Software and the Positive Possibility}} presented by free/libre and commons conceptions of the same.

\paragraph{}In \textit{Empire} Hardt and Negri, like Marx before them, demonstrate that capital has an expansive character. Just as the hacker sees no boundary to what may be hacked, capital similarly has no boundaries, expanding into culture, science, and sex. The expansion and increasing sophistication of production possibilities, created from nature by hacking, necessarily increases the sophistication of the capitalist system. The contemporary economic paradigm has been variously described as a digital economy, an information economy and a weightless economy. Yet I think that Dean's concept `communicative capitalism' most accurately describes the complexity of this conjuncture, by properly retaining a reference to `captialism'!

\paragraph{}Dean's term highlights the economic reconfiguration of communication, referring to the networked movement of information and the production of affect, in a technologically mediated society.\footnote{\cite[pp. 272-273]{dean:2005cc}} In conceiving of capitalism as communicative in this way, Dean brings to the foreground the social and technical aspects of informational capitalism in networked society. By social I am referring to the affective dimension of production and reproduction, taking a biological turn similar to Hardt and Negri in their use of Foucault's concept of biopower.\footnote{\cite[p. 28]{Hardt:2001jl}} The `biological turn' can thought of as a ``techno-scientific reconceptualisation of life''\footnote{\cite[p. 101]{Terranova:2004ly}} away from the hierarchical structures and the centralisation of power of modernity's oligopolistic Cathedral model of organisation and production, towards more ad-hoc, rhizomatic network conceptions, such as the Bazaar.

\paragraph{}By technology, like Appadurai, I am referring to technology high and low, both physical and informational technologies. Similarly, Barry demonstrates that global regimes, such as ISO certifications and international `intellectual property' regulation are equally technological as more conventional understandings of technology, such as equipment and scientific knowledge.\footnote{\cite[p. 34]{Appadurai:1996lp}. See also \cite[Chapter 2, \textit{Technological Zones}]{Barry:2001ff} for a discussion of the `content' of Appadurai's technoscapes} Dean's term `communicative capitalism' reflects this understanding of technology. It reveals the multiple sites of contestation and diverse technologies of mediation in ``rhizomatic communications networks that are themselves biopolitical, generative, productive, of capital, of subjectivity, of life itself.''\footnote{\cite[p. 285]{dean:2005cc}}

\paragraph{}Crucial to my argument here are the concepts of abstract labour and of immaterial production. These post-workerist terms move beyond the over-determined political subject of the industrial proletariat and describe the biopolitical nature of production and re-production under communicative capitalism.\footnote{See \cite[Chapter 3 \textit{Free Labour}]{Terranova:2004ly} and also \cite{Wright:2005wb}} Using these categories, we can see that all collective labour is channelled and structured within the logic of capital.

\paragraph{}The informatisation of cultural labour and production, and of that information then becoming property under the conditions of communicative capitalism, tends towards the increasing indistinguishability of economic and cultural phenomena. The publishing of creative works as commodities and the digital distribution of these works blurs the markers of the industrial production paradigm. From abstract labour and immaterial production similarly abstract property forms emerge, such as `intellectual property' and patents, accompanied by copyright regimes to protect these new property forms. With the modern site of rights, and of rights protection, diminished in our networked society the defence of the rights of the holders of these new property forms becomes increasingly problematic. The battle for protection of these abstract property forms becomes global, with communicative capitalism accelerating the expansion and normalisation of transnational structures of `governance without governing', such as the World Trade Organisation (WTO), the World Intellectual Property Organisation (WIPO) and the Organisation Internationale de Normalisation (ISO).\footnote{The International Organization for Standardization is non-governmental transnational organisation that works to globally normalise technical standards.}

\paragraph{}The WIPO understand `intellectual property' as referring to ``creations of the mind: inventions, literary and artistic works, and symbols, names, images, and designs used in commerce.''\footnote{WIPO website, \textit{What Is Intellectual Property}, \url{http://www.wipo.int/about-ip/en/}, Accessed 26th April 2008.} For Stallman and the FSF, ``the term ``intellectual property'' carries a hidden assumption - that the way to think about all these disparate issues [copyright, patents, trademarks] is based on an analogy with physical objects, and our ideas of physical property.''\footnote{\cite{Stallman:2008wa}} Johnson is critical of the FSF position, insisting that the ``word `intellectual' is used as a prefix to `intellectual property' precisely to distinguish it from material or real property.''\footnote{\cite{Johnson:2004wa}} This definitional argument doesn't necessarily require resolution, as it is serves me here just to highlight the problematic relationship between culture and property in advanced capitalist societies. If culture is understood as art, science, code, the radical diversity and interdependence of all human knowledge, then the enclosure of knowledge, its commodification as exchangeable property, deprives humans of access to it.

\paragraph{}The hegemony of liberal political economy models in the realms of technology and culture is demonstrated by the spread of international patent protection and `intellectual property' law. The radicalism of free/libre concepts in transmission to non-software realms, particularly to virtual and abstract labour, but also to cultural production in general, have created the revolutionary abstractions of copyleft, a hack of contemporary capitalism's notion of copyright. It has also created creative-commons, a hack which institutionalises a opening dynamic of sharing against the closure of patents and licensing.\footnote{\cite[17m32sec]{moore:2002rv}} Copyleft and the creative-commons subvert the regimes of copyright and `intellectual property', reusing the main symbols of the latter pair of concepts to make public a new logic.\footnote{See The GNU Project, \textit{What is Copyleft?}, \url{http://www.gnu.org/copyleft/copyleft.html} and Creative Commons, \textit{Frequently Asked Questions}, \url{http://wiki.creativecommons.org/FAQ}. Both accessed 27th April 2008}

\paragraph{}The free/libre hacks of copy-left and creative-commons are based on an appeal to the rule of law, demonstrated by the constitutional nature of the Free Software Definition and the codification of expressions of the FSD in various licenses.\footnote{See \cite[Chapter 1, \textit{Open Source Licensing, Contract, and Copyright Law}]{Laurent:2004}} The appeal to the rule of law of much of free/libre discourse demonstrates the neo-liberal character of free/libre. How does this mean we should understand free/libre's first principle, freedom? For Hardie, the freedom espoused by the FSF in teh FSD is a freedom ``bound intimately with the logic of open democracy and with free and open markets.''\footnote{\cite[p. 2]{Hardie:2005px}} On this reading, the political economy of free/libre, is subversive only because it appears counter-intuitive. That is to say it appears paradoxical within the axioms of the capitalist market to produce only to then give away what is produced.

\paragraph{}Prug's work proposes a broader reading of free/libre philosophy as radical hack of the regime of rights in general. His approach demonstrates a true positive possibility in the free/libre hack of communicative capitalism. By hacking rights in favour of egalitarianism over meritocracy, the property rights central to liberal capitalism are changed from the substantive `right to' and the normative `right of' property to a `right in' property, which is both substantive and normative. As Prug argues, ``Stallman re-conceptualised the idea of rights to encourage [a] volunteer, co-operative and decommodified society with the notion of shared wealth.''\footnote{\cite{Prug:2007fs}}

\paragraph{}Influenced by the openness of free/libre philosophy the idea of Open Capital\footnote{See \url{http://www.opencapital.net/}} re-conceptualises other logics central to communicative capitalism in a radical hack of the terms `profit' and `loss'. In a departure from the ``competitive economy based upon shareholder value and unsustainable growth results from a transfer of risks outwards, and the transfer of reward inwards,''\footnote{\cite[p. 17]{cook:2004cp}} Open Capital neutralises the conflict created by that centrifugal/centripetal strategy, by hacking `profit' and `loss' to `reward' and `risk', as defined by all participants in the network, to result in a mutually satisfactory exchange of value from the productive activity.

\paragraph{}In conclusion, free/libre presents opportunities for reconfiguring the logic(s) of communicative capitalism. Now I return to the political question of organisation and resistance, of the wider translation of the free/libre hacks of political categories and concepts.
